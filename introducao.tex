\chapter{Introdução}
\label{cap:introducao}

\setcounter{page}{12}

	De onde vieram os autovalores? O que são? Por que são importantes? Muitos estudantes de ciências exatas e engenharia fazem essas três perguntas durante as disciplinas introdutórias de Álgebra Linear. Grandes nomes da matemática se debruçaram sobre essas questões, como D'Alembert, Lagrange, Laplace, Sturm e Cauchy, culminando no final do século $\mathsf{XIX}$ na Teoria Espectral das Matrizes \cite{Hawkins75}.
	
	O estudo dos autovalores foi importantíssimo na Física dos séculos $\mathsf{XVIII}$ e $\mathsf{XIX}$. A primeira aparição do que hoje é chamado de autovalor aconteceu em 1743. Estudando o problema de várias massas ligadas umas às outras por molas, D'Alembert chegou a um sistema de equações diferenciais. Ao fazer algumas transformações de variáveis ele foi capaz de reduzir o estudo a apenas uma equação:

\begin{equation}\label{eq:EDO1}
	\frac{d^2u}{dt^2} + \lambda u = 0,
\end{equation}
sendo $u$ uma soma envolvendo o produto das velocidades e posições de cada massa, e $\lambda$ um escalar. D'Alembert aplicou o novo método a sistemas com duas e três massas ($n = 2$ ou $n = 3$) e, com argumentos relacionados à Física do problema, afirmou que $\lambda$ só poderia ser real. Até hoje usa-se a letra grega $\lambda$ para designar os autovalores.

	A partir de então, diversos matemáticos
	
	Paralela e independentemente, Euler, também em 1743, chegou à solução geral de equações com a forma da equação \ref{eq:EDO1}, mas D'Alembert não soube desse resultado. Foi por isso que apenas em 1750, quando tomou conhecimento dos resultados de Euler, ele retomou seu trabalho e verificou que as soluções para \ref{eq:EDO1}, aplicadas especificamente ao sistema massas--molas, eram do tipo

\begin{equation}
	u(t) = g\mathsf{e}^{-\lambda t},
\end{equation}
onde $g$ é um escalar e depende das condições iniciais do problema. Ou seja, a partir de agora $\lambda$ não é apenas um escalar utilizado como artifício matemático para simplificação de equações, mas está intimamente ligado à solução do problema físico em si. D'Alembert aplicou as soluções novamente para dois e três corpos. Oito anos depois, em 1758, ele descobre que $\lambda$ está relacionado à estabilidade do sistema.

	Lagrange estende o trabalho de D'Alembert tratando de um problema um pouco mais geral, e encontra a solução para qualquer $n$. Seu modelo envolve novos coeficientes $A_{ik}$, e Lagrange demonstra que $A_{ik}$ e $A_{ki}$ são equivalentes. Em notação matricial moderna, diríamos que o problema não muda para a matriz simétrica $A = {A_{ik}}$ e sua transporta $A^\dag = {A_{ki}}$. Como $A$ era simples, Lagrange foi capaz de escrever a equação polinomial característica, mas naquele tempo nada se sabia sobre a natureza de suas raízes.
	
	

História.

Importância no passado.

Importãncia no presente.

Entre eles estão 
	
	 e até no algoritmo \emph{PageRank}, base do mecanisno de busca do Google \cite{BrinPage98}. 
	
Nature, Science, Plos ONE.

Redes complexas.

Espectro do grafo.

Informações sobre topologia.

Redes Sociais.

Milhões de Nós (Big Data).

Milhões de Autovalores.

Novos métodos são necessários.

Computação de Auto Desempenho.

Nandy 2004.

GA é importante por si só.

História do GA.

Importância no passado.

Importância no presente.

Nature, Science.

GA é intrinsecamente paralelo.

Potencial lambda + HPC: tratar grandes problemas.

Objetivo dentro do contexto.


Não há fórmula para as raízes de um polinômio de grau $n > 4$. \cite{Pan97}.

Matriz \emph{companion} do polinômio. A Frobenius c

https://en.wikipedia.org/wiki/Companion\_matrix

No periódico Plos ONE (fator de impacto em 2014 de 3,23), há 275 artigos que envolvem diretamente o cálculo de autovalores. 

Aqui devo escrever sobre a importância e relevância do estudo. 