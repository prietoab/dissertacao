\chapter{Quociente de Rayleigh ($\rho$)\label{cap:algebra}}

O problema dos autovalores e autovetores pode ser definido brevemente da seguinte maneira: dada uma matriz $\mathsf{A}$ $n$ por $n$, o escalar $\lambda$ é chamado de \textbf{autovalor} de $\mathsf{A}$ se existe um vetor \textbf{\texttt{u}} não nulo tal que

\begin{equation}\label{eq:autovalor}
	\mathsf{A} \textbf{\texttt{u}} = \lambda \textbf{\texttt{u}}.
\end{equation}

O vetor \textbf{\texttt{u}} é chamado de \textbf{autovetor} de $\mathsf{A}$ associado a $\lambda$. Reescrevendo a equação \ref{eq:autovalor} chegamos a 

\begin{equation}
		\begin{array}{c}
			\mathsf{A} \textbf{\texttt{u}} - \lambda \textbf{\texttt{u}} = 0 \\
			(\mathsf{A} - \lambda \mathsf{I})\textbf{\texttt{u}} = 0,
		\end{array}
\end{equation}
onde $\mathsf{I}$ é a matriz identidade. Pela teoria das equações lineares, a equação acima só tem soluções se

\begin{equation}\label{eq:det}
	\mbox{det}(\mathsf{A} - \lambda \mathsf{I}) = 0.
\end{equation}

A equação \ref{eq:det} leva a um \textbf{Polinômio Característico} de ordem $n$, portanto, $\mathsf{A}$ pode ter até $n$ autovalores.

Seja $\mathsf{A}$, a partir de agora, uma matriz auto$-$adjunta. Todos os seus autovalores $\lambda_i$ são reais e, consequentemente, podem ser ordenados do menor para o maior:

	\begin{equation}\label{eq:autovalores_ordenados}
		\lambda_1 \leq \lambda_2 \leq \cdots \leq \lambda_n.
	\end{equation}

Para $\mathsf{A}$, que opera sobre os vetores \textbf{\texttt{u}} do espaço euclidiano $\varepsilon^n$, define-se o \textbf{Quociente de Rayleigh} $\rho (\textbf{\texttt{u}})$:

\begin{equation}\label{eq:rho}
	\rho (\textbf{\texttt{u}}) \equiv \rho(\textbf{\texttt{u}}; \mathsf{A}) \equiv \frac{\textbf{\texttt{u}}^\dag \mathsf{A}\textbf{\texttt{u}}}{\textbf{\texttt{u}}^\dag \textbf{\texttt{u}}},   \textbf{\texttt{u}} \neq 0,
\end{equation}
onde $\textbf{\texttt{u}}^\dag$ é o complexo conjugado de $\textbf{\texttt{u}}$. Todo vetor \textbf{\texttt{u}} possui um $\rho(\textbf{\texttt{u}})$.

	O Quociente de Rayleigh está diretamente relacionado aos autovalores e autovetores. Suponha que $\textbf{\texttt{w}}_i$ seja um dos $n$ autovetores de $\mathsf{A}$. A equação \ref{eq:rho} fica
	
	\begin{equation}\label{eq:rho_no_w}
		\rho(\textbf{\texttt{w}}_i) = \frac{\textbf{\texttt{w}}_i^\dag \mathsf{A} \textbf{\texttt{w}}_i}{ \textbf{\texttt{w}}_i^\dag \textbf{\texttt{w}}_i} = \frac{\textbf{\texttt{w}}_i^\dag  \lambda_i \textbf{\texttt{w}}_i}{\textbf{\texttt{w}}_i^\dag \textbf{\texttt{w}}_i} = \frac{ \lambda_i \textbf{\texttt{w}}_i^\dag  \textbf{\texttt{w}}_i}{\textbf{\texttt{w}}_i^\dag \textbf{\texttt{w}}_i} = \lambda_i,
	\end{equation}
	ou seja, o quociente de Rayleigh de um autovetor é o autovalor associado:
	
	\begin{equation}\label{eq:rho_no_w_eh_lambda}
		\rho (\textbf{\texttt{w}}_i) = \lambda_i.
	\end{equation}
	
	Portanto, se temos um autovetor de uma matriz auto$-$adjunta, não é necessário resolver o sistema linear \ref{eq:autovalor} para obter o autovalor, basta efetuar as multiplicações de \ref{eq:rho}.
	
	O vetor gradiente de $\rho$ é \cite{Wilkinson1965}
	
	\begin{equation}\label{eq:gradrho}
		\nabla \rho (\textbf{\texttt{u}}) = \frac{2[\mathsf{A} - \rho(\textbf{\texttt{u}})] \textbf{\texttt{u}}}{\textbf{\texttt{u}}^\dag \textbf{\texttt{u}}}.
	\end{equation}

	Ele é nulo se, e somente se, $\textbf{\texttt{u}}$ é um autovetor $\textbf{\texttt{w}}_i$  de $\mathsf{A}$:
	
	\begin{equation}\label{eq:grad_rho_no_w}
		\nabla \rho (\textbf{\texttt{w}}_i) = \frac{2[\mathsf{A} - \rho(\textbf{\texttt{w}}_i)] \textbf{\texttt{w}}_i}{\textbf{\texttt{w}}_i^\dag \textbf{\texttt{w}}_i} = \frac{2[\mathsf{A} - \lambda_i]\textbf{\texttt{w}}_i}{\textbf{\texttt{w}}_i^\dag \textbf{\texttt{w}}_i} = \frac{2[\mathsf{A}\textbf{\texttt{w}}_i - \lambda_i\textbf{\texttt{w}}_i]}{\textbf{\texttt{w}}_i^\dag \textbf{\texttt{w}}_i} = \frac{2[0]}{\textbf{\texttt{w}}_i^\dag \textbf{\texttt{w}}_i} = 0,
	\end{equation}
	
	\begin{equation}\label{eq:grad_rho_nulo}
		\nabla \rho (\textbf{\texttt{w}}_i) = 0.
	\end{equation}

Além disso, para todos os vetores \textbf{\texttt{u}} $n-$dimensionais diferentes de zero, $\rho(\textbf{\texttt{u}})$ é limitado no intervalo [$\lambda_1, \lambda_n$] entre o menor e o maior autovalor da matriz $\mathsf{A}$ \cite{Parlett1998}. Em seguida agrupo as três propriedades de $\rho(\textbf{\texttt{u}})$ citadas acima.

\textbf{Algumas propriedades de $\rho(\textbf{\texttt{u}})$}

\begin{enumerate}
		
	\item Se $\textbf{\texttt{w}}_i$ é um autovetor, $\rho(\textbf{\texttt{w}}_i) = \lambda_i$ .
	
	\item \textbf{Fronteira}: $\rho(\textbf{\texttt{u}})$ é limitado no intervalo [$\lambda_1, \lambda_n$].
	
	\item \textbf{Estacionaridade}: o gradiente de $\rho(\textbf{\texttt{u}})$ é nulo se $\textbf{\texttt{u}}$ é um autovetor.
	
\end{enumerate}

	A segunda e terceira propriedades podem ser utilizadas para tratar o problema dos autovalores como um problema de otimização matemática. A propriedade de fronteira, por exemplo, permite transformar o problema de encontrar o menor (ou maior) autovalor de uma matriz auto$-$adjunta em um problema de minimização (ou maximização) de $\rho(\textbf{\texttt{u}})$. A estacionaridade pode, inclusive, auxiliar na definição da função objetivo. Minimizar $\nabla \rho(\textbf{\texttt{u}})$ também leva a um autovetor, entretanto, note que, quando $\nabla \rho(\textbf{\texttt{u}}) = 0$, $\textbf{\texttt{u}}$ pode ser qualquer um dos $n$ autovetores ($\textbf{\texttt{w}}_1, \textbf{\texttt{w}}_2, \cdots, \textbf{\texttt{w}}_n$) de $\mathsf{A}$. Ou seja, não há garantia que o autovalor obtido é o menor ou maior.

		A primeira propriedade permite que a obtenção dos autovalores seja transformada em um método de busca de autovetores. O espaço de busca é o conjunto de todos os vetores $\textbf{\texttt{u}}_i$, e o espaço de soluções é composto pelos $n$ autovetores ($\textbf{\texttt{w}}_1, \textbf{\texttt{w}}_2, \cdots, \textbf{\texttt{w}}_n$).
		
		Nesta dissertação estudei um método que cria um Algoritmo Genético para fazer essa busca. Ele faz uso das três propriedades.