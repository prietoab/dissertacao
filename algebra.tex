\chapter{Álgebra Linear\label{cap:algebra}}

%=============================================================================
\section{Vetores}
%=============================================================================
Vetores são objetos matemáticos representados geometricamente por setas. Possuem direção, sentido e magnitude (módulo). Algebricamente, são representados no espaço $\mathbb{R}^n$ por $n-$uplas ordenadas $X = (c_1, c_2, ..., c_n)$. Por exemplo, a dupla $\vec{w} = (3,4)$ é um vetor do espaço $\mathbb{R}^2$. Porém, a única informação que o escalares (3,4) nos dá é que o módulo do vetor $\vec{w}$ é $\| \vec{w} \| = \sqrt{3^2 + 4^2} = 5$.

Para sabermos qual é a direção e o sentido de $\vec{w}$, é necessária a definição de uma base. No espaço $\mathbb{R}^2$ a base comumente utilizada é formada pelos vetores unitários $\hat{x}$ e $\hat{y}$ (ou $i$ e $j$). Agora, $\vec{w}$ é a soma de dois vetores proporcionais aos vetores da base,

\begin{equation}\label{eq_vetorNaBase}
	\vec{w} = (3,4) = 3\hat{x} + 4\hat{y},
\end{equation}
tem direção de aproximadamente $53$ graus com o eixo $x$ e sentido do ponto $(0,0)$ para o ponto $(3,4)$.

Um vetor $\vec{v}$ qualquer do $\mathbb{R}^3$ utiliza os eixos $x$, $y$ e $z$, e é escrito como a soma de três vetores proporcionais aos versores $\hat{x}$, $\hat{y}$ e $\hat{z}$:

\begin{equation}
	\vec{v} = (\alpha,\beta,\gamma) = \alpha\hat{x} + \beta \hat{y} + \gamma\hat{z},
\end{equation}
onde $\alpha$, $\beta$ e $\gamma$ são números reais.

Generalizando para $\mathbb{R}^n$, um vetor $\psi$ é dado por

\begin{equation}
	\psi = (c_1, c_2, ..., c_n) = c_1\hat{x}_1 + c_2\hat{x}_2, \mbox{ ... }, c_n\hat{x}_n.
\end{equation}

Utilizando o somatório,  

\begin{equation}\label{eq:vetor}
	\psi = \sum_{p=1}^n c_p \hat{x}_p.
\end{equation},
onde o índice $p$ é tomado de $1$ até a dimensão $n$ do espaço vetorial.


Veremos adiante que o método estudado nessa dissertação utiliza os coeficientes $c_i$ da equação \ref{eq:vetor} para representar um vetor no algoritmo genético.

%=============================================================================
\section{Autovalores}
%=============================================================================	
	\begin{mydef}
	
		Seja $T: \mathbb{V} \rightarrow \mathbb{V}$ um operador linear. Um escalar $\lambda$ é chamado \textbf{autovalor} de $T$ se existe um vetor não nulo $\vec{v} \in \mathbb{V}$ tal que
		
		\begin{equation}\label{eqAA}
			T(\vec{v}) = \lambda \vec{v}
		\end{equation}
		
		Um vetor não nulo e que satisfaça a equação \ref{eqAA} acima é chamado de \textbf{autovetor} de T.
		
	\end{mydef}
	
	\vspace{1cm}
	
	O operador linear $T$	é representado, na prática, por uma matriz quadrada $H$, e aplicar $T$ em $\vec{v}$ significa multiplicar a matriz $H$ pelo vetor (coluna) $\vec{v}$:
	
	\begin{equation}
		H\vec{v} = \lambda \vec{v}.
	\end{equation}
	
	Reescrevendo a equação acima chegamos a 
	
	\begin{equation}\label{eqEqMatricial}
		(H - \lambda I)\vec{v} = 0,
	\end{equation}
	onde $I$ é a matriz identidade.
	
	Para que a equação acima possua solução não trivial, é necessário que
	
	\begin{equation}\label{eqDeterminanteZeto}	
		\mbox{det}(H - \lambda I) = 0.
	\end{equation}
	
	O cálculo desse determinante resultará na busca das raízes (reais ou imaginárias) de um polinômio com o mesmo grau da ordem da matriz $H$; cada raiz será um autovalor de $H$, e cada autovalor utilizado na equação \ref{eqAA} levará ao autovetor correspondente.
	
	\textbf{Exemplo} \cite{Santos2010}: determinar os autovalores do operador $T: \mathbb{R}^3 \rightarrow \mathbb{R}^3$ definido por $T(\vec{v}) = H\vec{v}$, para todo $\vec{v} \in \mathbb{R}^3$, em que
		
		\begin{equation}
			H=\left[\begin{array}{rrr}
								2		&		0		&		1		\\
								2		&		1		&		2		\\
								1		&		0		&		2								
							\end{array}\right]
		\end{equation}
		
		Para este operador o polinômio característico é
		
		$$
			p(\lambda) = \mbox{det}(H - \lambda I_3) = \mbox{det}\left[\begin{array}{rrr}
						2	-\lambda	&		0						&		1		\\
						2						&		1	-\lambda	&		2		\\
						1						&		0						&		2	-\lambda							
						\end{array}\right]
		$$
		
		\begin{equation}\label{eqPolCarac}
			p(\lambda) = (1 - \lambda)^2 (3 - \lambda)
		\end{equation}
	
	Igualar a zero a equação \ref{eqPolCarac} acima
	
		\begin{equation}\label{eqIgualAZero}
			p(\lambda) = (1 - \lambda)^2 (3 - \lambda) = 0
		\end{equation}
		leva às seguintes raízes:
		
		\begin{equation}\label{autovaloresExemplo}
			\lambda_1 = 1, \lambda_2 = 1 \mbox{ e } \lambda_3 = 3.
		\end{equation}
		
		Portanto, $\lambda_1 = 1, \lambda_2 = 1 \mbox{ e } \lambda_3 = 3$ são os autovalores de $H$.
		
		Obter os autovalores por meio do polinômino característico é prático para matrizes pequenas. Mas, para as de médio e grande porte, o custo computacional de lidar com grandes polinômios torna o cálculo inviável. Vários métodos numéricos mais eficientes $-$ e sofisticados $-$ foram desenvolvidos \cite{Wilkinson1965, Parlett1998}. E alguns utilizam propriedades de matrizes Hermitianas.

\section{Matrizes Hermitianas}

	---------------------------------------------
	
\section{Matrizes Simétricas}
	
	---------------------------------------------
	
\section{Quociente de Rayleigh}\label{sec:quocienteRayleigh}
	
	
	\begin{equation}\label{eq:rho}
		\rho_i = \frac{X_i^\dagger A X_i}{X_i^\dagger X_i}
	\end{equation}
	
	
	\begin{equation}\label{eq:grad_rho}
		\rho_i = \frac{2[H - \rho_i]C_i}{C_i^\dagger C_i}
	\end{equation}
	
	Página 12 do \cite{Parlett1998}:
	
	\begin{equation}
		\rho(u) \equiv \rho(u; A) \equiv \frac{u^* A u}{u^*y}, \mbox{    } u \neq 0.
	\end{equation}
	
	
	Página 13 do \cite{Parlett1998}. Fato 1.8:
	
	O quociente de Rayleigh possui as seguintes propriedades básicas:
	
	\textbf{Homogeneidade}:
	
	\begin{equation}
		\rho(\alpha u) = \rho(u), \mbox{   } \alpha \neq 0
	\end{equation}
	
	
	---------------------------------------------