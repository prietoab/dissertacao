%%*********************************************************
%% Essa versão corresponde ao padrão CCPG 001/2015.
%%
%% Ela foi adaptada do trabalho anterior de várias pessoas,
%% e mais detalhes podem ser encontrados abaixo.
%% 
%% Adriano Batista Prieto, 16 de outubro de 2015.
%% adrianoprieto@yahoo.com.br
%% *********************************************************
%Modelos para a edição de dissertações e teses
%
%Modelo LaTeX segundo o formato especificado na Informação CCPG 002/2013.
% http://www.prpg.unicamp.br/arqpdfnormas/infccpg002_2013.pdf
%
%Esse modelo para a edição de dissertações e teses foi adaptado da versão elaborada pelos alunos Daniel Guerreiro e Silva, Filipe Ieda Fazanaro e  Marcos Ricardo Covre da Faculdade de Engenharia Elétrica e de Computação
%
%http://www0.fee.unicamp.br/cpg/Modelos.html
%
%Modelo baseado no abntex2 disponível em https://code.google.com/p/abntex2/
%
%Instruções para instalação do abntex2 disponíveis em https://code.google.com/p/abntex2/wiki/Instalacao
%
%
%%
%% Faculdade de Tecnologia - Junho/2014.
% ------------------------------------------------------------------------

\input{preambulo}

% ---- compila o índice  ----
\makeindex
\makenomenclature
% ---

% ---- Início do documento ----
\begin{document}

% Retira espaço extra obsoleto entre as frases.
\frenchspacing

% ---- ELEMENTOS PRÉ-TEXTUAIS ----
\pretextual

%\pagenumbering{roman}

% 1. Capa
\imprimircapa
% ---

% 2. Folha de Rosto. (o * indica que haverá a ficha catalográfica) ---
\thispagestyle{empty}
\imprimirfolhaderosto*
% --

% 3. Ficha catalográfica ---

% A biblioteca da FT lhe fornecerá um PDF
% com a ficha catalográfica definitiva após a defesa do trabalho. Quando estiver
% com o documento, salve-o como PDF no diretório do seu projeto e substitua todo
% o conteúdo de implementação deste arquivo pelo comando abaixo:

% --- Para a versão final, delete as linhas abaixo e insira a linha do \includepdf
\thispagestyle{empty}
 \begin{fichacatalografica}
    \vspace*{\fill}
    \begin{center}
        \textsc{Inclua aqui o pdf com a ficha catalográfica fornecida.}
    \end{center}
    \vspace*{\fill}
% --- --- ---
    %\includepdf{./figs/ficha-catalografica.pdf}
 \end{fichacatalografica}
% ---


% 4. Folha de aprovação ---

% Após a aprovação do trabalho, substitua todo o conteúdo deste arquivo por uma
% imagem da página assinada pela banca com o comando abaixo:

% --- Na versão final, exclua essas linhas e insira o \includepdf
\newpage
\thispagestyle{empty}
\vspace*{\fill}
	%\textsc{Inclua aqui a Folha de aprovação.}
	
	\vspace{5.0cm}
	
	A banca examinadora composta pelos membros abaixo aprovou esta dissertação.

	\vspace{1.5cm}
	
	Prof. Dr. Vitor Rafael Coluci (Presidente).
	
	Unicamp, Faculdade de Tecnologia.

	\vspace{0.5cm}
	
	Prof. Dr. Varese Salvador Timoteo.
	
	Unicamp, Faculdade de Tecnologia.
	
	\vspace{0.5cm}
	
	Dr. Ronaldo Giro.
	
	IBM, Brazil Research Lab.
	
	\vspace{1.5cm}
	
	``A Ata da Defesa, assinada pelos membros da Comissão Examinadora, consta no
processo de vida acadêmica do aluno''.

\vspace*{\fill}
\newpage
% --- --- ---
%\includepdf[pagecommand={\thispagestyle{plain}}]{./figs/folha-assinaturas.pdf}	
\clearpage
\cleardoublepage

% 5. Dedicatória ---
\begin{dedicatoria}
    \thispagestyle{empty}
		\vspace*{\fill}
    \centering
    \noindent
    \textit{À minha mãe, Sandra, e ao meu filho, Heitor.}
    \vspace*{\fill}
\end{dedicatoria}
% ---

% 6. Agradecimentos ---
%\begin{agradecimentos}
%    \thispagestyle{empty}
%    Agradecimentos aqui.
%\end{agradecimentos}

% ---

% --- Epígrafe  ---
%\begin{epigrafe}
%	\thispagestyle{empty}
%    \vspace*{\fill}
%	\begin{flushright}
%		\textit{``Escreva aqui a sua epígrafe''\\
%		(Citação)}
%	\end{flushright}
%\end{epigrafe}


\thispagestyle{empty}
\begin{resumo}
			% 7. RESUMO em português			
Um programa (disponível no GitHub) foi desenvolvido em Linguagem C para estudar um método que transforma, por meio de Algoritmos Genéticos (GAs), a obtenção de alguns autovalores de Matrizes Simétricas em um problema de Otimização Combinatória. A análise das Funções de Avaliação (\emph{fitness}) mostrou em quais condições o autovalor mínimo é encontrado, gerando a hipótese de que um artigo de 2004, publicado em periódico internacional e indexado, está parcialmente incorreto. Justificativas matemáticas, baseadas nas propriedades do Quociente de Rayleigh, e experimentos computacionais, executados com matrizes de Coope--Sabo, confirmaram a hipótese.
     
    \vspace{\onelineskip}

    \noindent\textbf{Palavras-chaves}: Autovalores; Quociente de Rayleigh; Algoritmos Genéticos; Otimização Combinatória; Linguagem C.

    \vspace{\onelineskip}
    \vspace{\onelineskip}
		
		% 8. Abstract (resumo traduzido para o inglês);
    \begin{otherlanguage*}{english}
    \begin{center}{\ABNTEXchapterfont\huge Abstract}\end{center}
    
    A program (available on GitHub) was developed in C language to study a method that transforms, by means of Genetic Algorithms (GAs), the calculation of some \emph{eigenvalues} of Symmetric Matrices in a Combinatorial Optimization problem. The analysis of Evaluation Functions (fitness) showed in which conditions the minimum \emph{eigenvalue} can be found, generating the hypothesis that a 2004 article, published in an international and indexed periodic, is partially incorrect. Mathematical justifications, based on properties of Rayleigh Quotient, and computational experiments, executed with Coope--Sabo matrices, confirmed the hypothesis.
		
    \vspace{\onelineskip}

    \noindent\textbf{Keywords}: Eigenvalues; Rayleigh Quotient; Genetic Algorithms; Combinatorial Optimization; C Language.

    \end{otherlanguage*}
		
		% 9. Resumo em uma terceira língua (opcional);
		% \begin{otherlanguage*}{english} %% substitua o ``english'' pelo idioma desejado.
    % \begin{center}{\ABNTEXchapterfont\huge Abstract}\end{center}
    
    % Put abstract here.
    % \vspace{\onelineskip}

    % \noindent\textbf{Keywords}: keyword 1; keyword 2; keyword 3.

    % \end{otherlanguage*}
		
\end{resumo}

% 10. Lista de ilustrações (opcional);
\pdfbookmark[0]{\listfigurename}{lof}
\listoffigures*
\thispagestyle{empty}
\clearpage
\cleardoublepage
% ---

% 11. Lista de Tabelas (opcional);
\pdfbookmark[0]{\listtablename}{lot}
\listoftables*
\thispagestyle{empty}
\clearpage
\cleardoublepage
% ---

% 12. Lista de Abreviaturas e Siglas (opcional);
%\renewcommand{\nomname}{Lista de Acrônimos e Abreviações}
%\pdfbookmark[0]{\nomname}{las}
%\printnomenclature
%\thispagestyle{empty}
%\cleardoublepage
% ---

% 13. Lista de Símbolos (opcional);

% ---

% 14. Sumário.
\pdfbookmark[0]{\contentsname}{toc}
\tableofcontents*
\thispagestyle{empty}
\clearpage
\cleardoublepage
% ---

% ---- ELEMENTOS TEXTUAIS ----
\textual

% ---- Capítulos ----
\chapter{Introdução}
\label{cap:introducao}

\setcounter{page}{12}

Não há fórmula para as raízes de um polinômio de grau $n > 4$. \cite{Pan97}.

Matriz \emph{companion} do polinômio. A Frobenius c

https://en.wikipedia.org/wiki/Companion\_matrix

Aqui devo escrever sobre a importância e relevância do estudo. 
\chapter{Quociente de Rayleigh\label{cap:algebra}}

Com referência à equação \ref{eq:detIntro} da Introdução, seja $\mathsf{A}$, a partir de agora, uma matriz auto$-$adjunta ($A^{\dag} = A$). Todos os seus autovalores $\lambda_i$ são reais e, consequentemente, podem ser ordenados do menor para o maior:

	\begin{equation}\label{eq:autovalores_ordenados}
		\lambda_1 \leq \lambda_2 \leq \cdots \leq \lambda_n.
	\end{equation}

Para $\mathsf{A}$, que opera sobre os vetores \textbf{\texttt{u}} do espaço euclidiano $\varepsilon^n$, define-se o \textbf{Quociente de Rayleigh} $\rho (\textbf{\texttt{u}})$:

\begin{equation}\label{eq:rho}
	\rho (\textbf{\texttt{u}}) \equiv \rho(\textbf{\texttt{u}}; \mathsf{A}) \equiv \frac{\textbf{\texttt{u}}^\dag \mathsf{A}\textbf{\texttt{u}}}{\textbf{\texttt{u}}^\dag \textbf{\texttt{u}}}, 
\end{equation}
onde $\textbf{\texttt{u}}^\dag$ é o complexo conjugado de $\textbf{\texttt{u}}$, que por sua vez deve ser sempre diferente de zero ($\textbf{\texttt{u}} \neq 0$). Todo vetor \textbf{\texttt{u}} possui um $\rho(\textbf{\texttt{u}})$.

	O Quociente de Rayleigh está diretamente relacionado aos autovalores e autovetores. Suponha que $\textbf{\texttt{w}}_i$ seja um dos $n$ autovetores de $\mathsf{A}$. A equação \ref{eq:rho} fica
	
	\begin{equation}\label{eq:rho_no_w}
		\rho(\textbf{\texttt{w}}_i) = \frac{\textbf{\texttt{w}}_i^\dag \mathsf{A} \textbf{\texttt{w}}_i}{ \textbf{\texttt{w}}_i^\dag \textbf{\texttt{w}}_i} = \frac{\textbf{\texttt{w}}_i^\dag  \lambda_i \textbf{\texttt{w}}_i}{\textbf{\texttt{w}}_i^\dag \textbf{\texttt{w}}_i} = \frac{ \lambda_i \textbf{\texttt{w}}_i^\dag  \textbf{\texttt{w}}_i}{\textbf{\texttt{w}}_i^\dag \textbf{\texttt{w}}_i} = \lambda_i,
	\end{equation}
	ou seja, o quociente de Rayleigh de um autovetor é o autovalor associado:
	
	\begin{equation}\label{eq:rho_no_w_eh_lambda}
		\rho (\textbf{\texttt{w}}_i) = \lambda_i.
	\end{equation}
	
	Portanto, se temos um autovetor de uma matriz auto$-$adjunta, não é necessário resolver o sistema linear \ref{eq:detIntro} para obter o autovalor, basta efetuar as multiplicações de \ref{eq:rho}.
	
	O vetor gradiente de $\rho$ é \cite{Wilkinson1965}
	
	\begin{equation}\label{eq:gradrho}
		\nabla \rho (\textbf{\texttt{u}}) = \frac{2[\mathsf{A} - \rho(\textbf{\texttt{u}})] \textbf{\texttt{u}}}{\textbf{\texttt{u}}^\dag \textbf{\texttt{u}}}.
	\end{equation}

	Ele é nulo se, e somente se, $\textbf{\texttt{u}}$ é um autovetor $\textbf{\texttt{w}}_i$  de $\mathsf{A}$:
	
	\begin{equation}\label{eq:grad_rho_no_w}
		\nabla \rho (\textbf{\texttt{w}}_i) = \frac{2[\mathsf{A} - \rho(\textbf{\texttt{w}}_i)] \textbf{\texttt{w}}_i}{\textbf{\texttt{w}}_i^\dag \textbf{\texttt{w}}_i} = \frac{2[\mathsf{A} - \lambda_i]\textbf{\texttt{w}}_i}{\textbf{\texttt{w}}_i^\dag \textbf{\texttt{w}}_i} = \frac{2[\mathsf{A}\textbf{\texttt{w}}_i - \lambda_i\textbf{\texttt{w}}_i]}{\textbf{\texttt{w}}_i^\dag \textbf{\texttt{w}}_i} = \frac{2[0]}{\textbf{\texttt{w}}_i^\dag \textbf{\texttt{w}}_i} = 0,
	\end{equation}
	
	\begin{equation}\label{eq:grad_rho_nulo}
		\nabla \rho (\textbf{\texttt{w}}_i) = 0.
	\end{equation}

Além disso, para todos os vetores \textbf{\texttt{u}} $n-$dimensionais diferentes de zero, $\rho(\textbf{\texttt{u}})$ é limitado no intervalo [$\lambda_1, \lambda_n$] entre o menor e o maior autovalor da matriz $\mathsf{A}$ \cite{Parlett1998}. Em seguida agrupo as três propriedades de $\rho(\textbf{\texttt{u}})$ citadas acima.

\textbf{Algumas propriedades de $\rho(\textbf{\texttt{u}})$}

\begin{enumerate}
		
	\item Se $\textbf{\texttt{w}}_i$ é um autovetor, $\rho(\textbf{\texttt{w}}_i) = \lambda_i$ .
	
	\item \textbf{Fronteira}: $\rho(\textbf{\texttt{u}})$ é limitado no intervalo [$\lambda_1, \lambda_n$].
	
	\item \textbf{Estacionaridade}: o gradiente de $\rho(\textbf{\texttt{u}})$ é nulo se $\textbf{\texttt{u}}$ é um autovetor.
	
\end{enumerate}

	A segunda e terceira propriedades podem ser utilizadas para tratar o problema dos autovalores como um problema de otimização matemática. A propriedade de fronteira, por exemplo, permite transformar o problema de encontrar o menor (ou maior) autovalor de uma matriz auto$-$adjunta em um problema de minimização (ou maximização) de $\rho(\textbf{\texttt{u}})$. A estacionaridade pode, inclusive, auxiliar na definição da função objetivo. Minimizar $\nabla \rho(\textbf{\texttt{u}})$ também leva a um autovetor, entretanto, note que, quando $\nabla \rho(\textbf{\texttt{u}}) = 0$, $\textbf{\texttt{u}}$ pode ser qualquer um dos $n$ autovetores ($\textbf{\texttt{w}}_1, \textbf{\texttt{w}}_2, \cdots, \textbf{\texttt{w}}_n$) de $\mathsf{A}$. Ou seja, não há garantia que o autovalor obtido é o menor ou maior.

		A primeira propriedade permite que a obtenção dos autovalores seja transformada em um método de busca de autovetores. O espaço de busca é o conjunto de todos os vetores $\textbf{\texttt{u}}_i$, e o espaço de soluções é composto pelos $n$ autovetores ($\textbf{\texttt{w}}_1, \textbf{\texttt{w}}_2, \cdots, \textbf{\texttt{w}}_n$).
		
		Nesta dissertação estudei um método que cria um Algoritmo Genético para fazer essa busca. Ele faz uso das três propriedades.
\chapter{Algoritmos Genéticos\label{cap:ga}}

% INÍCIO TIAGO

Proposto inicialmente por Holland \cite{Holland}, os GAs são uma classe particular de algoritmos evolutivos que buscam por soluções aproximadas em problemas com vasto espaço de busca e que, por sua complexidade, tornam inviável o uso de métodos convencionais.
	
	No decorrer deste trabalho serão expostas as origens históricas e as principais características dos GAs, apresentando exemplos de implementações de forma a ilustrar na prática o seu funcionamento.

 \section{Histórico da Computação Evolutiva}
 
A teoria da evolução, proposta por Darwin (1859), é a essência das técnicas utilizadas na Computação Evolutiva (CE). Surgida em meados dos anos 60 e partindo da idéia que a Inteligência Computacional busca imitar o comportamento e os fenômenos encontrados na natureza, os conceitos da CE são inspirados no darwinismo, traduzindo em termos computacionais a teoria da seleção natural e o adaptacionismo.

Na natureza, seja pela falta de recursos ou dificuldades apresentadas pelo ambiente, a tendência da evolução é perpetuar os genes mais adaptados de uma população, e, nesta lei de sobrevivência do mais forte, aqueles que se mantêm tem uma probabilidade maior de se reproduzir (\textit{crossover}) e gerar descendentes com as mesmas capacidades. Existe ainda um fator de mutação, geralmente causado por erro na cópia do código genético, que, numa taxa muito pequena, acaba trazendo alterações na codificação genética do individuo, podendo ou não beneficiá-lo.

As técnicas evolutivas traduzem este cenário artificialmente da seguinte forma: Uma população de soluções candidatas (indivíduos) à resolução de um problema (ambiente), passará pelos processos de seleção, \textit{crossover} e mutação. Geração após geração, o algoritmo evolui até que um destes indivíduos seja escolhido como a solução mais próxima (indivíduo mais apto) para o problema em questão. Os operadores envolvidos nestas etapas serão descritos adiante neste trabalho.

No âmbito da CE surgiram diferentes linhas de pesquisa, que por divergirem em alguns aspectos acabaram por criar sub-áreas independentes de estudo. Dentre as principais, podemos citar:
\begin{itemize}
 \item Programação Evolutiva \cite{Fogel}; 
 \item Estratégias Evolutivas \cite{Rechenberg};
 \item Algoritmos Genéticos \cite{Holland}
\end{itemize} 

Esses métodos vêm sendo muito explorados por diversos pesquisadores em busca de soluções para problemas demasiadamente complexos, pela ausência ou incapacidade de resolução via métodos determinísticos convencionais.

É importante salientar que as técnicas evolutivas, pelo seu caráter estocástico, nem sempre encontram a solução ótima para o problema, e devem ser utilizadas com muito critério. A prioridade deve ser sempre dada para os algoritmos exatos (quando disponíveis).

\section{Algoritmos Genéticos}

Este trabalho tem como objetivo principal apresentar uma técnica de busca baseada nos conceitos da CE, conhecida como Algoritmos Genéticos (GAs) e proposta inicialmente por Holland \cite{Holland}.

Os GAs usam terminologia baseada na CE, fazendo analogia entre sistemas naturais e artificiais, exemplificado na tabela \ref{tabSist}:

\begin{table}[htp]
 \caption{\label{tabSist}Sistemas Naturais x Sistemas Artificiais}
 \begin{center}
  \begin{tabular}{c|c}
   \hline
   \textbf{Genética Natural}  & \textbf{Genética Artificial} \\
   \hline
   gene    & caractere \\
   alelo    & valor do caractere \\
   cromossomo & cadeia de caracteres (individuo) \\ 
   locus    & posição do gene na cadeia de caracteres \\
   ambiente  & problema a ser solucionado \\
   \hline
   \end{tabular}
 \end{center}
\end{table}
 
A estrutura básica de um GA é composta pelos seguintes passos:

\begin{itemize}

 \item \textbf{Avaliação}: Uma função de avaliação é aplicada a todos os indivíduos de uma população a fim de determinar os mais aptos (soluções mais próximas do objetivo). Esta medida é conhecida como \textit{Fitness}. 
 
 \item \textbf{Seleção}: A seleção escolhe os indivíduos que serão copiados para a próxima geração, sendo os com maior \textit{Fitness} aqueles com maior probabilidade de serem escolhidos.
 
 \item \textbf{Variação}: Os indivíduos escolhidos pela seleção passam pelos processos de \textit{crossover} e mutação, de acordo com a probabilidade pré-estabelecida que estes operadores podem ocorrer.
 
\end{itemize} 

Normalmente os GAs partem de uma população inicial gerada aleatoriamente e evoluem em busca de uma solução que satisfaça o objetivo traçado para a aplicação. Pela sua natureza estocástica, a responsabilidade de sucesso de um GA decai sobre a sua função de avaliação, tornando-a peça fundamental de todo o processo.

\section{Motivação e Relevância}

Os GAs têm uma estrutura simples e singular, que pode ser utilizada para as mais diversas aplicações. Isso torna-os extremamente genéricos, e facilita o reuso de uma mesma implementação para problemas distintos.

A sua capacidade de atuar em problemas NP-Completo, aliada à facilidade de reutilização, tornaram esta técnica muito interessante, levando pesquisadores de diversas áreas a utilizá-la com sucesso. Podemos citar publicações nas áreas de:
 
\begin{itemize}
\item \textbf{Otimização Combinatória}: Diversos trabalhos na área de otimização combinatória utilizam com sucesso o os GAs. Em \cite{Anna} é mostrada a aplicação na configuração de uma turbina hidrocinética.

\item \textbf{Robótica}: O trabalho de Harashima \cite{Harashima} propõe a utilização de GAs na movimentação de um braço robótico.

\item \textbf{Jogos}: Em \cite{Appolinario} é apresentada uma solução para navegação autônoma de agentes em jogos utilizando GAs. É feito um comparativo com uma solução baseada em árvores de busca. Apesar da ligeira desvantagem nos experimentos apresentados, os GAs mostram-se uma solução promissora para problemas em maior escala.

\item \textbf{Reconhecimento de padrões em imagens}: Em \cite{Marinho}, utiliza-se uma combinação de técnicas de segmentação de imagens e GAs para facilitar o reconhecimento de áreas devastadas na mata atlântica, buscando por padrões entre as imagens capturadas por satélite;

\item \textbf{Treinamento de Redes Neurais}: A capacidade dos GAs em gerar populações de boas soluções pode ser usada para o treinamento de Redes Neurais \cite{VanRooij1996}.

\end{itemize} 

\section{Principais Elementos de um Algoritmo Genético}
	
	\subsection{Representação Cromossomial}

	Conforme citado anteriormente, o cromossomo é uma cadeia de caracteres (genes) de comprimento \textit{L}, que representa um indivíduo candidato a solução do problema proposto.
	
	Esta representação compreende parte de grande importância num GA, pois é a partir desta estrutura que os indivíduos serão avaliados e também sofrerão a atuação dos operadores de variação (\textit{crossover} e mutação) e seleção natural. A codificação genética encontrada nos cromossomos dos seres vivos caracteriza as diferenças encontradas em cada espécie. Tal importância não poderia ser menos crítica quando traduzida para a genética artificial, portanto, uma boa codificação do cromossomo influi diretamente no sucesso da aplicação de um GA.
	
	No momento da definição desta representação, deve-se ter em mente \cite{Linden2008}:
	
	\begin{itemize}
\item Deve ser a mais simples possível.
\item Soluções proibidas não devem ser representadas. 
\item Condições de qualquer tipo devem estar implícitas na representação.
\end{itemize} 
	
	A representação binária introduzida por Holland \cite{Holland} é a mais comum entre os GAs, por facilitar a utilização dos operadores genéticos e ser de fácil manipulação.
	
	Caso se faça necessário, um cromossomo pode conter mais de uma variável em sua cadeia, sendo que estas serão concatenadas para representar o indivíduo. Na tabela \ref{tabCromo} é exibido um exemplo com três possíveis indivíduos com cromossomo multivariável (variáveis x1 e x2) de comprimento $L = 5$ e alelos que variam entre 1 e 0. Porém, é possível implementar soluções com \textit{k} diferentes alelos para cada locus do cromossomo, contudo a dificuldade em manipular esta estrutura será maximizada.
	
	Pela simplicidade apresentada pela representação binária, este trabalho optou por utilizá-la em seus exemplos e implementações. 
	
	\begin{table}[htp]
 \caption{\label{tabCromo}Exemplo de representação cromossomial}
 \begin{center}
  \begin{tabular}{c|c|c}
   \hline
   Indivíduo & x1  & x2 \\
   \hline
   1 & 10010    & 01101 \\
   2 & 00110    & 11100 \\
   3 & 11101		& 01001 \\ 
   \hline
   \end{tabular}
 \end{center}
\end{table}

%========================TIAGO - fim ========================================================
	
	\subsection{Função de Avaliação}
	
	Na Introdução foi dito que a função de avaliação tem um papel fundamental em um algoritmo genético. Podemos ir além, afirmando que ela é o elo mais forte - senão o único - entre o algoritmo e o problema que tentamos resolver no mundo real. Aliás, geralmente a principal diferença entre dois GAs reside apenas, e justamente, na função de avaliação. Portanto, não seria estranho concluir que ela deve conter todo o conhecimento do problema, incluindo suas condições e restrições.
	
	Mas, afinal, o que \textit{é} a função de avaliação? Uma vez definido o problema e, em seguida, o objetivo do algoritmo, ela representa nesse contexto a qualidade de um indivíduo. Em outras palavras, através da função de avaliação devemos ser capazes de identificar se um cromossomo leva ou não à uma boa solução. Assim, ela deve refletir a meta que desejamos atingir.
	
	Ao aplicarmos a função de avaliação\footnote{Essa função também pode ser chamada de Função Custo, por isso o $f_c$ na equação \ref{funcaval}.} em um indivíduo obtemos uma nota associada aquele cromossomo. Essa nota é um número, um escalar, que pode ser discreto (inteiro) ou contínuo (real):
	
	\begin{equation}\label{funcaval}
		n = f_c(cromossomo)
	\end{equation}
	
	Então, já podemos apresentar a primeira característica de uma função de avaliação: quanto maior a nota, melhor o indivíduo. Pensando em termos da Teoria da Evolução, maiores notas exprimem indivíduos mais adaptados ao ambiente (metas). Além disso, na equação \ref{funcaval}, $n$ é uma métrica que deve identificar o quão próximo um cromossomo está de uma boa solução.
	
	Por exemplo, suponha que o cromossomo $c_1$ tem nota $n_1 = 10$, enquanto $n_2 = 9,7$ é atribuída ao cromossomo $c_2$. Imaginando hipoteticamente que uma boa solução está próxima de $n_{boa} = 11$, podemos concluir que ambos são bons, mas $c_1$ é melhor. Sintetizando, uma boa função de avaliação deve quantificar, dentre boas soluções, quais são as melhores.
	
	Outro número importante é o \textit{fitness} médio ($<n>$), ou seja, a razão entre a soma das notas de todos os indivíduos e o número de indivídios na geração ($N$):
	
	\begin{equation}\label{fitness_medio}
		<n> = \frac{\sum_{i = 1}^{N} n_i}{N}
	\end{equation}
	
	
	Quando não sabemos qual será a maior nota possível para o nosso problema, temos a opção de usar a estabilidade de $<n>$ como critério de parada. Por quê? Porque se a média das notas não muda muito com o passar do tempo, podemos concluir que o material genético disponível indivíduo a indivíduo é muito semelhante, e essa ausência de variabilidade faz com que uma geração futura se pareça com a passada. Em linguagem mais técnica, estamos confinados a uma região específica no espaço de soluções.
	
	Com relação ao comportamento da função de avaliação, É extremamente desejável que  seja suave e regular. O que isso quer dizer? Que, se um indivíduo é levemente superior a outro, sua nota deve ser apenas um pouco maior. Infelizmente, na maioria das vezes isso não acontece, e o impacto pode surgir em forma de instabilidade do \textit{fitness} médio. Na figura \ref{figFitness} encontramos dois exemplos.

	\begin{figure}[htp]
		\begin{center}
			\includegraphics[height=4.7cm]{figs/ga/func_aval_estavel.png}
			\includegraphics[height=4.7cm]{figs/ga/func_aval_instavel.png}
		\end{center}
		\caption{\label{figFitness}Exemplo de instabilidade do \textit{fitness}. Enquanto em (a) o \textit{fitness} cresce de maneira contínua e depois se estabiliza, em (b) há alguns pontos onde onde o comportamento da nota sofre uma mudança razoavelmente brusca. Gráficos retirados de \cite{Bhattacharyya2004}.}
	\end{figure}

	Na próxima seção discutiremos um exemplo de definição de uma função de avaliação.
		
	\subsection{\label{MaxSeno}Exemplo: Máximos de $f(x) = \sin(x) / x$}
	
	A função
	
	\begin{equation}\label{eqSinX}
		f(x) = \frac{\sin(x)}{x}
	\end{equation}
	aparece em várias áreas da matemática. Seu caráter oscilatório é muito útil para descrever pacotes de onda na mecânica quântica \cite{QuGA2006} ou auxiliar no estudo das transformadas de Fourier \cite{James2002}.
	
	\begin{figure}[htp]
		\begin{center}
			\includegraphics[width=13cm]{figs/ga/sen_x__x.png}
		\end{center}
		\caption{\label{figSen}Gráfico da função $f(x) = \sin(x) / x$ para $x = [-20,20]$. Se desejamos encontrar os máximos (global ou locais), uma boa função de avaliação seria a própria $f(x)$. Gráfico gerado com o GnuPlot \cite{gnuplot}.}
	\end{figure}
	
	Observando seu gráfico na figura \ref{figSen}, vemos que ela tem um máximo global em $x = 0$ e vários máximos locais. Imaginando que quiséssemos obter os valores de $x$ onde esses máximos aparecem, como definir uma boa função de avaliação?
	
	Nesse caso podemos utilizar a própria $f(x)$ (equação \ref{eqSinX}), pois ela possui boas características. Pelo gráfico \ref{figSen} concluímos que ela é suave e, se compararmos dois valores bem próximos de $x$, o resultado de $f(x)$ também é. Por exemplo, $f(0,5) = 0,9589$ e $f(0,51) = 0,9572$, evidenciando que o ponto $x = 0,5$ é um indivíduo levemente melhor.
	
	
	\begin{table}[htp]		
		\caption{\label{tabSen}Valores de $x$ gerados aleatoriamente para a função $f(x) = \sin(x)/x$. A própria $f(x)$ pode ser usada como função de avaliação.}
		\begin{center}
			\begin{tabular}{c|c|c}
				\hline
				\textbf{Indivíduo}& $\textbf{x}$		& $\textbf{f(x)}$ \\
				\hline
				01 & 	2							& 0,455 \\
				02 & 	3							& 0,047 \\
				03 &	-9						& 0,046\\	
				04 &	-8							& 0,124 \\
				05 &	19							& 	0,008 \\
				\hline
				\multicolumn{2}{r}{\textbf{Soma:}} & 0,680 \\
				\hline
			\end{tabular}
		\end{center}
	\end{table}
	
	
	Se escolhermos vários valores para $x$ aleatoriamente, basta aplicar a $f(x)$ e selecionar os maiores\footnote{Em geral deve-se impor algumas restrições na função de avaliação. Veremos na seção \ref{selecao} que valores negativos são proibidos para o método da Roleta.}. Na tabela \ref{tabSen} listamos cinco pontos como exemplo, assim como a soma dos resultados de $f(x)$. Como veremos na próxima seção, os pontos $01$ e $04$ têm, respectivamente, $0,455/0,680 = 66,9\%$ e $ 0,124/0,680 = 18,2\%$ de chance de serem selecionados no processo de Seleção.
	
	\subsection{\label{selecao}Seleção}
	
	A parte do algoritmo genético que chamamos de \textit{Seleção}\footnote{Alguns autores chamam esse módulo de Seleção de Pais, enquanto outros o definem exatamente como na biologia, Seleção Natural. Para evitar mal entendidos, optamos por chamá-lo apenas de Seleção.} tem como objetivo simular o processo de Seleção Natural da Evolução. Basicamente, os  mais aptos, leia-se ``com maior \textit{fitness}'', devem gerar mais descendentes.
	
	Porém, exatamente como na natureza, os indivíduos avaliados com notas menores não devem ser totalmente descartados, e há bons motivos para isso. Em primeiro lugar, esses cromossomos, apesar de mal avaliados, podem conter informação genética importante, senão fundamental, para uma boa solução. Em segundo, a seleção apenas dos melhores, chamada de Elitismo \cite{Bhattacharyya2004}, pode levar à uma convergência precose e com soluções não tão boas.
	
	\subsubsection{\label{ExemploVariabilidade}Exemplo: Variabilidade Genética}
	
	Como exemplo da importância da variabilidade genética, imagine o problema de encontrar o valor máximo da função $f(x) = -x^2 + 36$ no intervalo (discreto) $x = [0,7]$. Assim, uma representação cromossomial binária com 3 \textit{bits} é suficiente, pois 0 = 000 e 7 = 111 (tabela \ref{tabRepCroX2}).
	
	\begin{table}[htp]
 		\caption{\label{tabRepCroX2}Representação cromossomial para os pontos $x = 0$ até $x = 7$ dentro do problema de máximo da função $f(x) = -x^2 + 36$.}
 		\begin{center}
  		\begin{tabular}{c|c}
   			\hline
   			\textbf{x (decimal)}  & \textbf{x (representação binária)} \\
   			\hline
   			0 & 000 \\
   			1 & 001 \\
   			2 & 010 \\ 
   			3 & 011 \\
   			4 & 100 \\
   			5 & 101 \\ 
   			6 & 110 \\
   			7 & 111	\\
   			\hline
   		\end{tabular}
 		\end{center}
	\end{table}
	
	\begin{figure}[htp]
		\begin{center}
			\includegraphics[width=13cm]{figs/ga/Parabola.png}
		\end{center}
		\caption{\label{figParabola}Gráfico da função $f(x) = -x^2 + 36$ para $x = [-1,1]$. Gráfico gerado com o GnuPlot \cite{gnuplot}.}
	\end{figure}
	
	 Vemos na figura \ref{figParabola} que a resposta correta é $x = 0$, cujo valor é $f(0) = 36$. Suponha agora que os dados da tabela \ref{tabFuncMax} representem a população inicial do algoritmo. Se escolhêssemos apenas os indivíduos com as melhores notas, ou seja, $x = 2$ e $x = 3$, descartaríamos o indivíduo $x = 5$ e perderíamos uma importante característica genética: o zero no \textit{bit} central.
	 
\begin{table}[htp]
		\caption{\label{tabFuncMax}População inicial para o problema de máximo da função $f(x) = -x^2 + 36$. O valor de máximo ocorre em x = 0, ou x = 000 na representação binária.}
		\begin{center}
			\begin{tabular}{c|c}
				\hline
				$\textbf{x}$ (decimal)		& $\textbf{x}$ (binário) \\
				\hline
				2							& 010 \\
				7							& 111 \\
				5							& 101 \\	
				3							& 011 \\
				\hline
			\end{tabular}
		\end{center}
	\end{table}
		
		O que aconteceria depois? O mais próximo do valor máximo que o algoritmo conseguiria chegar seria $f(2) = 32$, independentemente do \textit{crossover} entre os indivíduos restantes. Portanto, esse resultado final seria considerado prematuro, efeito denominado \textbf{Convergência Genética}. Em outras palavras, se apenas os melhores indivíduos se reproduzirem, as novas gerações chegarão rapidamente a um estado em que os cromossomos são muito semelhantes entre si, minando a diversidade genética e impedindo a evolução de prosseguir satisfatoriamente.
		
		Vários métodos foram criados para executar a Seleção de maneira coerente, ou seja, privilegiando os indivíduos com alta função de avaliação, mas não desprezando totalmente os de menor nota. O método mais utilizado em aplicações de Algoritmos Genéticos é o da Roleta, assunto da próxima seção.
	
	\subsubsection{O método da Roleta (\textit{Roulette Wheel})}
	
	A ideia do método é simular uma roleta parecida com as utilizadas em cassinos. Entretando, há duas diferenças. Enquanto na roleta tradicional temos o mesmo tamanho de fatia para cada número, na roleta para os algoritmos genéticos a fatia que cada indivíduo ganha é proporcional à sua avaliação.
	
	Uma vez calculados os ângulos para cada cromossomo, giramos a roleta e selecionamos o indivíduo. Nesse ponto reside a segunda diferença. Nos cassinos, o número é selecionado por uma bolinha que também está em movimento. Já nos algoritmos genéticos, o cromossomo é selecionado por um marcador fixo em uma roleta que lembra a utilizada no jogo \textit{Roda a Roda} do SBT \cite{roda-a-roda}.
	
	Na tabela \ref{tabSumFitness} temos os valores dessas fatias (em porcentagem e graus) para o exemplo da seção \ref{ExemploVariabilidade}. Note que o indivíduo 2 foi descartado, e esse é um ponto muito importante no método da roleta. \textbf{indivíduos com avaliações negativas não podem ocupar a roleta}, e o motivo é simples: notas negativas levam a pedaços negativos na roleta, como, por exemplo, - 15\% ou -125$^\circ$.
		
	\begin{table}[htp]		
		\caption{\label{tabSumFitness}Todas as notas dos indivíduos para o exemplo da seção \ref{ExemploVariabilidade}. Lembre-se que para obter máximo da função $f(x) = -x^2 + 36$ podemos utilizar, com algumas restrições, a própria $f(x)$ como função de avaliação $f_c(x)$ (seção \ref{MaxSeno}).}
		\begin{center}
			\begin{tabular}{c|c|c|c|c|c}
				\hline
				\textbf{Indivíduo}	& $\textbf{x}$		& $\textbf{f(x)}$	& \textbf{f$_c$(Indivíduo)} & \textbf{Ocupação (\%)} & \textbf{Ângulo ($^\circ$)} \\
				\hline
				01	& 2	& 32  & 32 	& 46	& 165 						\\
				02	& 7	& -13 & 0		& 0		& 0						\\
				03	&	5	& 11	& 11	& 16	& 57						\\	
				04	&	3	& 27	& 27	& 39	& 139						\\
				\hline
				\multicolumn{3}{c|}{Totais:}  & \textbf{70} & \textbf{100} & \textbf{360} \\
				\hline
			\end{tabular}
		\end{center}
	\end{table}
	
	
	\begin{figure}[htp]
		\begin{center}
			\includegraphics[width=13cm]{figs/ga/roleta_viciada.png}
		\end{center}
		\caption{\label{figRoleta}Roleta criada a partir dos dados da tabela \ref{tabSumFitness}. Note que o indivíduo x = 7 foi descartado porque sua avaliação foi um número menor do que zero.}
	\end{figure}
	
	Mas como implementar tal roleta? Um exemplo em Linguagem C encontra-se na figura \ref{figCodRoleta}. Na linha 226 um \textit{\texttt{for}} é definido de modo que as suas instruções sejam executadas \textit{\texttt{numindividuos}} vezes. Dessa maneira, a população de indivíduos mantêm-se fixa, e podemos trabalhar com vetores constantes. A próxima linha de código armazena em \textit{\texttt{vlrRoleta}} um número aleatório entre 0 e a soma das notas de todos os indivíduos (\textit{\texttt{sumFitness}}). Essa soma também é exibida na tabela \ref{tabSumFitness}. A variável \textit{\texttt{iSelecionado}} armazenará o índice do indivíduo selecionado.
	
	No laço \textit{\texttt{do ... while}} a roleta começa de fato a girar. A partir do primeiro indivíduo, somamos o \textit{fitness} de cada um em \textit{\texttt{sumFitness}}. Quando essa variável atingir um valor maior do que \textit{\texttt{vlrRoleta}}, o indivíduo com índice \textit{\texttt{iSelecionado}} na geração atual será transferido para a posição \textit{\texttt{iIndividuo}} da próxima geração (linha 238 da figura \ref{figCodRoleta}). 
	
	Apesar do caráter randômico do método, ele prevê, estatisticamente, que a quantidade de vezes que um cromossomo aparecerá na próxima geração é proporcional à sua nota. Portanto, não despreza completamente os indivíduos com menor \textit{fitness}, ao mesmo tempo que privilegia os mais aptos.

	\begin{figure}[htp]
		\begin{center}
			\includegraphics[width=13cm]{figs/ga/CodigoRoleta.png}
		\end{center}
		\caption{\label{figCodRoleta}Exemplo de código em Linguagem C para o método da Roleta.}
	\end{figure}
	
	\subsubsection{Seleção por torneio}
	
	Utilizada no mestrado porque o paralelismo é direto.
	
	\subsubsection{Exemplo: máximo da função $f(x) = -x^2 + 36$}
	
	Usaremos os dados da tabela \ref{tabSumFitness} para montar a roleta da figura \ref{figRoleta} e mostrar, passo a passo, o funcionamento do método.
	
	Como temos quatro indivíduos na população,
	
	$$
	numindividuos = 4.
	$$

	Devemos obter um número aleatrório entre 0 e 70 (soma dos fitness). Imagine que esse primeiro número tenha sido 65:
	
	$$
	vlrRoleta = 65.
	$$
	
	Entramos no \textit{\texttt{for}} e \textit{\texttt{iSelecionado}} é inicializada com -1:
	
	$$
	iSelecionado = -1.
	$$
	
	Já dentro do laço \textbf{\texttt{do ... while}}O primeiro indivíduo na geração é o $x = 2$, com \textit{fitness} igual a 32. Então \textit{\texttt{sumRoleta}} passa a ser também 32, mas continua menor do que \textit{\texttt{vlrRoleta}}:
	
	$$
		sumRoleta = 32 < vlrRoleta = 65.
	$$
	
	Ainda não é possível sair do laço \textit{\texttt{do ... while}}, e os passos são repetidos para o segundo indivíduo, $x = 7$. Entretanto, esse cromossomo obteve nota nula na função de avaliação, fazendo com que o valor de \textit{\texttt{sumRoleta}} não seja alterado pela soma da linha 234 (figura \ref{figCodRoleta}).
	
	O terceiro cromossomo possui nota 11, fazendo com que ainda permaneçamos dentro do laço:
	
	$$
		sumRoleta = 43 < vlrRoleta = 65.
	$$
	
	Chegamos ao último indivíduo da população. Seu \textit{fitness} é 27, permitindo a seleção:
	
	$$
		sumRoleta = 70 > vlrRoleta = 65.
	$$
	
	Na tabela \ref{tabRoletaManual} encontra-se uma comparação entre a geração inicial e final após a seleção via método da roleta. Os valores de \textit{\texttt{vlrRoleta}} foram obtidos utilizando a função \texttt{ALEATÓRIOENTRE} do Microsoft Excel.
	
\begin{table}[htp]
	\caption{\label{tabRoletaManual}Valores obtidos aleatoriamente para \textit{\texttt{vlrRoleta}} e os respectivos indivíduos selecionados. Note que o cromosso com \textit{fitness} zero foi eliminado e o \textit{fitness} médio aumentou.}
	\begin{center}
		\begin{tabular}{c|c|c|c|c}
			\hline
			\multicolumn{2}{c|}{\textbf{Geração inicial}} & \textbf{Roleta}& \multicolumn{2}{c}{\textbf{Geração final}}  \\
			\hline
			x 					& \textit{fitness} (n)	& vlrRoleta						& x						& \textit{fitness} (n)	\\
			\hline
			2 					& 32										& 65 									& 3						&	27   \\
			7 					& 0 										& 27 									& 2						&	32\\
			5 					& 11										& 70 									& 3						&	27\\	
			3 					& 27										& 41 									& 5						&	11\\
			\hline
			\multicolumn{2}{c|}{$<n>$ = 17,5} & - & \multicolumn{2}{c}{$<n>$ = 24,25}  \\
			\hline
		\end{tabular}
	\end{center}
\end{table}
	
	Dois fatos interessantes aconteceram. O indivíduo com \textit{fitness} zero foi eliminado conforme o esperado. Ele não ocupou espaço na roleta e, obviamente, não poderia participar do processo de seleção. Além disso, o \textit{fitness} médio ($<n>$) da nova população é superior ao da anterior, indicando que a Seleção cumpriu o seu papel em manter os mais aptos.
	
	
	\subsection{\label{crossover}Reprodução (\textit{Crossover})}

	A Reprodução consiste em combinarmos a informação genética de dois cromossomos da população e gerar um ou mais descendentes, até que o número de indivíduos da nova população seja atingido. Uma maneira simples é através do \textbf{\textit{Crossover} de ponto único} (figura \ref{figCrossOver}).
	
	\begin{figure}[htp]
		\begin{center}
			\includegraphics[width=6cm]{figs/ga/PontosCorte.png}
			\includegraphics[width=6cm]{figs/ga/CrossOver.png}
		\end{center}
		\caption{\label{figCrossOver}Definição de Ponto de \textit{Corte} e o \textit{Crossover} de ponto único. Com esse operador conseguimos gerar até dois filhos para cada par de pais.}
	\end{figure}
	
	O primeiro passo é escolher dois cromossomos que farão o papel dos pais. Qualquer critério pode ser usado para compor esse par, como indivíduos que são diferentes geneticamente, ou agrupando os melhores e os piores separadamente. Porém, a escolha aleatória é a utilizada na maioria dos livros e aplicações, e, dada a sua simplicidade, será ela a abordada aqui. Tendo os pais em mãos, obtemos também aleatoriamente o ponto de corte, e estamos prontos para a reprodução (figura \ref{figCrossOver}). 
	
	Entretanto, voltando à Natureza, sabemos que a reprodução não acontece necessariamente sempre, e o nosso algoritmo deve contemplar esse efeito. Fazemos isso através de uma probabilidade $p_c$ associada à ocorrência do \textit{Crossover}. Digamos que $p_c = 70\%$. Após escolhidos os pais e o ponto de corte, sorteamos um número $p_{aux}$ entre $0$ e $1$ e, se $p_{aux} <= p_c$, o \textit{Crossover} acontece.
	
	Esse processo deve ser repetido até que o número de indivíduos desejado seja atingido. Nas figuras \ref{figCodLoopCrossOver} e \ref{figCodCrossOver} há um exemplo de código.
	
	\subsubsection{Exemplo: crossover para $f(x) = -x^2 + 36$}
	
	Para que o funcionamento do operador \textit{crossover} fique claro, faremos uma execução manual do código contido na figura \ref{figCodLoopCrossOver}. Os indivíduos que utilizaremos serão aqueles selecionados pela Roleta, presentes na tabela \ref{tabRoletaManual}. Ao final teremos quatro indivíduos provenientes de oito pares de cromossomos.
	
	Entrentando, antes de continuarmos, explicaremos o algoritmo.	A segunda linha, 248, inicializa a variável \textit{\texttt{flagCrossOver}} com zero, ou \texttt{falso} na linguagem C. A função dela é garantir que em todas as iterações do laço \textit{\texttt{for}} sairemos com um descendente.
	
	Entramos no \textit{\texttt{while}} e chamamos a função \textit{\texttt{GeraPontoDeCorte()}}, cujo trabalho é definir o ponto de corte e armazená-lo em uma variável global. Conforme pode ser visto na figura \ref{figCodCrossOver}, essa variável (\textit{\texttt{PontoDeCorte}}) é utilizada na função \textit{\texttt{CrossOver()}}.
	
	Cada elemento do vetor \textit{\texttt{geracao}} (linhas 251 e 252) armazena uma população com um número fixo de indivíduos, definida em \textit{\texttt{numindividuos}}. Então, para obter um cromossomo aleatoriamente, geramos, através da função \textit{\texttt{Randomico()}}, um valor entre zero e (\textit{\texttt{numindividuos}} - 1). Esse número será a posição do indivíduo escolhido dentro da geração. 	As variáveis \textit{\texttt{indPai}} e \textit{\texttt{indMae}} receberão os dois indivíduos. 
	
	Em \textit{\texttt{probAux}} guardamos um valor entre 0 e 1, obtido novamente de maneira aleatória. Se ele for menor ou igual a \textit{\texttt{probCrossOver}} (probabilidade de ocorrer a Reprodução), os indivíduos \textit{\texttt{indPai}} e \textit{\texttt{indMae}} gerarão um filho, que será armazenado na posição \textit{\texttt{iIndividuo}} da \textit{\texttt{geracao\_auxiliar}}. A \textit{\texttt{flagCrossOver}} recebe \textit{\texttt{verdadeiro}}, indicando que temos um descendente e podemos sair do \textit{\texttt{while}}.
	
	Voltemos à execução manual do nosso exemplo. Começamos definindo que a probabilidade de ocorrência da Reprodução será de 70\%\footnote{Não há um valor ótimo e absoluto para $p_c$. Para cada caso deve-se ajustar esse parâmetro e verificar a qualidade das soluções. Podemos afirmar apenas que $p_c$ deve ser alta, entre 70\% e 100\%.}:
	
	$$
		probCrossOver = 0,7.
	$$	
	
	Através da tabela \ref{tabRoletaManual} lembramos que 
	
	$$
		numindividuos = 4.
	$$
	
	Ao entrar no \textit{\texttt{for}} a \textit{\texttt{flagCrossOver}} recebe \textit{\texttt{falso}}:
	
	$$
		flagCrossOver = 0.
	$$

	Assim que chegamos ao \textit{\texttt{while}} a função \textit{\texttt{GeraPontoDeCorte()}} é executada. Como a nossa representação cromossomial possui três genes, há apenas dois pontos de corte possíveis (figura \ref{figCrossOver}). Digamos que a função escolha
	
	$$
		PontoDeCorte = 1,
	$$
	e que
	
	$$
		probAux = 0,86.
	$$
	
	Para esse caso, a condição \textit{\texttt{probAux <= probCrossOver}} é falsa, obviamente não entramos no \textit{\texttt{if}} e voltamos ao início do \textit{\texttt{while}}.
	
	Entretanto, suponha que agora os seguintes valores tenham sido obtidos:
	
	$$
		PontoDeCorte = 1 \mbox{  e  } probAux = 0,65.
	$$
	
	Na linha 251 \textit{\texttt{Randomico(0, numindividuos)}} retorna 0, e isso significa que \textit{\texttt{indPai}} recebeu $x = 3 = 011$:
	
	$$
		indPai = 011.
	$$
	
	Em seguida, \textit{\texttt{Randomico(0, numindividuos)}} retorna 3\footnote{Lembre-se que na linguagem C o primeiro elemento de um vetor possui índice \textit{i} = 0. Por isso o número 3 retornou o quarto elemento.}, e isso implica
	
	$$
		indMae = 101.
	$$
	
	\begin{figure}[htp]
		\begin{center}
			\includegraphics[width=13cm]{figs/ga/exemplo_crossover.png}
		\end{center}
		\caption{\label{figExemploCrossOver} Exemplo do \textit{crossover} entre os indivíduos 011 e 101 para o primeiro ponto de corte.}
	\end{figure}
	
	O resultado da reprodução, veja a figura \ref{figExemploCrossOver}, é armazenado na primeira posição da \textit{\texttt{geracao\_auxiliar}}:
	
	$$
		geracao\_auxiliar[0] = 001.
	$$
	
	Vamos refletir um pouco sobre esse resultado. Na geração atual (\textit{Geração Final} na tabela \ref{tabRoletaManual}) o indivíduo com melhor \textit{fitness} é o $x = 2$, enquanto $x = 5$ possui a pior nota e $x = 3$ obteve uma avaliação intermediária. Por causa do caráter completamente aleatório do algoritmo, o melhor indivíduo não foi selecionado para gerar descendentes nessa reprodução. Poderíamos supor, num primeiro momento, que esse não tenha sido o melhor caminho. Entretando, o resultado do \textit{crossover} entre $x = 5$ e $x = 3$ gerou um indivíduo com o maior \textit{fitness} desde a geração inicial! Usando $x = 1$ na $f(x)$ chegamos a $f_c(1) = 35$, maior do que o $f(2) = 32$.
	
\begin{table}[htp]
	\caption{\label{tabCrossoverManual}Geração antes e depois do \textit{Crossover}. O melhor indivíduo dos descendentes possui o melhor \textit{fitness} entre todas as gerações anteriores. Além disso, o \textit{fitness} médio ($<n>$) também aumentou.}
	\begin{center}
		\begin{tabular}{c|c|c|c}
			\hline
			\multicolumn{2}{c|}{\textbf{Geração Selecionada}} &  \multicolumn{2}{c}{\textbf{Descendentes}}  \\
			\hline
			x 					& \textit{fitness} (n)	& x						& \textit{fitness} (n)	\\
			\hline
			3 					& 27										& 1						&	35 \\
			2 					& 32 										& 5						&	11 \\
			3 					& 27										& 3						&	27 \\	
			5 					& 11										& 3						&	27\\
			\hline
			\multicolumn{2}{c|}{$<n>$ = 24,25} & \multicolumn{2}{c}{$<n>$ = 25,00}  \\
			\hline
		\end{tabular}
	\end{center}
\end{table}
	
	A tabela \ref{tabCrossoverManual} apresenta os dados do nosso processo de Reprodução feito manualmente. É interessante notar que, apesar dessa nova geração ter perdido o indivíduo $x = 2$, a maior avaliação da população anterior, o \textit{fitness} médio aumentou. Não só isso, mas também o melhor cromossomo tem uma nota maior do que todos os indivíduos anteriores. 
	
	Apesar de simples, o exemplo acima exibiu o papel fundamental do \textit{Crossover} na busca pelas melhores soluções. Na próxima seção mostraremos como funciona a Mutação, essencial para a variabilidade genética. 
	
	\begin{figure}[htp]
		\begin{center}
			\includegraphics[width=13cm]{figs/ga/CodigoLoopCrossOver.png}
		\end{center}
		\caption{\label{figCodLoopCrossOver} Código para a Reprodução. Nesse exemplo o algoritmo gera apenas um descendente. Detalhes da função \texttt{CrossOver()} estão na figura \ref{figCodCrossOver}.}
	\end{figure}

\begin{figure}[htp]
		\begin{center}
			\includegraphics[width=13cm]{figs/ga/CodigoCrossOver.png}
		\end{center}
		\caption{\label{figCodCrossOver}Detalhes da função \texttt{CrossOver()}.}
	\end{figure}
	
	\subsection{\label{mutacao}Mutação}
		
	Chegamos à última operação de um algoritmo genético, a Mutação. Assim como no \textit{Crossover}, existe uma probabilidade $p_m$ associada ao acontecimento da Mutação. Porém, nesse momento, não operaremos sobre um par de cromossomos, mas na estrutura interna de cada cromosso: os genes.
	
	Se nossos indivíduos possuem dez genes, para cada um dos dez \textit{locus} testamos a condição $p_{aux} <= p_m$, onde $p_{aux}$ é um valor entre zero e um, escolhido aleatoriamente. Caso a desigualdade seja verdadeira, invertemos o \textit{bit} daquela posição e partimos para o próximo \textit{locus}.
	
	Seguindo com o exemplo da seção anterior, a população atual, depois da Seleção e do \textit{Crossover}, possui o melhor indivíduo comparado com os anteriores, além do \textit{fitness} médio também ter crescido (tabela \ref{tabCrossoverManual} na página \pageref{tabCrossoverManual}). Na tabela \ref{tabPopAntesMutacao} listamos esses indivíduos e a sua representação cromossomial.
	
	\begin{table}[htp]
 		\caption{\label{tabPopAntesMutacao}Representação cromossomial para os indivíduos que passaram pela Seleção e pelo \textit{Crossover}.}
 		\begin{center}
  		\begin{tabular}{c|c}
   			\hline
   			\textbf{x (decimal)}  & \textbf{x (representação binária)} \\
   			\hline
   			1 & 001 \\
   			5 & 101 \\ 
   			3 & 011 \\
   			3 & 011 \\
   			\hline
   		\end{tabular}
 		\end{center}
	\end{table}
	
	Do ponto de vista da informação genética, o indivíduo que mais se aproxima de $x = 0 = 000$, a solução ideal, é o $x = 1 = 001$. Portanto, bastaria evoluir essa população e, em algum momento, obteríamos a melhor solução, certo? Errado.
	
	Nenhum indivíduo possui um zero na última posição. Então, mesmo fazendo todas as combinações possíveis entre os cromossomos, nunca chegaríamos ao indivíduo $x = 0$. O único que possuía tal característica era o $x = 2$, presente na população inicial mas ``extinto'' ao longo da evolução.
	
	É aí que entra a Mutação: ainda que baixa, geralmente em torno de 1\%, há uma chance do último \textit{bit} de algum indivíduo sofrer mutação e ser alterado para zero. E esse é um dos pontos que torna a Mutação tão importante: ela insere uma nova informação genética que, ou foi perdida, ou não estava presente na população inicial (Heurística Exploratória).
	
	\begin{figure}[htp]
		\begin{center}
			\includegraphics[width=13cm]{figs/ga/Mutacao.png}
		\end{center}
		\caption{\label{figMutacao}Representação gráfica de uma mutação. Nesse exemplo a mutação no último \textit{bit} levou à solução ótima para o máximo da função $f(x) -x^2 + 36$.}
	\end{figure}
	
	Obviamente, existe a possibilidade de bons esquemas serem destruídos com a mudança em algum ``\textit{bit} errado''. Mesmo assim, essa chance é equivalente à probabilidade de transformarmos um péssimo esquema em um excelente espaço de soluções. Nessa linha, a Mutação é a responsável por manter a diversidade e evitar a Convergência Genética (seção \ref{ExemploVariabilidade}).
	
	Qual deve ser o valor de $p_m$ para uma Mutação eficiente? Não há uma resposta absoluta. Muitos pesquisadores e usuários dos GAs utilizam valores entre 0,5\% e 1\%, simplesmente porque nos primeiros trabalhos bons resultados foram obtidos com eles. Sabe-se que há um valor ideal, mas ele é diferente para cada problema e para cada representação cromossomial.
	
	Apesar disso, há um consenso de que o valor de $p_m$ deve ser pequeno, bem menor do que $p_c$. Em parte porque na genética natural essa probabilidade é de fato pequena, mas principalmente, no contexto da computação, porque valores grandes fariam com que a busca por soluções se comportasse de maneira semelhante ao \textit{Random Walk}.
	
	Partindo para o lado prático, o código para implementar a mutação é mais simples, e um exemplo encontra-se na figura \ref{figCodMutacao}. Na linha 281 um determinado indivíduo recebe ele próprio após o operador Mutação. Não há \textit{\texttt{if}} ou outra condição, ou seja, aplicamos o operador em todos os indivíduos da população.
	
	\begin{figure}[htp]
		\begin{center}
			\includegraphics[width=13cm]{figs/ga/CodigoMutacao.png}
			\includegraphics[width=13cm]{figs/ga/CodigoMutacaoFunc.png}
		\end{center}
		\caption{\label{figCodMutacao}Exemplo de código para o operador Mutação.}
	\end{figure}
		
	Mas, afinal, onde está a probabilidade $p_m$? Lembre-se que a mutação, caso aconteça, deve ocorrer isoladamente em cada gene. Por isso usamos $p_m$ dentro da função \textit{\texttt{Mutacao()}}. Ela recebe como parâmetro um indivíduo que possui um número de genes igual a \textit{\texttt{numGenes}}, uma constante definida de maneira global.
	
	Entramos em um \textit{\texttt{for}} que irá varrer todos os genes do \textit{\texttt{Individuo}} e, para cada um, faz o teste $p_{aux} <= p_m$. Se a condição retornar \textit{\texttt{verdadeiro}}, a Mutação é finalmente expressa como a inversão do \textit{bit} no \textit{locus} atual.
	
	Depois que todos os indivíduos da população passarem pelo operador Mutação, o ciclo estará fechado. A nova geração está pronta para passar por todo o processo: Avaliação, Seleção, Reprodução e Mutação.
	
	\section{Considerações finais}
	
	A Teoria da Evolução de Darwin causou uma das principais revoluções na ciência. Seu impacto não está restrito à Biologia, e os Algoritmos Genéticos (GAs), um ramo da Computação Evolutiva, são um bom exemplo disso.
	
	Fortemente inspirado no processo de Seleção Natural, todo GA possui uma representação chamada Cromossomial, o equivalente ao DNA. A população de soluções é avaliada para verificar quais são as ruins, as boas e as melhores. As mais adequadas têm maior chance de se reproduzirem e perpetuar seus genes.
	
	Do ponto de vista da Inteligência Artificial, os GAs são aplicados com sucesso em vários problemas. Alguns exemplos, dentre muitos, são o reconhecimento de padrões em visão computacional, movimentação de membros em robótica, criação de ambimentes mais realistas em jogos e treinamento de redes neurais. 
	
	Estruturalmente todo GA é semelhante, facilitando a modularização do código. Gera-se a população inicial que é avaliada e, em seguida, ela passa pela Seleção, \textit{Crossover} e Mutação. Novos indivíduos são criados e, depois de algumas gerações, chegamos a uma população com um conjunto de boas soluções.
	
	Mesmo com boa aplicabilidade, os GAs devem ser utilizados com cuidado. Dada a natureza aleatória do método, não há garantia em geral de uma solução ótima seja encontrada e, por isso, algoritmos que levam à soluções exatas têm prioridade.
	
	Porém, problemas combinatoriais muito grandes, como os NP Completos, são um excelente campo de utilização. Além desses, os GAs são eficientes na otimização de problemas multiobjetivos, como escala de horários ou planejamento de logística e produção.




\chapter{Autovalores de Matrizes Simétricas com Algoritmos Genéticos\label{sec:metodo}}

	Os métodos que estudei para esta dissertação estão contidos em uma série de artigos \cite{metodo2002, metodo2004, metodo2006, metodo2008, metodo2009, metodo2011}. O objetivo é encontrar, sequencialmente, do menor para o maior, os autovalores de uma matriz simétrica. Essa matriz é o Hamiltoniano presente na formulação matricial da equação de Schrödinger independente do tempo\footnote{Ela é importante para a física moderna pois está associada à quantização da energia na Mecânica Quântica.}. A estratégia é transformar o problema do autovalor em um problema de otimização, buscando um escalar $a_i$ e um vetor $X_i$ de modo que a equaçaõ $HX_i = a_iX_i$ seja satisfeita.
	
	Os algoritmos apresentados em \cite{metodo2004} e \cite{metodo2011} são mais simples. Por exemplo, em \cite{metodo2002} o Hamiltoniano é alterado por rotações de Jacobi e, só então, o \emph{fitness} é calculado. Nos artigos \cite{metodo2006}, \cite{metodo2008} e \cite{metodo2009} o espaço vetorial é dividido em duas partes de dimensões diferentes, levando a um Hamiltoniano que contém alguns autovalores de interesse\footnote{\emph{Partitioned matrix eigenvalue problem.}}. Isso não ocorre com os GAs das publicações de 2004 e 2011. Nelas o Hamiltoniano original é sempre mantido.
	
	Pode-se dizer que \cite{metodo2011} é a continuação de \cite{metodo2004}. A representação cromossomial e os operadores de seleção, \emph{crossover} e mutação são os mesmos. No entanto, ele adiciona dois operadores genéticos. O primeiro é complementar à mutação, e atua para criar mais diversidade na população. O segundo acentua a pressão seletiva via Elitismo. Não há justificativa para os novos operadores. Acredito que o intuito tenha sido melhorar a qualidade dos resultados, mas, infelizmente, não há comparação com os obtidos em 2004. E, de fato, isso seria impossível, pois em 2011 há uma mudança drástica: a função de avaliação foi alterada. Além disso, \cite{metodo2011} paraleliza o GA e compara os desempenhos.
	
	Assim, optei por seguir uma combinação entre \cite{metodo2004} e \cite{metodo2011}. Isso permitiu, por exemplo, comparar com segurança os resultados das duas funções de avaliação, algo que os autores não fizeram. A partir do próximo parágrafo descrevo o conteúdo dos dois artigos e como fiz a combinação entre os dois.

%========================================================	
%\section{Descrição do Algoritmo}
%========================================================	

	Há casos em que não há solução exata para a equação de Schrödinger independente do tempo. Quando isso acontece, é preciso introduzir uma base ortonormal finita {$\phi$} e expandir o estado estacionário $\psi$ em termos dos vetores geradores dessa base
	
	\begin{equation}
		\psi = \sum_k c_k \phi_k.
	\end{equation}
	
	Isso leva ao problema de autovalores
	
	\begin{equation}\label{eq:HCEC}
		HC = EC,
	\end{equation}
	onde $H$ é uma matriz real e simétrica, construída na base ${\phi}$, que representa o operador Hamiltoniano. A diagonalização de $H$ encontra os autovalores $E_n$ correspondentes às energias possíveis do sistema quântico. Com isso, é possível obter os autovetores (autoestados) $C_n$ associados.
	
	No artigo \cite{metodo2004} é apresentada uma maneira de reduzir o problema de autovalores a um problema de busca. Dados todos os vetores $C$ existentes na base $\phi$, conjunto chamado de \textbf{Espaço de Busca}, o objetivo é encontrar os autovetores $C_n$ que satisfaçam a equação \ref{eq:HCEC}. O conjunto de todos os autovetores $C_n$ é o \textbf{Espaço de Soluções}. O mecanismo de busca é um algoritmo genético (GA).
	
	Conforme visto no capítulo \ref{cap:ga}, o elo entre o GA e o problema a ser resolvido está na Representação Cromossomial e na Função de Avaliação. O cromossomo deve codificar a solução desejada na forma de um \emph{string}, seja de caracteres, símbolos, números inteiros ou reais. O \emph{fitness} deve ser capaz de definir, objetivamente, a qualidade de todos os indivíduos da população, de modo que seja possível comparar cada um com as soluções desejadas. Quanto mais próximo um cromossomo está da solução, mais alto deve ser seu \emph{fitness}.

%-------------------------------------------------------		
	\section{Representação Cromossomial}
%-------------------------------------------------------	
	
	Como a solução pretendida é um autovetor, o cromossomo codifica um vetor. Cada indivíduo $i$ da população é um vetor $\psi_i$ candidato a autovetor na forma
	
	\begin{equation}
		\psi_i = \sum_{p=1}^m c_{pi}\phi_p, \mbox{   } i = 1,2, \cdots, n
	\end{equation}
	
	Na equação acima, $i$ varia de 1 até o número máximo de indivíduos na população do GA, mantida fixa ao longo de toda a execução. O índice $p$ é tomado de 1 até a $m$, que é ordem da matriz $H$ (ou a dimensão do espaço vetorial).
	
	O cromossomo é definido como uma cadeia de números reais, cujos valores são os coeficientes $c_{pi}$. O \emph{string} $S_i$, codificação para o membro $\psi_i$, é dado por
	
	\begin{equation}
		S_i \equiv  (c_{1i}, c_{2i}, \cdots, c_{pi}, \cdots, c_{mi}) = C^{\dagger}_i,
	\end{equation}
	enquanto que para outro membro $\psi_k$ da população, o \emph{string} $S_k$ é
	
	\begin{equation}
		S_k \equiv  (c_{1k}, c_{2k}, \cdots, c_{pk}, \cdots, c_{mk}) = C^{\dagger}_k.
	\end{equation}
	
%-------------------------------------------------------
	\section{População}
%-------------------------------------------------------

	A população inicial é gerada aleatoriamente. Não há nenhuma preferência com relação aos valores iniciais de cada gene.
	
	Seu tamanho (número de indivíduos) é sempre mantido fixo a cada nova geração.
	
%-------------------------------------------------------
	\section{Funções de Avaliação (\emph{fitness})}
	\label{sec:fitness_metodo}
%-------------------------------------------------------	

	O \emph{fitness} proposto em \cite{metodo2004} é obtido em três passos:
	
	\begin{enumerate}
		\item \label{item:passo1} Cálculo do quociente de Rayleigh $\rho_i$ associado ao $i-$ésimo indivíduo $C_i$;
		\item \label{item:passo2} Cálculo do gradiente de $\rho_i$;
		\item \label{item:passo3} Cálculo da função de avaliação.
	\end{enumerate}
	
	Retomando o capítulo \ref{cap:algebra}, o quociente de Rayleigh é dado por
	
	\begin{equation}\label{eq:rho-GA}
		\rho_i = \frac{C_i^\dagger H C_i}{C_i^\dagger C_i},
	\end{equation}
	e seu gradiente por
	
	\begin{equation}\label{eq:grad_rho_metodo}
		\nabla \rho_i = \frac{2[H - \rho_i]C_i}{C_i^\dagger C_i}.
	\end{equation}
		
	O \emph{fitness} é então definido como \footnote{
		No artigo \cite{metodo2004} a equação originalmente apresentada é $f_i = e^{-\lambda (\nabla \rho_i)^{\nabla \rho_i}}$. Acredito que tenha sido um erro de digitação por dois motivos. O primeiro é que tal definição não leva, necessariamente, ao comportamento esperado: $f_i \rightarrow 1$ quando $\nabla \rho_i \rightarrow 0$. Em segundo lugar, nos artigos \cite{metodo2006},  \cite{metodo2008} e \cite{metodo2009}, que seguem o mesmo método, a função de avaliação é definida como $f_i = e^{-\lambda (\nabla \rho_i)^{\dagger} (\nabla \rho_i)}$. Portanto, segui com a definição de $f_i$ da equação \ref{eq:fitness_extenso}.}
	
	\begin{equation}\label{eq:fitness_extenso}
		f_i = e^{-\lambda (\nabla \rho_i)^\dagger (\nabla \rho_i)}.
	\end{equation}
	
	Lembrando que o módulo de um vetor $V$ é dado por $|V| = \sqrt{V^{\dagger} V}$, a equação \ref{eq:fitness_extenso} fica
	
	\begin{equation}\label{eq:fitness_grad}
		f_i(\nabla \rho_i) = e^{-\lambda |\nabla \rho_i|^2}.
	\end{equation}
			 
	A função de avaliação $f_i$ está limitada entre zero e um, $f_i$ = (0,1]. Se $|\nabla \rho|^2 \gg 0$, $f_i \rightarrow 0$, indicando que $C_i$ possui baixa qualidade, está longe da solução. Por outro lado, se $\nabla \rho_i \rightarrow 0 $, $f_i \rightarrow 1$, e $C_i$ é uma boa aproximação para um autovetor. O parâmetro $\lambda$ é escolhido para não haver \emph{over flow} ou \emph{under flow} da função exponencial.
	
	De acordo com os autores, a equação \ref{eq:fitness_grad} leva ao autovalor mínimo. Se algum $C_i$, em algum momento, é o autovetor fundamental $C_0$, o $\nabla \rho$ é nulo. Eles afirmam que ``\textit{Claramente, $f_i \rightarrow 1$ quando $\nabla \rho_i \rightarrow 0$, sinalizando que a evolução atingiu o verdadeiro autovetor fundamental de $H$ em $C_i$}''\footnote{Tradução livre de ``\textit{Clearly, $f_i \rightarrow 1$, as $\nabla \rho_i \rightarrow 0$, signalling that the evolution has hit the true ground state eigenvector of $H$ in the vector $C_i$}''.}. 
	
	Uma vez que $C_0$ foi encontrado, o próximo passo é obter o autovalor $E_0$ associado. Na verdade ele já foi calculado, e é simplesmente o valor do quociente de Rayleigh para $C_i$:
	
	\begin{equation}
		\rho_0 = \frac{C_i^{\dagger} H C_i}{C_i^{\dagger} C_i} = E_0.
	\end{equation}
	
	Quando o algoritmo chega nesse estágio tem-se o par $(C_0,E_0)$.
	
	Como já dito anteriormente, a função de avaliação foi alterada em \cite{metodo2011}, e é dada por 
	
	\begin{equation}\label{eq:fitness_EL_metodo}
		f_i(\rho_i) = e^{-\lambda(\rho_i - E_L)^2},
	\end{equation}
	onde $E_L$ é um limite inferior para o autovalor mínimo procurado.
	
	Ela compartilha algumas propriedades com a equação \ref{eq:fitness_grad}. Está limitada no conjunto $f_i(\rho) = (0,1]$ e, quanto maior seu valor, melhor a qualidade do indivíduo. O parâmetro $\lambda$ tem exatamente a mesma função. A busca utilizando $f_i(\rho)$ também termina quando $f_i \rightarrow 1$. ``\emph{Se $\rho_i \rightarrow E_L$ durante a busca, $f_i \rightarrow 1$ e $C_i$ se aproxima do autovetor fundamental de $H$}''\footnote{Tradução livre de ``If $\rho_i \rightarrow E_L$ during the search, $f_i \rightarrow 1$ and $C_i$ approaches the ground eigenvector of $H$''.}. Novamente, como já temos $C_0$, o autovalor $E_0$ é simplesmente o $\rho_i$ já calculado.	O cálculo de $f_i(\rho)$ executa os passos \ref{item:passo1} e \ref{item:passo3} de $f_i(\nabla\rho)$.
				
		A não ser por acidente, a condição $f_i \rightarrow 1$ não é satisfeita logo na primeira população. É necessário evoluir os indivíduos por meio dos operadores genéticos de seleção, reprodução e mutação. Eles serão apresentados nas seções seguintes.

%-------------------------------------------------------
\section{Seleção}
%-------------------------------------------------------	

			O operador de seleção utilizado tanto em \cite{metodo2004} quanto em \cite{metodo2011} é o da Roleta, com fatias proporcionais aos valores do \emph{fitness} \cite{Linden2008}. Se a população possui $q$ indivíduos, a roleta é ``girada'' $q$ vezes, de modo a criar a nova população com os $q$ cromossomos selecionados.
			
			Entretanto, utilizei a seleção por Torneio pelos motivos apresentados na seção \ref{sec:torneio}. Mantive o tamanho da população fixa.
									
%-------------------------------------------------------
\section{Reprodução}
%-------------------------------------------------------

	A operação de reprodução ($crossover$) é aplicada na nova população após a Seleção. Há diferença entre os operadores de reprodução de \cite{metodo2004} e \cite{metodo2011}. Apresentarei ambos e, ao final, justificarei porque escolhi como base o segundo.

%----------------------------------------------------	
\subsection{\emph{Crossover} em \cite{metodo2004}}
%----------------------------------------------------	
	
	Suponha que tenham sido escolhidos, aleatoriamente, um par de cromossomos ($S_k$, $S_l$) dentre todos os $N$ indivíduos da população:
	
	\begin{equation}
		\begin{array}{l}
			S_k = (c_{k1}, c_{k2}, \cdots, c_{kn})	\\
			S_l = (c_{l1}, c_{l2}, \cdots, c_{ln})	
		\end{array}
	\end{equation}

	Em seguida, um inteiro $p$ é obtido, também aleatoriamente, entre 1 e $n - 1$ [p = [$1$, $n$)]. Lembre-se que $n$ é a ordem da matriz $H$ e, portanto, a quantidade de coeficientes no \emph{string} $S_i$. A função de $p$ é determinar em qual posição (\emph{locus}) do cromossomo acontecerá a alteração.
	
	O operador cria o novo par ($S^{'}_k$, $S^{'}_l$):
	
	\begin{equation}
		\begin{array}{l}
			S^{'}_k = (c_{k1}, c_{k2}, \cdots, c^{'}_{kp} c_{k,p+1}, \cdots, c_{kn})	\\
			S^{'}_l = (c_{l1}, c_{l2}, \cdots,  c^{'}_{lp} c_{l,p+1}, \cdots, c_{ln}),	\\
			
		\end{array}
	\end{equation}
	onde
	
	\begin{equation}
		\begin{array}{l}
			c^{'}_{kp} = f c_{kp} + (1 - f) c_{lp}     \\
			c^{'}_{lp} = (1 - f) c_{kp} + f c_{lp}.
		\end{array}
	\end{equation}
	
	O parâmetro $f$ faz o papel da mistura que cria nova informação. Ele é gerado aleatoriamente com valores entre zero e um. Nesse caso os valores limite não estão inclusos [f = (0,1)]. Dessa maneira há garantia de mistura.
	
	Esse operador só age em uma certa fração da população, dada por uma probabilidade $p_c$ que, em geral, é grande (70$-$75\%). O restante da população não é alterado.

%-----------------
\textbf{Exemplo}
%-----------------

Na figura \ref{fig:cross2004_tabelaAntes} há dois cromossomos de tamanho $n = 6$. A posição onde ocorrerá a alteração dos valores foi obtida aleatoriamente entre $p = [1,6)$, e, para esse caso, vale $p = 4$. O valor de $S_k$ na posição $p = 4$ é $c_{k4} = 0,80$, e para $S_l$ é $c_{l4} = 0,39$. Para $p + 1$ os valores são $c_{k5} = 0,15$ e $c_{l5} = 0,89$.

\begin{figure}[htbp]
	\centering
		\includegraphics[width=0.70\textwidth]{figs/materiais_metodo/autovalores_com_ga/cross2004_tabelaAntes.png}
	\caption{Exemplo do \emph{crossover} de \cite{metodo2004}. Indivíduos antes da reprodução.}
	\label{fig:cross2004_tabelaAntes}
\end{figure}

O parâmetro $f$ teve seu valor determinado aleatoriamente como $f = 0,62$. Os valores de $c^{'}_{k4}$ e $c^{'}_{l4}$ ficam:

\begin{equation}\label{eq:ck4}
	\begin{array}{lccl}
		c^{'}_{kp} = & f c_{kp} & + & (1 - f) c_{lp} 						\\
		c^{'}_{k4} = & 0,62 c_{k4} & + & (1 - 0,62) c_{l4} 			\\
		c^{'}_{k4} = & 0,62 * 0,80 & + & 0,38 * 0,39 						\\		
		c^{'}_{k4} = & 0,496 & + & 0,1482	\\		
		c^{'}_{k4} = & 0,6442 &  & 
	\end{array}
\end{equation}

\begin{equation}\label{eq:cl4}
	\begin{array}{lccl}
		c^{'}_{lp} = & (1-f) c_{kp} & + & f c_{lp} 						\\
		c^{'}_{l4} = & (1-0,62) c_{k4} & + & 0,62 c_{l4} 						\\
		c^{'}_{l4} = & 0,38 * 0,80 & + & 0,62 * 0,39 						\\
		c^{'}_{l4} = & 0,304 & + & 0,2418 						\\
		c^{'}_{l4} = & 0,5458 &  & 
	\end{array}
\end{equation}

Finalmente, os valores para a posição $p = 4$ nos novos indivíduos $S^{'}_k$ e $S^{'}_l$ são

\begin{equation}\label{eq:cross2004_novo_valor_sk}
	\begin{array}{ll}
	\mbox{Novo valor na posição 4 de  } S^{'}_k & = c^{'}_{kp} c_{k,p+1} \\
								& = c^{'}_{k4} c_{k5} \\
								& = 0,6442 * 0,15	\\
								& = 0,09663	\\
								& \approx 0,1
	\end{array}
\end{equation}

\begin{equation}\label{eq:cross2004_novo_valor_sl}
	\begin{array}{ll}
	\mbox{Novo valor na posição 4 de  } S^{'}_l & = c^{'}_{lp} c_{l,p+1} \\
								& = c^{'}_{l4} c_{l5} \\
								& = 0,5458 * 0,89	\\
								& = 0,485762 \\
								& \approx  0,49
	\end{array}
\end{equation}


Na figura \ref{fig:cross2004_tabelaDepois} há os dois novos indivíduos.

\begin{figure}[htbp]
	\centering
		\includegraphics[width=0.70\textwidth]{figs/materiais_metodo/autovalores_com_ga/cross2004_tabelaDepois.png}
	\caption{Exemplo do \emph{crossover} de \cite{metodo2004}. Indivíduos depois da reprodução.}
	\label{fig:cross2004_tabelaDepois}
\end{figure}

%----------------------------------------------------
\subsection{\emph{Crossover} em \cite{metodo2011}}	
%----------------------------------------------------

	A representação cromossomial usada em \cite{metodo2011} é a mesma de \cite{metodo2004}, portanto, o par ($S_k$, $S_l$) é igual, assim como a probabilidade $p_c$:
	
	\begin{equation}
		\begin{array}{l}
		S_k = (c_{k1}, c_{k2}, \cdots, c_{kn})	\\
		S_l = (c_{l1}, c_{l2}, \cdots, c_{ln})	
		\end{array}
	\end{equation}
	
	Diferentemente de \cite{metodo2004}, utiliza-se o \emph{crossover} de dois pontos. Aleatoriamente escolhe-se dois inteiros, $o$ e $p$, cuja função é determinar a região do cromossomo que sofrerá miscigenação. O valor em $o$ indica o primeiro gene, e $p$ o último. Portanto, todos os genes entre os dois, incluindo eles próprios, sofrerão a ação do \emph{crossover} ($o <= c_i <= p$, $p \geq o$). Os novos indivíduos são ($S^{'}_k$, $^{'}S_l$)
	
	\begin{equation}
		\begin{array}{l}
			S^{'}_k = (c_{k1}, c_{k2}, \cdots, c^{'}_{ko}, \cdots , c^{'}_{kp}, c_{k,p+1}, \cdots, c_{kn})	\\
			S^{'}_l = (c_{l1}, c_{l2}, \cdots, c^{'}_{lo}, \cdots , c^{'}_{lp}, c_{l,p+1}, \cdots, c_{ln}).	
		\end{array}
	\end{equation}
	
	Para todos genes selecionados, a transformação ocorre da seguinte maneira ($i = o, o + 1, \cdots, p$):
	
	\begin{equation}
		\begin{array}{l}
			c^{'}_{ki} = f_c c_{ki} + (1 - f_c) c_{li}     \\
			c^{'}_{li} = (1 - f_c) c_{ki} + f_c c_{li}
		\end{array}
	\end{equation}
	onde $f_c$ é dado por
	
	\begin{equation}
		f_c = 0,75 + 0,25r,
	\end{equation}
	sendo $r$ um número aleatório ($0 \leq r \leq 1$). Assim como o parâmetro $f$ de \cite{metodo2004}, $f_c$ faz o papel da mistura que cria nova informação.
	
	\textbf{Exemplo}.
	
	
	Usei os mesmos indivíduos da figura \ref{fig:cross2004_tabelaAntes}. Os parâmetros $o$ e $p$ foram escolhidos no intervalo [1,6] e têm valores $o = 2$ e $p = 4$. Portanto, no \emph{string} $S_k$ os elementos que sofrerão alteração são $c_{k2} = 0,47$, $c_{k3} = 0,52$ e $c_{k4} = 0,80$. Eles se misturarão com os elementos $c_{l2} = 0,33$, $c_{l3} = 0,37$ e $c_{l4} = 0,39$ de $S_l$. Aleatoriamente obtive $r = 0,2$, que leva a $f_c = 0,80$.
	
	\begin{figure}[htbp]
	\centering
		\includegraphics[width=0.70\textwidth]{figs/materiais_metodo/autovalores_com_ga/cross2011_tabelaAntes.png}
	\caption{Exemplo do \emph{crossover} de \cite{metodo2011}. Indivíduos antes da reprodução.}
	\label{fig:cross2011_tabelaAntes}
\end{figure}
	
	Os elementos $c^{'}_{ki}$ são:
	
	\begin{equation}
		\begin{array}{llcl}
			c^{'}_{k2}	& = f_c c_{k2} 		& + & (1- f_c) c_{l2} \\
									& = 0,80 * 0,47		& + &	(1 - 0,80) * 0,33 \\
									& = 0,376					& + & 0,2 * 0,33	\\
									& = 0,376					& + & 0,066	\\
									& = 0,442 \\
									& \approx 0,44
		\end{array}
	\end{equation}
	
	
	\begin{equation}
		\begin{array}{llcl}
			c^{'}_{k3}	& = f_c c_{k3} 		& + & (1- f_c) c_{l3} \\
									& = 0,80 * 0,52		& + &	(1 - 0,80) * 0,37 \\
									& = 0,416					& + & 0,2 * 0,37	\\
									& = 0,416					& + & 0,074	\\
									& = 0,49
		\end{array}
	\end{equation}
	
	
	\begin{equation}
		\begin{array}{llcl}
			c^{'}_{k4}	& = f_c c_{k4} 		& + & (1- f_c) c_{l4} \\
									& = 0,80 * 0,80		& + &	(1 - 0,80) * 0,39 \\
									& = 0,64					& + & 0,2 * 0,39	\\
									& = 0,64					& + & 0,078	\\
									& = 0,718 \\
									& \approx 0,72
		\end{array}
	\end{equation}
	
	
	Os elementos $c^{'}_{li}$ são:
	
	\begin{equation}
		\begin{array}{llcl}
			c^{'}_{l2}	& = (1 - f_c) c_{k2} 		& + & f_c c_{l2} \\
									& = (1 - 0,80) * 0,47		& + &	0,80 * 0,33 \\
									& = 0,2 * 0,47					& + & 0,264	\\
									& = 0,094					& + & 0,264	\\
									& = 0,358 \\
									& \approx 0,36
		\end{array}
	\end{equation}
	
	\begin{equation}
		\begin{array}{llcl}
			c^{'}_{l3}	& = (1 - f_c) c_{k3} 		& + & f_c c_{l3} \\
									& = (1 - 0,80) * 0,52		& + &	0,80 * 0,37 \\
									& = 0,2 * 0,52					& + & 0,296	\\
									& = 0,104								& + & 0,296	\\
									& = 0,40									
		\end{array}
	\end{equation}
	
	
	\begin{equation}
		\begin{array}{llcl}
			c^{'}_{l4}	& = (1 - f_c) c_{k4} 		& + & f_c c_{l4} \\
									& = (1 - 0,80) * 0,80		& + &	0,80 * 0,39 \\
									& = 0,2 * 0,80					& + & 0,312	\\
									& = 0,16								& + & 0,312	\\
									& = 0,472 \\
									& \approx 0,47
		\end{array}
	\end{equation}
	
	Os novos indivíduos estão na figura \ref{fig:cross2011_tabelaDepois}.
		
	\begin{figure}[htbp]
	\centering
		\includegraphics[width=0.70\textwidth]{figs/materiais_metodo/autovalores_com_ga/cross2011_tabelaDepois.png}
	\caption{Exemplo do \emph{crossover} de \cite{metodo2011}. Indivíduos depois da reprodução.}
	\label{fig:cross2011_tabelaDepois}
\end{figure}
	
	
%----------------------------------------------------
\subsection{\emph{Crossover} utilizado}
\label{sec:crossover_utilizado}
%----------------------------------------------------

	A quantidade de nova informação não é escalável com o tamanho do cromossomo no \emph{crossover} de \cite{metodo2004}. Ele altera apenas uma posição (locus) de cada indivíduo, independentemente do tamanho do cromossomo. Por exemplo, se a ordem do Hamiltoniano for $n = 100$, o \emph{string} terá 100 elementos, mas apenas um sofrerá alteração. O mesmo aconteceria para $n = 1.000$, 10.000 e assim por diante.
	
	Ainda sobre o artigo de 2004, é impossível haver troca de informação na última posição. O parâmetro $p$, obtido aleatoriamente, dá a posição na qual ocorrerá o \emph{crossover}. Porém, parte do cálculo envolve os elementos $c_{k,p+1}$ e $c_{l,p+1}$. Note o índice $p+1$. Como a posição máxima é $n$, o maior índice possível para estes elementos é $p + 1 = n$, impondo o limite $p \leq n - 1$. Nos exemplos apresentados, $n = 6$, $p \leq 5$ e, portanto, a posição $6$ nunca será escolhida.
	
	Isso pode dificultar a busca. Suponha que uma solução precise do valor 23 no último coeficiente do cromossomo ($c_n = 0,23$). Se nenhum indivíduo da população inicial já tiver nascido com ele, esse valor nunca aparecerá por meio do \emph{crossover}. Resta como esperança a Mutação, porém, por definição, ela tem baixa probabilidade \cite{Linden2008}. Também não há garantia de que, após muitas gerações com sucessivas operações de \emph{crossover}, mutação e seleção, haverá um indivíduo na população que, além de ter 0,23 na última posição, possua os outros $(n-1)$ primeiros elementos necessários.
		
		O \emph{crossover} de \cite{metodo2004} é um caso especial do \emph{crossover} de \cite{metodo2011}. O operador de reprodução de \cite{metodo2011} é o clássico \emph{crossover} de dois pontos \cite{Linden2008}. Nele, $p \geq o$, e, se $o = p$, há mistura em apenas uma posição do \emph{string}, exatamente o que acontece em \cite{metodo2004}. Como $1 \leq o \leq n$ e $1 \leq p \leq n$, o último gene também está sujeito à operação, e o problema citado no parágrafo anterior não existe.
		
	Pelo que expus acima, escolhi utilizar o \emph{crossover} de \cite{metodo2011}, mas com uma pequena modificação. No cálculo dos novos coeficientes $c^{'}_{i}$, em vez do parâmetro $f_c$, usei o $f$ conforme definido em \cite{metodo2004}. O $f$ pode variar entre $0$ e $1$, enquanto o $f_c$, além de necessitar do parâmetro adicional $r$, é limitado apenas entre $0,75$ e $1$. Assim, acredito que $f$ seja mais abrangente como parâmetro de mistura e criação de nova informação.
	
	\textbf{\emph{Crossover} utilizado}
	
	A seguir apresento sinteticamente a estrutura do \emph{crossover} utilizado. Um par de indivíduos $(S_k, S_l)$
	
	\begin{equation}
		\begin{array}{l}
			S_k = (c_{k1}, c_{k2}, \cdots, c_{kn})	\\
			S_l = (c_{l1}, c_{l2}, \cdots, c_{ln})	
		\end{array}
	\end{equation}
	é obtido aleatoriamente da população. Com probabilidade $p_c$, a operação de \emph{crossover} acontece. Dois pontos de corte $o$ e $p$ são obtidos, também aleatoriamente, gerando os novos \emph{strings} $S^{'}_k$ e $S^{'}_l$ na forma
		
	\begin{equation}
		\begin{array}{l}
			S^{'}_k = (c_{k1}, c_{k2}, \cdots, c^{'}_{ko}, \cdots , c^{'}_{kp}, c_{k,p+1}, \cdots, c_{kn})	\\
			S^{'}_l = (c_{l1}, c_{l2}, \cdots, c^{'}_{lo}, \cdots , c^{'}_{lp}, c_{l,p+1}, \cdots, c_{ln}),
		\end{array}
	\end{equation}
	onde
	
	\begin{equation}\label{eq:recombinacao}
		\begin{array}{l}
			c^{'}_{ki} = f c_{ki} + (1 - f) c_{li}     \\
			c^{'}_{li} = (1 - f) c_{ki} + f c_{li}
		\end{array}
	\end{equation}
	e
	
	\begin{equation}
	0 \leq f \leq 1 \mbox{       (obtido aleatoriamente)}
	\end{equation}

%-------------------------------------------------------
\section{Mutação}
%-------------------------------------------------------

	O operador Mutação é semelhante nos dois artigos. Todos os cromossomos estão sujeitos à mutação. Se um indivíduo $k$ sofre mutação no gene $q$, o antigo valor $c^{'}_{kq}$ é alterado para $c^{''}_{kq}$ com a seguinte equação:

		\begin{equation}\label{eq:mutacao}
			c^{''}_{kq} = c^{'}_{kq} + (-1)^{L} r \Delta,
		\end{equation}
		onde $L$ é um número inteiro, $r$ um número aleatório ($0 \leq r \leq 1$) e $\Delta$ é a intensidade da mutação.
		
		
	Não ficou claro para mim qual a relação entre a probabilidade $p_m$ e como será a mutação no artigo \cite{metodo2004}. Os autores dizem que todos os \emph{indivíduos} estão sujeitos ao operador, mas não citam quais \emph{genes} podem sofrer mutação. Essa informação está explícita em \cite{metodo2011}, que permite mutação, caso aconteça, em apenas um gene de cada indivíduo. Ou seja, se houver mutação logo no primeiro gene de um cromossomo, o algoritmo parte para o próximo indivíduo.
	
	Assumi que no artigo \cite{metodo2004} os autores utilizaram o operador clássico. Conforme a literatura \cite{Mitchell98, Linden2008} o operador clássico de mutação age com probabilidade $p_m$ em cada gene, de maneira individual e independente. Ou seja, se um gene $c_{k,j}$ sofreu ou não mutação, essa informação não é utilizada para avaliar a ocorrência de mutação no próximo gene $c_{k,j+1}$. Portanto, a probabilidade de cada \emph{gene} sofrer mutação é $p_m$, mas a probabilidade $P_m$ de um \emph{indivíduo} sofrer mutação é, na verdade, $P_m = n*p_m$. Quanto maior o $n$, mais provável ocorrer a mutação, o que me parece razoável. Mutação é causada principalmente por erros de cópia do DNA. Quanto maior a fita, mais chance de acontecer erro.

	Há diferença na intensidade da mutação $\Delta$. No trabalho de 2004 ela é pequena ($10^{-2}-10^{-3}$) e mantida constante. Já no segundo artigo ela é proporcional ao melhor \emph{fitness} $f_t$ da geração atual $t$, e dada pela equação \ref{eq:Delta2011}. Como o \emph{fitness} começa pequeno e termina próximo de 1, $\Delta \rightarrow 0$ à medida que $f \rightarrow 1$.

\begin{equation}\label{eq:Delta2011}
	\Delta^{(t)}_m =  1 - f_t.
\end{equation}	

	Não fiquei confortável com a definição da equação \ref{eq:Delta2011}, por isso usei a intensidade $\Delta$ constante de \cite{metodo2004}. No início de um GA a criação de variabilidade genética é dominada pelo \emph{crossover}. No final, como os indivíduos são parecidos, fica a cargo da mutação fazer isso. A intensidade da mutação deve ser pequena e, se não for constante, é desejável que cresça com o tempo, justamente porque o papel da mutação passa a ser dominante na variabilidade comparado com o \emph{crossover} \cite{Linden2008}. Os autores de \cite{metodo2011} fazem o inverso. Como dito anteriormente, $\Delta$ diminui com o tempo, mas eles não justificam porque optaram por esse comportamento.
	
	Com relação à probabilidade de mutação, optei novamente a favor de \cite{metodo2004}. Ambos os trabalhos utilizam valor constante de $p_m$. Entretanto, em \cite{metodo2011} $p_m$ é inversamente proporcional ao tamanho do cromossomo $n$ (equação \ref{eq:probM2011}). Na prática, quanto maior a ordem $n$ do Hamiltoniano, menor a probabilidade de mutação, e no artigo não há justificativa para essa escolha. Fiquei com a probabilidade constante, discutida em livros tradicionais de GA \cite{Mitchell98, Linden2008}.

	\begin{equation}\label{eq:probM2011}
		p_m = 4.0/n
	\end{equation}


	\textbf{Mutação utilizada.}
	
	Abaixo resumo a estrutura da Mutação implementada na dissertação.
	
	
	\begin{itemize}
		\item \textbf{Probabilidade de mutação $p_m$}
			\begin{itemize}
				\item Constante.
				\item Um dos parâmetros do algoritmo.
				\item Pode ter qualquer valor $0 \leq p_m \leq 1$.
			\end{itemize}
		\item \textbf{Equação da mutação}.
			\begin{itemize}
				\item No indivíduo $S$, transforma o valor $c^{'}_{kq}$ do gene $q$ no novo valor $c^{''}_{kq}$.
				\item Equação: $c^{''}_{kq} = c^{'}_{kq} + (-1)^{L} r \Delta$
				\item Parâmetro $L$: inteiro aleatório.
				\item Parâmetro $r$: aleatório ($0 \leq r \leq 1$).
				\item Intensidade de mutação $\Delta$: Constante. Valores pequenos ($10^{-3}-10^{-2}$).
			\end{itemize}
	\end{itemize}
	

%-------------------------------------------------------
\section{Fluxograma do algoritmo implementado}
%-------------------------------------------------------

	O algoritmo genético seguiu a estrutura de um GA básico \cite{Mitchell98, Linden2008}. Após gerar a população inicial, Avaliação, Seleção, \emph{Crossover} e Mutação são executados nessa sequência até que uma condição de parada seja atingida.
	
	Na figura \ref{fig:fluxo} está o fluxograma. Ele não é um diagrama completo do \emph{software} desenvolvido, mas é útil para ter a visão geral do algoritmo.

\begin{figure}[htbp]
	\centering
		\includegraphics[width=1.00\textwidth]{figs/materiais_metodo/autovalores_com_ga/fluxo.png}
	\caption{Fluxo algoritmo genético}
	\label{fig:fluxo}
\end{figure}
\chapter{Materiais e Métodos\label{cap:metodologia}}

	A metodologia teve três pilares: sólida base em Algoritmos Genéticos, estudo detalhado dos artigos e desenvolvimento de um \emph{software} confiável para execução do modelo. Os três são apresentados nas seções seguintes.

%========================================================
\section{Algoritmos Genéticos}\label{seq:medologia_ga}
%========================================================
	
	Iniciei os estudos em Algoritmos Genéticos pelos livros \cite{Mitchell98} e \cite{Linden2008} e, antes de partir para o trabalho da dissertação em si, ataquei três problemas completamente distintos.
	
	O primeiro foi o ONEMAX \cite{onemaxNaGPU}, desenvolvido ``do zero'', em Linguagem C. Considerado o \emph{hello world} do GA, tem representação cromossomial binária, o \emph{fitness} é a soma dos \emph{bits} de cada indivíduo e o objetivo é obter um indivíduo com o maior número de '1' possível. Com ele pude estudar os parâmetros de um GA simples, como número de indivíduos e a probabilidade de \emph{crossover}, e verificar a influência de cada um na qualidade da solução, convergência, desempenho, evolução do \emph{fitness} etc. Uma versão paralelizada em CUDA foi apresentada em evento de computação de alto desempenho \cite{ERAD12} e é discutida na seção \ref{sec:oneMaxNaGPU}.
	
	\begin{figure}[htbp]
		\centering
			\includegraphics[width=0.60\textwidth]{figs/resultados/onemax/onemax_objetivo.png}
		\caption{ONEMAX: representação cromossomial, \emph{fitness} e objetivo.}
		\label{fig:onemax_objetivo_metodologia}
	\end{figure}
	
	Tendo a estrutura básica do ONEMAX, o próximo passo foi tentar fazer um modelo. Abordei o Problema das Oito Rainhas, que consiste em posicionar oito rainhas em um tabuleiro de xadrez de modo que não se ataquem. Sem nenhuma referência externa, propus um modelo de GA que conseguiu chegar em algumas soluções \cite{qualificacao_adriano}. Uma delas está na figura \ref{fig:OitoRainhasSolucao}.
	
	\begin{figure}[htbp]
		\centering
			\includegraphics[width=0.30\textwidth]{figs/materiais_metodo/ga/OitoRainhasSolucao.png}
		\caption{Uma solução para o Problema das 8 Rainhas.}
		\label{fig:OitoRainhasSolucao}
	\end{figure}
	
	Tratando o tabuleiro de xadrez como um plano cartesiano, a representação cromossomial era um \emph{string} com os oito pares de coordenadas $(x,y)$. Cada coordenada gerava uma matriz característica de 0's e 1's que, quando somadas, resultava em uma matriz com a informação da quantidade $c$ de ataques que cada rainha sofreu. O objetivo, portanto, era encontrar um indivíduo que levasse a $c = 0$. A função de avaliação foi definida como $f = 1/(1 + c)$.
	
	Criar um modelo para o Oito Rainhas foi importante para o entendimento do elo entre um GA e o problema a ser resolvido. Essa ligação encontra-se na representação cromossomial e na função de avaliação. Defini ambas de maneira adequada, mas a representação cromossomial apresentou um problema: nada impede de haver pontos $(x,y)$ repetidos no cromossomo. Em outras palavras, ela permite que duas rainhas sejam colocadas na mesma posição do tabulareiro, o que é proibido no xadrez. 
	
	Por fim, utilizei GA na criação de um robô, chamado \emph{Genético}, para ser testado contra os campeões do torneio de Robocode da Faculdade de Tecnologia da Unicamp\footnote{\texttt{http://torneiorobocode.orgfree.com/torneio-ft.php}}. Robocode é um jogo\footnote{\texttt{http://robocode.sourceforge.net/}}, cujo objetivo é programar um tanque de guerra robô para competir contra outros robôs em uma arena de batalha. Ele começou como um projeto pessoal no ano 2000 e depois foi incorporado pela IBM\footnote{\texttt{http://robocode.sourceforge.net/docs/ReadMe.html}}. Atualmente é um projeto de código aberto.
	
	\begin{figure}[htbp]
		\centering
			\includegraphics[width=0.50\textwidth]{figs/materiais_metodo/ga/Robocode_Battle_Field.PNG}
		\caption{Arena de batalha do Robocode.}
		\label{fig:Robocode}
	\end{figure}
	
	Me baseei no artigo \cite{robocodeGA}, publicado em um congresso de computação evolutiva da IEEE. As primeiras versões do \emph{Genético} não travaram boas batalhas. Alterei o \emph{fitness} e o processo de treinamento. Na versão final o ele foi capaz de vencer os robôs que ficaram em primeiro e terceiro lugar no torneio daquele ano (o segundo lugar não estava disponível para \emph{download}). O \emph{Genético} vencia, inclusive, contra os dois simultaneamente \cite{robocodeGA_adriano}.
	

%========================================================
\section{\emph{Software}}
%========================================================

	O \emph{software} utilizado foi totalmente desenvolvido por mim, e há razões metodológicas que justificam essa escolha. Obtive bons resultados criando os próprios programas nos estudos iniciais de GA. Eles foram fundamentais para que eu soubesse exatamente o que estava acontecendo durante a execução. Quis ter esse controle total também sobre o programa que executaria o GA proposto para essa dissertação. Ele foi escrito em Linguagem C, utilizando apenas quatro bibliotecas padrão: \texttt{stdio.h}, \texttt{stdlib.h}, \texttt{time.h} e \texttt{math.h}. Portanto, é totalmente portável para os sistemas operacionais que possuem um compilador C$++$. Além disso, C é a linguagem nativa da arquitetura CUDA, o que facilitará sua paralelização.
	
	Tive muito cuidado com a confiabilidade dos resultados. Quando opta-se por não gerá-la automaticamente, a \emph{semente} dos números pseudo-aleatórios é um dos parâmetros de entrada. Duas execuções com os mesmos parâmetros, incluindo a semente, levam a exatamente os mesmos resultados. Por isso nas tabelas e gráficos deixei explícito seu valor. Os gráficos encontrados em \cite{metodo2004} e \cite{metodo2011} referem-se ao maior \emph{fitness} (e $\rho$ associado) de cada geração. O programa gera essa informação, mas também exibe as médias dessas variáveis. De acordo com \cite{Mitchell98}, uma teoria geral para entender e prever o comportamento dos GAs seria análoga à Mecânica Estatística na Física. Ao contrário de lidar com grande número de componentes do sistema, como a composição genética exata de cada população, tal abordagem trabalha com uma estatística mais ``macroscópica'', como o \emph{fitness} médio da população. Portanto, tanto os critérios de parada do programa quanto a minha análise foram baseadas em médias.
	
	Para utilizar o \emph{software} basta baixar o código e compilar o arquivo \texttt{Serial\_novo.c} (será necessário alterar o diretório no \emph{include} das bibliotecas que desenvolvi). O código está disponível na internet para qualquer um utilizar, testar e, inclusive melhorar\footnote{\texttt{https://github.com/prietoab/msc\_code}}. Caso isso aconteça, peço apenas que cite essa dissertação.
	
		A execução se dá via linha de comando passando os parâmetros necessários. Porém, como são muitos parâmetros, é aconselhável a utilização de um arquivo de \emph{script}. Assim é possível criar processos de varredura para variar os parâmetros desejados ou repetir várias execuções. Fiz isso para automatizar o estudo e ter dados suficientes para análise. Na figura \ref{fig:script_windows} há um exemplo de \emph{script} Windows para fazer execuções variando o número de genes.
			
		\begin{figure}[htbp]
			\centering
				\includegraphics{figs/materiais_metodo/software/script_windows.png}
			\caption{Exemplo de \emph{script} Windows para fazer execuções variando o número de genes.}
			\label{fig:script_windows}
		\end{figure}

	Não é necessário ter uma matriz para execução. Através do parâmetro \emph{Tamanho do cromossomo} (seção \ref{sec:listaParametros}) o programa gera automaticamente uma matriz de Coope \cite{Coope1977}, definida na equação \ref{eq:MatrizCoope}. Essa matriz foi utilizada nos testes de \cite{metodo2011}.
	
	\begin{equation}\label{eq:MatrizCoope}
		\begin{array}{ccl}
			H(i,i) = 2i - 1 			& , & i = 1, 2, ..., n. \\
														&		&		\\
			H(i,j) = H(j,i) = 1		& , & i \neq j; \\
														&		& i = 1, 2, ..., n; \\
														&		& j = 1, 2, ..., n.
		\end{array}
	\end{equation}
	
%---------------------------------------------------
\subsection{Informações de saída}
%---------------------------------------------------	
		
Há cinco grupos de informações na saída do programa. Usados em conjunto dão um panorama geral do algoritmo genético. Para apenas um deles (Estatística) um arquivo texto é gerado automaticamente. Os outros são exibidos na tela. Para alterar a saída padrão para um arquivo texto, é necessário utilizar o caractere de redirecionamento da saída padrão. No Windows, Linux e Unix esse caractere é o ``>''. No Linux, por exemplo, o comando ``\texttt{ls > dirs.txt}'' lista o conteúdo do diretório atual e armazena no arquivo \texttt{dirs.txt}.

\begin{enumerate}

	\item \textbf{Cabeçalho}
	
		Contém todos os parâmetros de execução recebidos na linha de comando. Impresso na tela.
	
	\item \textbf{Comportamento do \emph{fitness}}
	
		Imprime na tela, para cada geração, além de alguns parâmetros de execução, as seguintes informações: 
		
		\begin{itemize}
			\item $\rho$ mínimo
			\item $\rho$ médio (<$\rho$>)
			\item \emph{fitness} médio (<\emph{fitness}>)
			\item Maior \emph{fitness}
			\item $\rho$ associado ao maior \emph{fitness}
			\item <$|\nabla \rho|^2$>
			\item Posição do melhor indivíduo
		\end{itemize}

	\item \textbf{Tempos de processamento}
	
		Imprime na tela uma estimativa para o tempo de processamento (em \emph{clocks}) para cada operador. Se o programa foi executado por 200 gerações, haverá 200 tempos de processamento para o \emph{fitness}, seleção, \emph{crossover} e mutação.
		
	\item \textbf{Geração final}
	
				Impressão de todos os indivíduos da última geração. 
	
	\item \textbf{Estatística}
	
				A cada execução é criado (ou atualizado caso já exista) um arquivo chamado \texttt{estatistica.txt}. Nele, além de todos os parâmetros, há informações relacionadas à geração que atingiu algum critério de parada. Por exemplo, além do próprio número da última geração, há o \emph{fitness} médio, o maior \emph{fitness}, o $\rho$ associado ao maior \emph{fitness}, o $\rho$ médio e o tempo total de processamento do programa.
	
\end{enumerate}
	
%---------------------------------------------------
\subsection{Lista dos arquivos fonte}
%---------------------------------------------------
	
	O programa é pequeno ($\approx$ 2500 linhas), e está distribuído em seis arquivos, um principal e cinco bibliotecas:
	
	\begin{itemize}
		\item \textbf{Serial\_novo.c}
		
		Arquivo principal. Contém o fluxo do GA (figura \ref{fig:fluxo}).
		
		\item \textbf{Estruturas.h}
		
		Contém as estruturas de dados e uma função que retorna automaticamente o parâmetro $\lambda$ do \emph{fitness} (seção \ref{sec:eq_lambda}).
		
		\item \textbf{Algebra\_Linear\_serial.h}
		
		Mutiplicação e subtração de matrizes.
		
		\item \textbf{Estatistica.h}
		
		Média, variância e desvios do Quociente de Rayleigh.
		
		\item \textbf{Auxiliares\_serial.h}
		
		Números pseudo-aleatórios, alocação de memória para indivíduos na população.
		
		\item \textbf{GA\_Serial.h}
		
		Contém as funções do Algoritmo Genético. A maoria do código está nessa biblioteca.	
		
	\end{itemize}
					
%---------------------------------------------------
\subsection{Lista dos parâmetros de execução}
\label{sec:listaParametros}
%---------------------------------------------------
	\begin{enumerate}
		\item \textbf{Código da máquina}.
		
				Número inteiro, utilizado para identificar o computador que o programa foi executado. Útil para comparar as execuções em computadores diferentes. Por exemplo, se há quatro computadores A, B, C e D, é possível classificá-los como A = 0, B = 1, C = 2, D = 3.
				
		\item \textbf{Serial ou Paralelo}?
		
				Número inteiro. Determina se a execução será serial (= 0) ou paralela (= 1). A atual versão permite apenas execução serial. Portanto, esse parâmetro pode ser fixado em zero.
				
		\item \textbf{Tamanho do cromossomo (ordem da matriz de Coope)}.
		
				Número inteiro. Determina a ordem da matriz de Coope. Inserir 200 nesse parâmetro gera uma matriz de Coope de tamanho 200 x 200.
			
		\item \textbf{Quantidade máxima de gerações}.
		
				Número inteiro. Um dos critérios de parada. Para evitar que o programa entre em um \emph{loop} infinito caso não haja convergência para uma solução. Se definida como 100, o programa executará, no máximo, 100 iterações de Avaliação, Seleção, \emph{Crossover} e Mutação.
		
		\item \textbf{Quantidade de indivíduos na população}.
		
		Número inteiro. Determina a quantidade de indivíduos em cada população.
		
		\item \textbf{Tipo do Fitness}.
		
		Número inteiro. O programa pode trabalhar com cinco tipos de \emph{fitness}. Dois deles são os apresentados na seção \ref{sec:fitness_metodo}. Qualquer código diferente dos apresentados abaixo faz com que o \emph{fitness} seja definido como $f = -1$.
		
			\begin{itemize}
				\item Tipo  0:
				
					\begin{equation}
					f = e^{-\lambda(\rho - E_L)^2}
					\end{equation}
				
				\item Tipo 1:
				
					\begin{equation}
					f = e^{-\lambda |\nabla \rho|^2}
					\end{equation}
					
				\item Tipo 2:
				
					\begin{equation}
					f = e^{-\lambda [(\rho - E_L)^2 + |\nabla \rho|^2]}
					\end{equation}
					
				\item Tipo 3:
				
					\begin{equation}
					f = e^{-\lambda |\nabla \rho|}
					\end{equation}
		
			\item Tipo 4:
				
					\begin{equation}
					f = e^{-\lambda [(\rho - E_L)^2 + |\nabla \rho|]}
					\end{equation}
			\end{itemize}
			
		\item \textbf{Tipo do Fitness Paralelo}.
		
				Número inteiro. A atual versão permite apenas execução serial. Portanto, esse parâmetro pode ser fixado em zero.
		
		\item \textbf{Tamanho do Torneio}
		
			Número inteiro. Define a quantidade de indivíduos selecionados para o torneio na Seleção.
		
		\item \textbf{Probabilidade do \emph{Crossover}}.
		
			Número real. Define, em porcentagem, a probabilidade de \emph{Crossover}. Exemplo: 80.5 = 80.5\%.
		
		\item \textbf{Quantidade de Pontos de Corte}.
		
			Número inteiro. Na versão atual a quantidade de pontos de corte foi fixada em dois conforme o operador de \emph{crossover} definido na seção \ref{sec:crossover_utilizado}. Pode ser mantido como zero.
		
		\item \textbf{Probabilidade de Mutação}.
		
		Número real. Define, em porcentagem, a probabilidade de mutação. Exemplo: 12.7 = 12.7\%.
		
		\item \textbf{Intensidade da Mutação - $\Delta$}.
		
			Número real. Define o parâmetro $\Delta$ utilizado no \emph{crossover}. É dividido por dez. Exemplo: $1.2$ no parâmetro $\rightarrow 0.12$ no $\Delta$.
		
		\item \textbf{Valor para $\lambda$}.
		
		Número real. Define o valor do parâmetro $\lambda$ do \emph{fitness}. Se for configurado como $-1$, um valor adequado para $\lambda$ é gerado automaticamente.
		
		\item \textbf{Valor para $E_L$}.
		
		Número real. Define um limite inferior para o autovalor mínimo ($E_0$) nos \emph{fitness} de tipo 0, 2 e 4. 
		
		\item \textbf{Precisão - $\xi$}.
		
		Número real. Define a precisão dos critérios de parada. O programa termina se a variável alvo é menor ou igual a $\xi$. Depende do tipo de \emph{fitness}.
		
		Condições de parada em função do \emph{fitness} ($<x>$ significa valor médio de $x$):
				
		\begin{itemize}
			\item Fitness tipos 0, 2 e 4:
			
			\begin{equation}
				\begin{array}{ccll}
				<|\nabla \rho_i|^2> & \leq & \xi & \mbox{  ou} \\
				| <\rho_i> - E_L | & \leq & \xi &
				\end{array}
			\end{equation}
			
			
			\item Fitness tipos 1 e 3:
			
			\begin{equation}
				\begin{array}{ccll}
					<|\nabla \rho_i|^2> & \leq & \xi &
				\end{array}
			\end{equation}
			
		\end{itemize}
		
		\item \textbf{Imprime comportamento do \emph{Fitness}}?
		
		Se configurado como Verdadeiro (1), imprime na saída padrão as variáveis de interesse para estudar o comportamento do \emph{fitness}. 
		
			0: \emph{Falso}. Não imprime.
			
			1: \emph{Verdadeiro}. Imprime.
		
		\item \textbf{Imprime tempos de execução}?
		
			Número inteiro. Se configurado como Verdadeiro (1), imprime na saída padrão os tempos estimados de execução (em \emph{clocks} do processador) das funções e operadores.
			
			
			0: \emph{Falso}. Não imprime.
			
			1: \emph{Verdadeiro}. Imprime.
			
		
		\item \textbf{Gera nova semente para números pseudo-aleatrórios}?
		
		Número inteiro. Se configurado como Verdadeiro (1), o programa cria uma nova semente para os números pseudo-aleatórios. Caso contrário, a semente definida no próximo parâmetro é utilizada.
		
		0: \emph{Falso}. Utiliza semente definida no parâmetro \emph{Semente}.
		
		1: \emph{Verdadeiro}. Cria uma nova semente. O parâmetro \emph{Semente} é ignorado.
		
		\item \textbf{Semente dos números pseudo-aleatrórios}.
		
		Número inteiro. Define a semente dos números pseudo-aleatórios. Depende do parâmetro anterior.
		
		\item \textbf{Tipo do cálculo de $\nabla \rho$}.
		
		Deve ser configurado como 1.
		
		Nas versões iniciais $\nabla \rho$ era calculado literalmente como na equação \ref{eq:grad_rho_metodo}, exigindo o uso de uma matriz identidade ($I$) da ordem do Hamiltoniano. A equação foi reescrita internamente de modo que $I$ não fosse necessária, liberando memória e fazendo menos operações. Usar ou não $I$ leva aos mesmos resultados.
		
		0: Utiliza matriz identidade.
		
		1: Não utiliza matriz identidade (libera memória e faz menos operações).
		
	\end{enumerate}
\chapter{Resultados e discussão}
\label{cap:resultados}

\begin{enumerate}
	
	\item Parágrafo de introdução do capítulo. Citar que, basicmente, o leitor encontrará no capítulo:
		\begin{enumerate}
			\item Resultados do ONEMAX, legitimando o uso do código para o programa mais complexo que foi utilizado no método dos indianos.
			\item o estudo dos tipos de \textit{fitness}, operador responsável pelo elo entre o algoritmo e o problema \cite{Linden2008}, que, para o nosso caso, é encontrar autovalores. Ponte para o próximo: para cada tipo de \textit{fitness}, um resultado diferente.
		\end{enumerate}

	\item Os dois tipos de fitness dos indianos. Ideia central: dois tipos, resultados diferentes. Com $\nabla \rho$ chegamos a um autovalor qualquer, com $(\rho - \rho_0)^2$ podemos chegar ao mínimo, mas dá mais trabalho. Ponte para o próximo: proposta de dois novos fitness.
	
	\item Combinação de $\nabla \rho$ com $(\rho - \rho_0)^2$. Se cada forma leva a comportamentos diferentes, tentamos combinar os dois termos em um único fitness. Uma hipótese seria a melhoria da qualidade dos resultados. A hipótese não foi confirmada. Ponte para o próximo: a busca pela qualidade levou à verificação da importância do parâmetro $\lambda$.
	
	\item Além do que os indianos disseram, que $\lambda$ é escolhido para não estourar a função exponencial, ele tem influência na convergência do algoritmo e na precisão (ou resolução) do fitness. Se na primeira população, geração inicial, o fitness médio é alto, isso provoca convergência precoce, fazendo com que o resultado final seja ruim. Por outro lado, se no início o fitness médio é muito baixo, não há muita discriminação entre os indivíduos, o fitness não cresce e não chegamos a uma solução. A medida que o fitness se aproxima de 1, a discriminação entre os indivíduos fica difícil, levando ao problema da resolução. Ponte para o próximo: vários testes levaram ao desenvolvimento de uma equação empírica para $\lambda$, restrita às matrizes de Coope$-$Sabo \cite{Coope1977}.
	
	\item Fórmula empírica. Por já conhecermos de antemão os autovalores das matrizes de Coope$-$Sabo, foi possível criar uma fórmula empírica para $\lambda$. Ela garante que na primeira população o \textsl{fitness} médio é baixo, previnindo o \textsl{underflow} do \textit{fitness} e a convergência prematura.
	
	\end{enumerate}
	
\section{Problemas com o mínimo global}	
	
	No capítulo \ref{sec:metodo} mostrei que o \textit{fitness} utilizado no artigo \cite{metodo2004}  foi
	
	\begin{equation}
		\label{eq:fitnessGrad2}
		f_i = e^{-\lambda |\nabla \rho_i|^2},
	\end{equation}
	onde $f_i$ é o \textit{fitness} do $i$-ésimo indivíduo da população, $\lambda$ é um parâmetro para evitar o estouro do \textit{fitness} e $| \nabla \rho_i|^2$ é o módulo ao quadrado do vetor gradiente de $\rho$, dado por
		
				\begin{equation}
					\nabla \rho_i = \frac{2[H - \rho_i]C_i}{C_i^t C_i},
				\end{equation}
	em que $C_i$ é um vetor candidato à solução do problema do autovalor
	
	\begin{equation}
		HC = EC.
	\end{equation}
	
	Além disso, se $C_i$ é de fato um dos autovetores, $\rho$ é o autovalor associado $E_i$:
	
	\begin{equation}\label{eq:rho_eh_E}
		\rho_i = \frac{C_i^t H C_i}{C_i^t C_i} = E_i.
	\end{equation}
	
	A fim de reproduzir os resultados, testei o método com matrizes de Coope$-$Sabo (equação \ref{eq:MatrizCoope}) de ordem 10, 20, 30 e 40, utilizando os mesmos parâmetros encontrados em \cite{metodo2004}: probabilidade de \textit{crossover} $p_c = 75\%$, probabilidade de mutação $p_m = 50\%$ e intensidade de mutação $\Delta = 0,01$. Com um bom ajuste de $\lambda$, que será discutido em detalhes posteriormente, o \textit{fitness} comportou-se conforme o esperado em todos os casos. Um exemplo está na figura \ref{fig:compFitnessTipo1N10}, que apresenta o melhor fitness de cada geração para uma matriz de ordem N = 10. Na primeira geração o melhor \textit{fitness} é pequeno, aproximadamente 0,1, cresce rapidamente e a partir da décima geração está próximo de 1.
	
	\begin{figure}[htbp]
		\centering
			\includegraphics{figs/resultados/fitnessGrad/N10_00_fitness.pdf}
			\caption{Comportamento do \textsl{fitness} $f_i = e^{-\lambda \| \nabla \rho_i \|^2}$ para N = 10. Na primeira geração o melhor \textit{fitness} é pequeno, aproximadamente 0,1, cresce rapidamente e a partir da décima geração está próximo de 1.}
		\label{fig:compFitnessTipo1N10}
	\end{figure}
	
	O próximo passo foi verificar o comportamento de $\rho$, o Quociente de Rayleigh, e, especificamente, sua convergência para o menor autovalor $E_0$. Ainda conforme \cite{metodo2004}, obteríamos uma curva semelhante à da figura \ref{fig:compFitnessTipo1N10}, mas invertida, ou seja, os primeiros valores de $\rho$ seriam grandes e, rapidamente, diminuiriam até haver convergência para o autovalor mínimo. Na figura \ref{fig:rho_N10} há um exemplo.	Os gráficos exibem os valores de $\rho$ para a mesma execução apresentada na figura \ref{fig:compFitnessTipo1N10}. Note no primeiro gráfico que até a geração 20 o quociente $\rho$ teve caráter oscilatório e, então, aparentemente estabilizou-se entre 6 e 8, valores muito superiores ao autovalor mínimo para essa matriz, $E_0 = 0,38675$. Entretanto, ainda no primeiro gráfico, observa-se que há uma tendência de queda do $\rho$ entre as gerações 40 e 50 e, portanto, existiria a possibilidade do algoritmo convergir para $E_0$. Porém, para esse exemplo especificamente, isso não aconteceu, como pode ser visto no segundo gráfico da figura \ref{fig:rho_N10}. Para garantir a estabilidade, o programa foi executado até a geração 400.000, e o valor médio obtido foi $<\rho> = 6,572898$. Para minha surpresa, além do valor obtido de $<\rho>$ não ser o mínimo, ele não é um valor qualquer, mas corresponde, com erro menor que $0,00002\%$, ao quarto autovalor da matriz, $E_3 = 6,572897$. Um gráfico expandido dessa execução está na figura \ref{fig:rho_N10_completa} da página \pageref{fig:rho_N10_completa}. Imaginei, então, que poderia haver algo de errado com o programa. 
	
	\begin{figure}[htbp]
		\centering
			\includegraphics[width=0.40\textwidth]{figs/resultados/rho_N10_g50.pdf}
			\includegraphics[width=0.40\textwidth]{figs/resultados/rho_N10_g400000.pdf}
		\caption{Comportamento de $\rho$ (Quociente de Rayleigh) para uma matriz de Coope$-$Sabo de ordem 10.}
		\label{fig:rho_N10}
	\end{figure}
	
	\newpage
	\begin{landscape}
	\begin{figure}[p]
		\centering
			\includegraphics[width=1.3\textwidth]{figs/resultados/rho_N10.png}
		\caption{Comportamento de $\rho$ (Quociente de Rayleigh) para uma matriz de Coope$-$Sabo de ordem 10.}
		\label{fig:rho_N10_completa}
	\end{figure}
	\end{landscape}
	\newpage
			
	Após esses resultados preliminares executei uma validação cuidadosa do programa, testando cada uma de suas quase 2500 linhas e comparando os resultados das operações e cálculos com o Microsoft Excel e SciLab. A hipótese era a de que erros numéricos, principalmente nas funções de álgebra linear e nos operadores genéticos, pudessem ter levado ao comportamento incorreto da não convergência para o menor autovalor. Nenhum erro significativo foi encontrado.
	
	Os testes com a versão corrigida do programa estão nas figuras \ref{fig:execucoes_N10}, \ref{fig:execucoes_N20}, \ref{fig:execucoes_N30} e \ref{fig:execucoes_N40}. Visando brevidade, apresentarei dados para matrizes de Coope$-$Sabo de ordem 10, 20, 30 e 40 apenas, sem perda de generalidade. Foram cinco execuções para cada matriz, até a geração 400.000, gerando sempre dois gráficos, um do \textit{fitness} médio (<\textit{fitness}>) e outro do Quociente de Rayleigh médio (<$\rho$>), ambos em função do número de gerações, e dando ênfase às primeiras 100 gerações. Essas escolhas, número máximo da geração e uso de médias sobre cada população, visaram garantir, respectivamente, a convergência genética e boa precisão. A exibição de apenas as primeiras 100 gerações tem como objetivo olhar em detalhe (com \textit{zoom}) o período em que o \textit{crossover} tem mais peso, ou seja, onde há geralmente os saltos no espaço de soluções de um Algoritmo Genético. Em todos os gráficos de $<\rho>$ há indicado nas legendas o autovalor mínimo $E_0$ e o autovalor obtido após as 400.000 gerações ($E_{obtido}$). Na tabela \ref{tab:autovalores10a40} há a lista de todos os autovalores. Por exemplo, para uma matriz de ordem $N = 10$, o menor autovalor é $E_0 = $0,386075, e o quinto autovalor para $N = 30$ é $E_4$ = 8,450274.

\begin{table}[htb]
	\caption{Lista de autovalores para matrizes de Coope$-$Sabo de ordem 10, 20, 30 e 40.}
	\label{tab:autovalores10a40}
% Table generated by Excel2LaTeX from sheet 'Todos'
\begin{center}
\begin{tabular}{r|r|r|r|r}
	\hline \hline
	\textbf{\#} &   \textbf{10} &   \textbf{20} &   \textbf{30} &   \textbf{40} \\
	\hline \hline
					 0 &   0,386075 &   0,341237 &   0,319737 &   0,306086 \\
	\hline
					 1 &   2,461056 &   2,397247 &    2,36844 &   2,350583 \\
	\hline
					 2 &   4,518931 &   4,436173 &   4,401134 &   4,379909 \\
	\hline
					 3 &   6,572897 &   6,468521 &   6,427419 &     6,4031 \\
	\hline
					 4 &   8,628524 &   8,497626 &   8,450274 &    8,42294 \\
	\hline
					 5 &   10,69057 &   10,52507 &   10,47105 &   10,44068 \\
	\hline
					 6 &   12,76574 &   12,55178 &    12,4905 &     12,457 \\
	\hline
					 7 &   14,86753 &   14,57845 &   14,50908 &   14,47232 \\
	\hline
					 8 &   17,03654 &   16,60562 &   16,52713 &   16,48692 \\
	\hline
					9 &   22,07215 &   18,63385 &   18,54488 &     18,501 \\
	\hline
					10 &            &    20,6637 &   20,56255 &    20,5147 \\
	\hline
					11 &            &   22,69588 &    22,5803 &   22,52816 \\
	\hline
					12 &            &   24,73127 &   24,59828 &   24,54146 \\
	\hline
					13 &            &   26,77114 &   26,61667 &   26,55469 \\
	\hline
					14 &            &   28,81733 &    28,6356 &   28,56792 \\
	\hline
					15 &            &   30,87288 &   30,65527 &   30,58122 \\
	\hline
					16 &            &   32,94325 &   32,67586 &   32,59466 \\
	\hline
					17 &            &   35,04014 &    34,6976 &   34,60831 \\
	\hline
					18 &            &   37,19805 &   36,72077 &   36,62223 \\
	\hline
					19 &            &    45,2308 &   38,74571 &   38,63648 \\
	\hline
					20 &            &            &   40,77285 &   40,65114 \\
	\hline
					21 &            &            &   42,80277 &    42,6663 \\
	\hline
					22 &            &            &   44,83625 &   44,68204 \\
	\hline
					23 &            &            &   46,87444 &   46,69846 \\
	\hline
					24 &            &            &   48,91902 &   48,71568 \\
	\hline
					25 &            &            &   50,97274 &   50,73385 \\
	\hline
					26 &            &            &   53,04052 &   52,75311 \\
	\hline
					27 &            &            &   55,13271 &   54,77369 \\
	\hline
					28 &            &            &   57,27946 &   56,79581 \\
	\hline
					29 &            &            &   68,37101 &   58,81981 \\
	\hline
					30 &            &            &            &   60,84608 \\
	\hline
					31 &            &            &            &   62,87517 \\
	\hline
					32 &            &            &            &   64,90781 \\
	\hline
					33 &            &            &            &   66,94504 \\
	\hline
					34 &            &            &            &   68,98845 \\
	\hline
					35 &            &            &            &   71,04053 \\
	\hline
					36 &            &            &            &   73,10578 \\
	\hline
					37 &            &            &            &   75,19353 \\
	\hline
					38 &            &            &            &   77,33102 \\
	\hline
					39 &            &            &            &   91,50634 \\
	\hline \hline
	\end{tabular}
	\end{center}  
\end{table}
	
	Há características encontradas em todas as execuções. Em qualquer gráfico do \textit{fitness} observa-se estabilidade do comportamento conforme esperado pelo método: no início seu valor é baixo, próximo de zero, cresce rapidamente nas primeiras gerações e fica estável próximo de $<fitness> = 1$. Com relação ao $\rho$, há sempre oscilações, sejam pequenas variações em torno de uma clara linha de tendência, como na execução 02 para N = 10, ou grandes saltos, como nas execuções 05 de N = 20 e 05 de N = 30. Novamente, o menor autovalor não foi obtido em nenhuma execução, contradizendo os resultados de \cite{metodo2004}, mas, por outro lado, o algoritmo sempre encontrou algum autovalor.
	
	De fato, verificando os dados da tabela \ref{tab:execucoes10a40}, concluí que tais valores não devem ser coincidência. Para todas as execuções o \textit{fitness} médio chegou ao valor máximo ($<f> = 1,000000$). As médias de $\rho$ sobre todos os indivíduos da última população possuem baixo desvio padrão ($\sigma$ < 0,0001), indicando que eles são muito parecidos entre si e que o algoritmo convergiu. Ou seja, não há variabilidade genética suficiente na população para alterar o rumo da busca de modo a atingir o menor autovalor, ou o mínimo global. Portanto, o algoritmo chegou em um mínimo local, corroborado pelos baixos erros relativos de $<\rho>$ quando comparado com o autovalor mais próximo. Por exemplo, para N = 30, execução 4,  $<\rho>$ = 40,772447, correspondendo, com erro relativo absoluno menor que $0,001\%$, ao vigésimo primeiro autovalor, $E_{20} = 40,772850$. Apesar das evidências descritas acima, até esse ponto ainda há dúvidas sobre a validade do programa e, obviamente, dos resultados produzidos. Então, busquei embasamento mais rigoroso.

De acordo com \cite{metodo2004}, se algum $C_i$, em algum momento, é o autovetor fundamental (associado ao menor autovalor), o $\nabla \rho$ é nulo. Com o \textit{fitness} da equação \eqref{eq:fitnessGrad2} os autores afirmam que ``\textit{Claramente, $f_i \rightarrow 1$ quando $\nabla \rho_i \rightarrow 0$, sinalizando que a evolução atingiu o verdadeiro autovetor fundamental de $H$ em $C_i$}''\footnote{Tradução livre de ``\textit{Clearly, $f_i \rightarrow 1$, as $\nabla \rho_i \rightarrow 0$, signalling that the evolution has hit the true ground state eigenvector of $H$ in the vector $C_i$}''.}. Há duas relações distintas de causalidade nessa frase, e acredito que nelas residam a explicação dos resultados obtidos até agora.

A primeira relação de causalidade refere-se à afirmação ``\textit{$f_i \rightarrow 1$ quando $\nabla \rho_i \rightarrow 0$}'', que está absolutamente correta. Retomando a seção \ref{sec:fitness_metodo}, o \textit{fitness} definido pela equação \ref{eq:fitnessGrad2} é limitado no intervalo (0,1] e, como $\lambda > 0$, só chega ao seu valor máximo quando $\nabla \rho_i = 0$. Em outras palavras, $\nabla \rho_i \rightarrow 0$ implica $f_i \rightarrow 1$.

Na afirmação ``(...) \textit{sinalizando que a evolução atingiu o verdadeiro autovetor fundamental de $H$ em $C_i$}'' reside a segunda relação de causalidade que, apesar de sutil, é muito poderosa:

\begin{equation}\label{eq:afirmacaoErrada}
	\mbox{Se } f_i \rightarrow 1\mbox{, } C_i = C_0.
\end{equation}

Ou seja, sempre que algum indivíduo $C_i$, de qualquer população, possuir \textit{fitness} muito próximo de 1, isso implica que, além de ter uma excelente ``nota'', ele, ainda por cima, é um vetor especial, o autovetor fundamental $C_0$. Portanto, possui autovalor associado $E_0$, o autovalor mínimo (conforme equação \ref{eq:rho_eh_E}). Grosso modo, $f_i(C_i) = 1$ implica que $C_i = C_0$ e que podemos obter $E_0(C_0)$:

\begin{equation}\label{eq:causalidadeErrada}
	f_i(C_i) = 1 \rightarrow C_i = C_0 \rightarrow E_0(C_0).
\end{equation}

As relações de causa e efeito da equação acima estão erradas. Em sua obra clássica sobre o problema de autovalores em matrizes simétricas, \cite{Parlett1998} abre o capítulo introdutório frisando que ``\textit{em muitos lugares no livro, é feita referência a fatos mais ou menos bem conhecidos sobre a teoria de matrizes}''\footnote{Tradução livre de ``\textit{At many places in the book, reference is made to more or less well known facts from matrix theory}''.}. Conforme já dito no capítulo \ref{cap:algebra}, um desses fatos diz que $\rho(\mbox{\textit{u}})$ é estacionário, ou seja, $\nabla \rho(\mbox{\textit{u}}) = 0$, apenas se o vetor \textit{u} é um autovetor $w$ de $HC = EC$. Consequentemente, o encadeamento correto se apresenta como:

\begin{equation}\label{eq:causalidadeCorreta}
	C_i \mbox{ é um autovetor} \rightarrow \nabla \rho(C_i) = 0 \rightarrow f_i = 1.
\end{equation}
	
Então, se $f_i = 1$, o máximo que podemos concluir é que $C_i$ é \textit{algum} autovetor, e não necessariamente \textit{o} autovetor fundamental.

Acredito que o programa não contém erros. Ao final de todos os testes o \textit{fitness} médio foi <\textit{f}> = 1, a população final era composta por autovetores e foi possível, com boa precisão, obter os autovalores relacionados (não necessariamente o autovalor mínimo). Os dados, portanto, confirmaram a matemática.

Apesar de não chegar ao mínimo, o método pode ser utilizado de maneira exploratória com relativa facilidade, bastando extrair $\rho$ sempre que $f_i \rightarrow 1$ e $\nabla \rho \rightarrow 0$. 

Resta a dúvida: afinal, como o autovalor mínimo foi obtido com o \textit{fitness} definido pela equação \ref{eq:fitnessGrad2}? Não sei. Esse \textit{fitness} foi utilizado não só em \cite{metodo2004}, mas também em \cite{metodo2006}, \cite{metodo2008} e \cite{metodo2009}, seguindo exatamente o argumento resumido pela equação \ref{eq:causalidadeErrada}. Não identifiquei nada nesses quatro artigos que pudesse levar à resposta. Segui o estudo com uma nova definição do \textit{fitness} encontrada em \cite{metodo2011}.

\begin{landscape}
\begin{center}
\begin{table}[htbp]
\caption{Execuções para matrizes de Coope$-$Sabo.}
\label{tab:execucoes10a40}
% Table generated by Excel2LaTeX from sheet 'Plan1 (2)'
\begin{tabular}{cccccccccc}
\hline \hline
   \textbf{N} & \textbf{Execução} & \textbf{Semente} & \textbf{$\lambda$} & \textbf{<\textit{Fitness}>} & \textbf{<$\rho$>} & \textbf{$\sigma$} & \textbf{\# autovalor} & \textbf{Autovalor} & \textbf{Erro relativo} \\
\hline \hline
        10 &          0 & 1445738835 &   0,128788 &   1,000000 &   2,461122 &   0,000023 &          1 &   2,461056 &    0,003\% \\
\hline
        10 &          1 & 1445780626 &   0,128788 &   1,000000 &   6,572898 &   0,000013 &          3 &   6,572897 &  0,00001\% \\
\hline
        10 &          2 & 1445780762 &   0,128788 &   1,000000 &   6,572883 &   0,000015 &          3 &   6,572897 &  -0,0002\% \\
\hline
        10 &          3 & 1445780907 &   0,128788 &   1,000000 &   6,572910 &   0,000016 &          3 &   6,572897 &   0,0002\% \\
\hline
        10 &          4 & 1445781049 &   0,128788 &   1,000000 &  12,765701 &   0,000016 &          6 &  12,765740 &  -0,0003\% \\
\hline
        10 &          5 & 1445781195 &   0,128788 &   1,000000 &   4,518952 &   0,000012 &          2 &   4,518931 &   0,0005\% \\
\hline
        20 &          1 & 1445795292 &   0,026665 &   1,000000 &   8,498192 &   0,000052 &          4 &   8,497626 &    0,007\% \\
\hline
        20 &          2 & 1445795501 &   0,026665 &   1,000000 &  12,551830 &   0,000018 &          6 &  12,551780 &   0,0004\% \\
\hline
        20 &          3 & 1445795718 &   0,026665 &   1,000000 &  12,551878 &   0,000020 &          6 &  12,551780 &   0,0008\% \\
\hline
        20 &          4 & 1445795953 &   0,026665 &   1,000000 &  14,578527 &   0,000035 &          7 &  14,578450 &   0,0005\% \\
\hline
        20 &          5 & 1445796166 &   0,026665 &   1,000000 &  18,634220 &   0,000062 &          9 &  18,633850 &    0,002\% \\
\hline
        30 &          1 & 1445796378 &   0,011171 &   1,000000 &  26,616790 &   0,000065 &         13 &  26,616670 &   0,0005\% \\
\hline
        30 &          2 & 1445796746 &   0,011171 &   1,000000 &  26,616595 &   0,000029 &         13 &  26,616670 &  -0,0003\% \\
\hline
        30 &          3 & 1445797109 &   0,011171 &   1,000000 &  22,580060 &   0,000051 &         11 &  22,580300 &   -0,001\% \\
\hline
        30 &          4 & 1445797473 &   0,011171 &   1,000000 &  40,772447 &   0,000071 &         20 &  40,772850 &   -0,001\% \\
\hline
        30 &          5 & 1445797882 &   0,011171 &   1,000000 &  30,655283 &   0,000022 &         15 &  30,655270 &  0,00004\% \\
\hline
        40 &          1 & 1445798248 &   0,006105 &   1,000000 &  26,554758 &   0,000040 &         13 &  26,554690 &   0,0003\% \\
\hline
        40 &          2 & 1445798838 &   0,006105 &   1,000000 &  54,773734 &   0,000078 &         27 &  54,773690 &  0,00008\% \\
\hline
        40 &          3 & 1445799429 &   0,006105 &   1,000000 &  58,819413 &   0,000087 &         29 &  58,819810 &  -0,0007\% \\
\hline
        40 &          4 & 1445800091 &   0,006105 &   1,000000 &  40,651473 &   0,000077 &         20 &  40,651140 &   0,0008\% \\
\hline
        40 &          5 & 1445800683 &   0,006105 &   1,000000 &  40,650764 &   0,000061 &         20 &  40,651140 &  -0,0009\% \\
\hline \hline
\end{tabular}
\end{table}  
\end{center}
\end{landscape}


\begin{figure}[htbp]
\centering
  \begin{tabular}{@{}cc@{}}
    \includegraphics[width=.45\textwidth]{figs/resultados/fitnessGrad/N10_01_fitness.pdf} &
    \includegraphics[width=.45\textwidth]{figs/resultados/fitnessGrad/N10_01_rho.pdf}   \\
		\includegraphics[width=.45\textwidth]{figs/resultados/fitnessGrad/N10_02_fitness.pdf} &
    \includegraphics[width=.45\textwidth]{figs/resultados/fitnessGrad/N10_02_rho.pdf}   \\
		\includegraphics[width=.45\textwidth]{figs/resultados/fitnessGrad/N10_03_fitness.pdf} &
    \includegraphics[width=.45\textwidth]{figs/resultados/fitnessGrad/N10_03_rho.pdf}   \\
		\includegraphics[width=.45\textwidth]{figs/resultados/fitnessGrad/N10_04_fitness.pdf} &
    \includegraphics[width=.45\textwidth]{figs/resultados/fitnessGrad/N10_04_rho.pdf}   \\
		\includegraphics[width=.45\textwidth]{figs/resultados/fitnessGrad/N10_05_fitness.pdf} &
    \includegraphics[width=.45\textwidth]{figs/resultados/fitnessGrad/N10_05_rho.pdf}
    %\multicolumn{2}{c}{\includegraphics[width=.23\textwidth]{example-image-a}}
  \end{tabular}
  \caption{Execuções N = 10.}
	\label{fig:execucoes_N10}
\end{figure}


\begin{figure}[htbp]
\centering
  \begin{tabular}{@{}cc@{}}
    \includegraphics[width=.45\textwidth]{figs/resultados/fitnessGrad/N20_01_fitness.pdf} &
    \includegraphics[width=.45\textwidth]{figs/resultados/fitnessGrad/N20_01_rho.pdf}   \\
		\includegraphics[width=.45\textwidth]{figs/resultados/fitnessGrad/N20_02_fitness.pdf} &
    \includegraphics[width=.45\textwidth]{figs/resultados/fitnessGrad/N20_02_rho.pdf}   \\
		\includegraphics[width=.45\textwidth]{figs/resultados/fitnessGrad/N20_03_fitness.pdf} &
    \includegraphics[width=.45\textwidth]{figs/resultados/fitnessGrad/N20_03_rho.pdf}   \\
		\includegraphics[width=.45\textwidth]{figs/resultados/fitnessGrad/N20_04_fitness.pdf} &
    \includegraphics[width=.45\textwidth]{figs/resultados/fitnessGrad/N20_04_rho.pdf}   \\
		\includegraphics[width=.45\textwidth]{figs/resultados/fitnessGrad/N20_05_fitness.pdf} &
    \includegraphics[width=.45\textwidth]{figs/resultados/fitnessGrad/N20_05_rho.pdf}
    %\multicolumn{2}{c}{\includegraphics[width=.23\textwidth]{example-image-a}}
  \end{tabular}
  \caption{Execuções N = 20.}
	\label{fig:execucoes_N20}
\end{figure}


\begin{figure}[htbp]
\centering
  \begin{tabular}{@{}cc@{}}
    \includegraphics[width=.45\textwidth]{figs/resultados/fitnessGrad/N30_01_fitness.pdf} &
    \includegraphics[width=.45\textwidth]{figs/resultados/fitnessGrad/N30_01_rho.pdf}   \\
		\includegraphics[width=.45\textwidth]{figs/resultados/fitnessGrad/N30_02_fitness.pdf} &
    \includegraphics[width=.45\textwidth]{figs/resultados/fitnessGrad/N30_02_rho.pdf}   \\
		\includegraphics[width=.45\textwidth]{figs/resultados/fitnessGrad/N30_03_fitness.pdf} &
    \includegraphics[width=.45\textwidth]{figs/resultados/fitnessGrad/N30_03_rho.pdf}   \\
		\includegraphics[width=.45\textwidth]{figs/resultados/fitnessGrad/N30_04_fitness.pdf} &
    \includegraphics[width=.45\textwidth]{figs/resultados/fitnessGrad/N30_04_rho.pdf}   \\
		\includegraphics[width=.45\textwidth]{figs/resultados/fitnessGrad/N30_05_fitness.pdf} &
    \includegraphics[width=.45\textwidth]{figs/resultados/fitnessGrad/N30_05_rho.pdf}
    %\multicolumn{2}{c}{\includegraphics[width=.23\textwidth]{example-image-a}}
  \end{tabular}
  \caption{Execuções N = 30.}
	\label{fig:execucoes_N30}
\end{figure}

\begin{figure}[htbp]
\centering
  \begin{tabular}{@{}cc@{}}
    \includegraphics[width=.45\textwidth]{figs/resultados/fitnessGrad/N40_01_fitness.pdf} &
    \includegraphics[width=.45\textwidth]{figs/resultados/fitnessGrad/N40_01_rho.pdf}   \\
		\includegraphics[width=.45\textwidth]{figs/resultados/fitnessGrad/N40_02_fitness.pdf} &
    \includegraphics[width=.45\textwidth]{figs/resultados/fitnessGrad/N40_02_rho.pdf}   \\
		\includegraphics[width=.45\textwidth]{figs/resultados/fitnessGrad/N40_03_fitness.pdf} &
    \includegraphics[width=.45\textwidth]{figs/resultados/fitnessGrad/N40_03_rho.pdf}   \\
		\includegraphics[width=.45\textwidth]{figs/resultados/fitnessGrad/N40_04_fitness.pdf} &
    \includegraphics[width=.45\textwidth]{figs/resultados/fitnessGrad/N40_04_rho.pdf}   \\
		\includegraphics[width=.45\textwidth]{figs/resultados/fitnessGrad/N40_05_fitness.pdf} &
    \includegraphics[width=.45\textwidth]{figs/resultados/fitnessGrad/N40_05_rho.pdf}
    %\multicolumn{2}{c}{\includegraphics[width=.23\textwidth]{example-image-a}}
  \end{tabular}
  \caption{Execuções N = 40.}
	\label{fig:execucoes_N40}
\end{figure}

\section{Outro \textit{fitness} para encontrar o mínimo global}

	O novo \emph{fitness}, apresentado em \cite{metodo2011}, é dado por
	
	\begin{equation}\label{eq:fitnessRho0}
		f_i = e^{-\lambda(\rho_i - E_L)^2},
	\end{equation}
e contém semelhanças com o definido pela equação \ref{eq:fitnessGrad2}. Há uso de uma exponencial, o parâmetro $\lambda$ foi mantido e possui exatamente o mesmo papel, $f_i$ depende apenas de $\rho$ e, como $(\rho_i - E_L)^2$ é claramente positivo, o \textit{fitness} continua limitado ao conjunto (0,1]. As diferenças estão na ausência do $\nabla \rho$ e na inclusão do parâmetro $E_L$, que representa um limite inferior para o \textit{menor} autovalor\footnote{L de \textit{lower}.}. Por exemplo, se soubermos de antemão que o autovalor \textit{mínimo} é maior que zero, poderíamos definir $E_L = 0$. 

	 A justificativa para o funcionamento do método em \cite{metodo2011} segue a mesma estrutura de \cite{metodo2004}: ``\textit{Se $\rho_i \rightarrow E_L$ durante a busca, $f_i \rightarrow 1$ e $C_i$ está próximo do autovetor fundamental de $H$}''\footnote{Tradução minha para ``\textit{If $\rho_i \rightarrow E_L$ during the search, $f_i \rightarrow 1$ and $C_i$ approaches the ground eigenvector of $H$}''.}. Parece que, outra vez, não há garantia de que, se $f_i \rightarrow 1$, $\rho$ tende, necessariamente, ao autovalor fundamental. E aqui há um agravante: nada na equação \ref{eq:fitnessRho0} está diretamente associado aos autovalores de $H$. Lembre-se que o \textit{fitness} anterior (equação \ref{eq:fitnessGrad2}) contém $\nabla \rho$, que possui relação direta com os autovalores de $H$ quando $\nabla \rho = 0$.
	
	O autovalor mínimo foi encontrado, mas a qualidade foi inferior. Repeti as execuções da tabela \ref{tab:execucoes10a40} alterando apenas o \textit{fitness} e configurando o parâmetro $E_L$ para $E_L = 0$, um pouco abaixo dos autovalores mínimos. Os resultados estão na página \pageref{tab:execucoesNovoFitness}, e os gráficos da evolução do \textit{fitness} e do quociente de Rayleigh estão nas páginas \pageref{fig:execucoes_N10_EL}, \pageref{fig:execucoes_N20_EL}, \pageref{fig:execucoes_N30_EL} e \pageref{fig:execucoes_N40_EL}. Surpreendentemente, apesar do que foi dito no parágrafo anterior, o programa encontrou o menor autovalor em \textbf{todos} os casos. Assim como nas primeiras execuções, o desvio padrão ($\sigma$) da média de $\rho$ na última geração (400.000) foi pequeno, indicando convergência genética. Entretanto, essa foi a única semelhança. Os próprios valores de $\sigma$ são uma ordem de grandeza menores, sugerindo que os indivíduos são mais semelhantes entre si. O \textit{fitness} médio só atingiu seu valor máximo para a matriz de ordem $N = 40$. Aliás, especificamente para $E_L$ fixado em $E_L = 0$, o <\textit{fitness}> final diminui com $N$, pois $E_L$ está mais distante de $E_0$ na matriz de ordem 10 do que na de ordem 40. Os erros relativos não ultrapassaram $1\%$, mas foram substancialmente maiores comparados aos obtidos com o primeiro \textit{fitness}. Enquanto nos testes anteriores seus valores permaneceram estáveis, agora os erros relativos apresentaram tendência de crescimento com N.
	
	\begin{landscape}
\begin{center}
\begin{table}[htbp]
\caption{Execuções novo \textit{Fitness}.}
\label{tab:execucoesNovoFitness}
	% Table generated by Excel2LaTeX from sheet 'Tabela LaTex'
\begin{tabular}{cccccccccc}
\hline \hline
   \textbf{N} & \textbf{Execução} & \textbf{Semente} & \textbf{$\lambda$} & \textbf{<Fitness>} & \textbf{<$\rho$>} & \textbf{$\sigma$} & \textbf{\# autovalor} & \textbf{Autovalor} & \textbf{Erro relativo (\%)} \\
\hline \hline
        10 &          0 & 1445738835 &   0,128788 &   0,999044 &   0,386176 &    0,00005 &          0 &  0,3860745 &     0,03\% \\
\hline
        10 &          1 & 1445780626 &   0,128788 &   0,999044 &   0,386169 &    0,00003 &          0 &  0,3860745 &     0,02\% \\
\hline
        10 &          2 & 1445780762 &   0,128788 &   0,999045 &   0,386132 &    0,00002 &          0 &  0,3860745 &     0,01\% \\
\hline
        10 &          3 & 1445780907 &   0,128788 &   0,999044 &   0,386175 &    0,00005 &          0 &  0,3860745 &     0,03\% \\
\hline
        10 &          4 & 1445781049 &   0,128788 &   0,999043 &   0,386211 &    0,00003 &          0 &  0,3860745 &     0,04\% \\
\hline
        10 &          5 & 1445781195 &   0,128788 &   0,999044 &   0,386183 &    0,00005 &          0 &  0,3860745 &     0,03\% \\
\hline
        20 &          1 & 1445795292 &   0,026665 &   0,999954 &   0,341484 &    0,00005 &          0 &  0,3412367 &     0,07\% \\
\hline
        20 &          2 & 1445795501 &   0,026665 &   0,999954 &   0,341693 &     0,0001 &          0 &  0,3412367 &      0,1\% \\
\hline
        20 &          3 & 1445795718 &   0,026665 &   0,999954 &    0,34147 &    0,00006 &          0 &  0,3412367 &     0,07\% \\
\hline
        20 &          4 & 1445795953 &   0,026665 &   0,999954 &   0,341689 &     0,0001 &          0 &  0,3412367 &      0,1\% \\
\hline
        20 &          5 & 1445796166 &   0,026665 &   0,999954 &    0,34153 &    0,00007 &          0 &  0,3412367 &     0,09\% \\
\hline
        30 &          1 & 1445796378 &   0,011171 &   0,999995 &   0,320582 &     0,0001 &          0 &   0,319737 &      0,3\% \\
\hline
        30 &          2 & 1445796746 &   0,011171 &   0,999995 &   0,320772 &     0,0002 &          0 &   0,319737 &      0,3\% \\
\hline
        30 &          3 & 1445797109 &   0,011171 &   0,999995 &   0,320699 &     0,0001 &          0 &   0,319737 &      0,3\% \\
\hline
        30 &          4 & 1445797473 &   0,011171 &   0,999995 &   0,320755 &     0,0001 &          0 &   0,319737 &      0,3\% \\
\hline
        30 &          5 & 1445797882 &   0,011171 &   0,999995 &   0,320274 &    0,00007 &          0 &   0,319737 &      0,2\% \\
\hline
        40 &          1 & 1445798248 &   0,006105 &          1 &   0,306968 &     0,0001 &          0 &   0,306086 &      0,3\% \\
\hline
        40 &          2 & 1445798838 &   0,006105 &          1 &   0,307128 &     0,0001 &          0 &   0,306086 &      0,3\% \\
\hline
        40 &          3 & 1445799429 &   0,006105 &          1 &   0,307297 &     0,0002 &          0 &   0,306086 &      0,4\% \\
\hline
        40 &          4 & 1445800091 &   0,006105 &          1 &   0,307816 &     0,0002 &          0 &   0,306086 &      0,6\% \\
\hline
        40 &          5 & 1445800683 &   0,006105 &          1 &    0,30765 &     0,0002 &          0 &   0,306086 &      0,5\% \\

\hline \hline
\end{tabular}
\end{table}  
\end{center}
\end{landscape}
	
	Apesar das diferenças dos valores finais, o comportamento do \textit{fitness} e do $\rho$ ao longo da busca não foi alterado significativamente. Na figura \ref{fig:N-10_E-0_fitness} estão os gráficos referentes à execução zero para o Hamiltoniano de ordem 10, semente 1445738835. A primeira usa o \textit{fitness} $f_i = e^{-\lambda(\rho_i - E_L)^2}$, que chega ao autovalor mínimo, enquanto a segunda utiliza o $f_i = e^{-\lambda | \nabla \rho_i |^2}$. Ambos saem de valores muito baixos e convergem para 1, entretanto, o da esquerda é muito ruidoso e, aparentemente, essa é a causa da convergência mais lenta. Quando a curva da direita já está estável em $\textit{<f>} \approx 1$ em torno da geração de número 15, a da esquerda ainda não ultrapassou o $\textit{<f>} = 0,1$. A princípio, não podemos comparar os dois comportamentos diretamente, visto que cada um chegou em um autovalor diferente. A execução da direita, lembre-se, obteve apenas um mínimo local ($E_1 = 2,461056$, tabela \ref{tab:execucoes10a40}).
	
	\begin{figure}[htbp]
		\centering
			\includegraphics[width=0.48\textwidth]{figs/resultados/fitnessEL/N-10_E-0_fitness.pdf}
			\includegraphics[width=0.48\textwidth]{figs/resultados/fitnessGrad/N10_00_fitness.pdf}
		\caption{Comportamento do \textit{fitness} para as execuções zero do Hamiltoniano de ordem 10, semente 1445738835. A primeira usa o \textit{fitness} $f_i = e^{-\lambda(\rho_i - E_L)^2}$, que chega ao autovalor mínimo, enquanto a segunda utiliza o $f_i = e^{-\lambda \| \nabla \rho_i \|^2}$.}
		\label{fig:N-10_E-0_fitness}
	\end{figure}
	
	De todo modo, as duas execuções estão conectadas pois, como partiram da mesma semente de números pseudoaleatórios, a população inicial foi \textit{exatamente} a mesma. Inclusive, na primeira geração, em ambas as execuções, os valores para $<\rho>$ e para o melhor $\rho$ foram, respectivamente, $9,876075$ e $9,557892$, igualmente distantes do autovalor mínimo $E_0 = 0,386075$. Os gráficos da figura \ref{fig:N-10_E-0_rho_comparacao} permitem comparar a evolução do $<\rho>$ nos dois casos. Assim como na figura anterior, a imagem da esquerda refere-se ao uso do \textit{fitness} $f_i = e^{-\lambda(\rho_i - E_L)^2}$.
	
	É tentador afirmar que a causa de uma execução ter sido mais lenta do que a outra foi porque percorreu um caminho mais longo ao sair de $<\rho> = 9,876075$, passar por $E_1 = 2,461056$ e continuar até encontrar $E_0 = 0,386075$, enquanto a mais rápida saiu do mesmo $<\rho>$ e parou logo que encontrou $E_1$. Infelizmente essa conclusão estaria incorreta. A maneira como os Algoritmos Genéticos viajam no espaço de soluções tem forte base estocástica e, portanto, qualquer comparação linear é extremamente arriscada, quiçá impossível. Objetivamente, posso apenas concluir que os valores finais encontrados por cada \textit{fitness} estão condizentes com a construção de cada função objetivo: $\nabla \rho_i$ leva a qualquer autovalor; $\rho_i - E_L$, com $E_L$ configurado apropriadamente, encontra o autovalor mínimo.
	
	
	\begin{figure}[htbp]
		\centering
			\includegraphics[width=0.48\textwidth]{figs/resultados/fitnessEL/N-10_E-0_rho.pdf}
			\includegraphics[width=0.48\textwidth]{figs/resultados/fitnessGrad/N10_00_rho.pdf}
		\caption{Comportamento do $\rho$ para as execuções zero do Hamiltoniano de ordem 10, semente 1445738835. A primeira usa o \textit{fitness} $f_i = e^{-\lambda(\rho_i - E_L)^2}$, que chega ao autovalor mínimo, enquanto a segunda utiliza o $f_i = e^{-\lambda \| \nabla \rho_i \|^2}$.}
		\label{fig:N-10_E-0_rho_comparacao}
	\end{figure}	
	
	O limite inferior $E_L$ é o responsável pelo funcionamento do \emph{fitness} \ref{eq:fitnessRho0} pois o termo $(\rho_i - E_L)$ minimiza diretamente $\rho$. Se $(\rho_i - E_L)$ é grande, $f_i$ é pequeno, e isso significa que o vetor $C_i$ associado a $\rho_i$ está distante do autovetor fundamental $C_0$. Portanto, indivíduos com $(\rho_i - E_L)$ menores (e \emph{fitness} maiores) serão selecionados mais vezes. De acordo com a equação \ref{eq:autovalores_ordenados}, o processo fica estável quando é impossível diminuir $\rho_i$, ou seja, quando $<\rho> \rightarrow \lambda_1$. 

	À primeira vista, esse \emph{fitness} não parece ser útil. Os valores finais do \emph{fitness} só foram próximos de 1 porque escolhi $E_L$ próximo de $E_0$. Ou seja, eu já tinha conhecimento prévio do menor autovalor. Se o objetivo é encontrar $E_0$ e eu não tenho conhecimento sobre a região onde ele se encontra, como escolher apropriadamente $E_L$? Os autores de \cite{metodo2011} não falam nada a respeito.
	
	Então, para a mesma semente 1445738835, $\lambda = 0,128788$ e $N = 10$, cujo $E_0 = 0,386075$, testei quatro cenários. O objetivo foi verificar em quais condições o autovalor mínimo é encontrado. O resumo dos resultados está na tabela \ref{tab:VariandoELPraPrimeiraExecucao}.
	
	\begin{itemize}
	
		\item \textbf{Cenário 1}: $E_L$ um pouco acima de $E_0$.
			
		\item \textbf{Cenário 2}: $E_L$ um pouco abaixo do $E_0$.
		
		\item \textbf{Cenário 3}: $E_L$ muito acima de $E_0$.
		
		\item \textbf{Cenário 4}: $E_L$ muito abaixo de $E_0$.
	\end{itemize}
	
	
	Não há surpresa no \textbf{Cenário 2}. Como já citado no início dessa seção, a escolha de $E_L$ um pouco abaixo de $E_0$ garante o mínimo global. Nos testes obtive o menor autovalor em todas as execuções, com \emph{fitness} médio próximo de 1 e $<|\nabla \rho|>$ próximo de zero. Portanto, o \textbf{Cenário 2} é o melhor cenário possível.	
	
	Apesar de não obter $E_0$, o \textbf{Cenário 1} dá uma informação importante. O algoritmo termina sempre quando $<\rho>$ fica próximo de $E_L$, mas não obtém $E_0$. Como pode ser visto na figura \ref{fig:execucoesSemente_EL_umPoucoAcima}, $<f>$ chega ao valor máximo 1 (gráfico da esquerda), mas o valor final médio para $\rho$ foi $E_{médio} = 0,387001$ (gráfico da direita). Então, se o algoritmo parar em $E_L$, significa que o autovalor mínimo é, com certeza, menor: $E_0 < E_L$. 
	
	\begin{figure}[htbp]
	\centering
  \begin{tabular}{@{}cc@{}}
	
		\includegraphics[width=.45\textwidth]{figs/resultados/variandoELSemente/T1_S-1445738835_fitness.pdf} &
    \includegraphics[width=.45\textwidth]{figs/resultados/variandoELSemente/T1_S-1445738835_rho.pdf}
  \end{tabular}
  \caption{Cenário 1. Execução para a semente 1445738835. $E_L$ um pouco acima de $E_0$ no \textit{fitness} $f_i = e^{-\lambda(\rho_i - E_L)^2}$.}
	\label{fig:execucoesSemente_EL_umPoucoAcima}
	\end{figure}
	
	Com relação ao \textbf{Cenário 3} (\emph{muito} acima), novamente o algoritmo chegou ao $E_L$ em todas as execuções. Entretando, na última geração $|\nabla\rho|$ foi mais do que trinta vezes maior comparado com o \textbf{Cenário 1} ($0,003 / 0,00009 \approx 33$). Retomando a condição de estacionaridade do Quociente de Rayleigh (equação \ref{eq:grad_rho_nulo}), isso significa que $<\rho>$ está mais distante de algum autovalor.

	\begin{figure}[htbp]
	\centering
  \begin{tabular}{@{}cc@{}}	
		\includegraphics[width=.45\textwidth]{figs/resultados/variandoELSemente/T3_S-1445738835_fitness.pdf} &
    \includegraphics[width=.45\textwidth]{figs/resultados/variandoELSemente/T3_S-1445738835_rho.pdf}
  \end{tabular}
  \caption{Execução para a semente 1445738835. $E_L$ muito acima de $E_0$ no \textit{fitness} $f_i = e^{-\lambda(\rho_i - E_L)^2}$.}
	\label{fig:execucoesSemente_EL_umMuitoAcima}
	\end{figure}
		
	O valor de $\lambda = 0,128788$, utilizado em todos os cenários anteriores, não é adequado para \textbf{Cenário 4}. Logo no início tanto o \emph{<fitness>} quanto o maior \emph{fitness} foram zero.  Nesse regime não há como, a princípio, distinguir os indivíduos, pois todos possuem avaliação muito próxima. Veja no segundo gráfico da figura \ref{fig:execucoesSemente_EL_umMuitoAbaixo500} que $<\rho>$ rapidamente fica estagnado um pouco abaixo de 6. Especificamente, na geração 500 $<\rho> = 5,651846$, que não corresponde a nenhum autovalor para uma matriz de Coope de ordem 10 (tabela \ref{tab:autovalores10a40}). Esse é um exemplo de \emph{underflow} do \emph{fitness}.
	
	\begin{figure}[htbp]
	\centering
  \begin{tabular}{@{}cc@{}}	
		\includegraphics[width=.45\textwidth]{figs/resultados/variandoELSemente/T4_S-1445738835_fitness.pdf} &
    \includegraphics[width=.45\textwidth]{figs/resultados/variandoELSemente/T4_S-1445738835_rho.pdf}
  \end{tabular}
  \caption{Execução para a semente 1445738835. $E_L$ muito abaixo de $E_0$ no \textit{fitness} $f_i = e^{-\lambda(\rho_i - E_L)^2}$. Até geração 500.}
	\label{fig:execucoesSemente_EL_umMuitoAbaixo500}
	\end{figure}
	
	Porém, algo inesperado aconteceu. A partir da geração 11 e até a 20.000, aproximadamente, há um decréscimo sistemático do $<\rho>$, com taxa muito pequena ($\partial <\rho> / \partial t < 0,001\%$). Na geração 31757 decréscimo de 24\%;.  pouco acima dahouve convergência genética precoce, estabilizando a média dos $\rho$ em aproximadamente 5.9 (verificar), que não é nenhum autovalor pra N = 10.
	
	Entretanto, após várias gerações houve convergência para o autovalor mínimo. Um pouco antes da geração 32.000 aconteceu um salto no \emph{fitness}, o que só pode ter sido possível com a criação de variabilidade genética na população. Se o \emph{fitness} estava estável desde a geração 500, Se, causado certamente por mutações, visto que os indivíduos eram semelhantes desde a geração 500. Apesar do \emph{fitness} médio ainda ser pequeno (<$f_i$>$< 0.025$), o \emph{crossover} foi capaz com a nova informação genética criada pela mutação, criou variabilidade suficiente para chegar ao autovalor mínimo.
	
	\begin{figure}[htbp]
	\centering
  \begin{tabular}{@{}cc@{}}	
		\includegraphics[width=.45\textwidth]{figs/resultados/variandoELSemente/T4_S-1445738835_fitness-extendido.pdf} &
    \includegraphics[width=.45\textwidth]{figs/resultados/variandoELSemente/T4_S-1445738835_rho_extendido.pdf}
  \end{tabular}
  \caption{Execução para a semente 1445738835. $E_L$ muito abaixo de $E_0$ no \textit{fitness} $f_i = e^{-\lambda(\rho_i - E_L)^2}$. Geração entre 30.000 e 40.000.}
	\label{fig:execucoesSemente_EL_umMuitoAbaixo40000}
	\end{figure}
	

	Na tabela \ref{tab:VariandoELPraPrimeiraExecucao} há os valores desses testes. Como nas tabelas anteriores, os valores médios de $\rho$ e do \emph{fitness} (<$\rho$> e <\emph{fitness}>) foram calculados na geração final, ou seja, na população que atingiu algum dos critérios de parada. O <$\rho$> foi comparado com $E_0 = 0,386075$ para calcular o erro relativo (coluna Erro do <$\rho$> (\%)).


%\begin{landscape}
\begin{center}	
\begin{table}[htbp]
\caption{Variando $E_L$ para a execução da semente 1445738835. Os tipos de teste são: \textbf{cenário 1}: $E_L$ um pouco acima de $E_0$; \textbf{cenário 2}: $E_L$ um pouco abaixo de $E_0$; \textbf{cenário 3}: $E_L$ muito acima de $E_0$; \textbf{cenário 4}: $E_L$ muito abaixo de $E_0$.}
\label{tab:VariandoELPraPrimeiraExecucao}
\scalefont{0.7}
\centering
% Table generated by Excel2LaTeX from sheet 'Plan2'
\begin{tabular}{cccccccc}
\hline \hline
\textbf{Teste} &  \textbf{$E_L$} & \textbf{Geração final} & \textbf{<$\rho$>} & \textbf{$\sigma$} & \textbf{Erro do <$\rho$> (\%)} & \textbf{|$\nabla \rho$|} & \textbf{<\emph{Fitness}>} \\
\hline \hline
         1 &   0,387000 &     42.577 &     0,3870 &     0,0004 &      0,2\% &    0,00009 &   1,000000 \\
\hline
         2 &   0,385000 &    400.000 &    0,38615 &    0,00003 &     0,02\% &   0,000006 &   1,000000 \\
\hline
         3 &   5,000000 &      9.622 &       5,00 &       0,02 &     1195\% &      0,003 &   0,999966 \\
\hline
         4 &  -5,000000 &    400.000 &    0,38617 &    0,00003 &     0,03\% &     0,0003 &   0,023843 \\
\hline \hline
\end{tabular}   
\end{table}
\end{center}	
%%\end{landscape}
	
	
	-------------------
	
	Concluir. Ponto para o lâmbda.
	
\begin{landscape}
\begin{center}	
\begin{table}[htbp]
\caption{Cinco execuções para cada tipo de teste de variação de $E_L$ em torno de $E_0$ no fitness $f_i = e^{-\lambda(\rho_i - E_L)^2}$.}
\label{tab:VariandoELCincoExecucoes}
\centering
% Table generated by Excel2LaTeX from sheet 'Plan2'
\begin{tabular}{ccccccccc}
\hline \hline
\textbf{Teste} & \textbf{Execução} & \textbf{Semente} & \textbf{Geração final} & \textbf{<$\rho$>} & \textbf{$\sigma$} & \textbf{Erro do <$\rho$> (\%)} & \textbf{|$\nabla \rho$|} & \textbf{<\emph{Fitness}>} \\
\hline \hline
         1 &          1 & 1448150274 &     47.945 &     0,3870 &     0,0005 &  0,00005\% &    0,00008 &   1,000000 \\
\hline
         1 &          2 & 1448150289 &     24.128 &     0,3870 &     0,0004 & -0,00004\% &    0,00008 &   1,000000 \\
\hline
         1 &          3 & 1448150298 &     40.795 &     0,3870 &     0,0003 & 0,000007\% &    0,00008 &   1,000000 \\
\hline
         1 &          4 & 1448150315 &     17.047 &     0,3870 &     0,0005 &  -0,0001\% &     0,0001 &   1,000000 \\
\hline
         1 &          5 & 1448150321 &     16.284 &     0,3870 &     0,0003 &  0,00002\% &    0,00008 &   1,000000 \\
\hline \hline
         2 &          1 & 1448150327 &    400.000 &    0,38616 &    0,00003 &     0,02\% &   0,000009 &   1,000000 \\
\hline
         2 &          2 & 1448150472 &    400.000 &    0,38613 &    0,00002 &     0,01\% &   0,000005 &   1,000000 \\
\hline
         2 &          3 & 1448150600 &    400.000 &    0,38613 &    0,00002 &     0,02\% &   0,000005 &   1,000000 \\
\hline
         2 &          4 & 1448150704 &    400.000 &    0,38624 &    0,00008 &     0,04\% &    0,00002 &   1,000000 \\
\hline
         2 &          5 & 1448150809 &    400.000 &    0,38624 &    0,00007 &     0,04\% &    0,00001 &   1,000000 \\
\hline \hline
         3 &          1 & 1448150912 &      8.074 &       5,00 &       0,05 &  0,00002\% &      0,007 &   0,999750 \\
\hline
         3 &          2 & 1448150914 &     14.604 &       5,00 &       0,03 & -0,000005\% &      0,009 &   0,999889 \\
\hline
         3 &          3 & 1448150918 &     41.659 &       5,00 &       0,02 & -0,00002\% &      0,003 &   0,999954 \\
\hline
         3 &          4 & 1448150929 &      9.775 &       5,00 &       0,03 & 0,000009\% &      0,006 &   0,999886 \\
\hline
         3 &          5 & 1448150932 &     12.637 &       5,00 &       0,03 & -0,0000006\% &      0,005 &   0,999904 \\
\hline \hline
         4 &          1 & 1448150935 &    400.000 &     0,3864 &     0,0001 &     0,07\% &      0,001 &   0,023837 \\
\hline
         4 &          2 & 1448151040 &    400.000 & 7,98166818 & 0,00000001 &     1967\% &        6,0 &   0,000000 \\
\hline
         4 &          3 & 1448151146 &    400.000 & 10,564998429558 & 0,000000000002 &     2637\% &        8,6 &   0,000000 \\
\hline
         4 &          4 & 1448151251 &    400.000 &    0,38613 &    0,00002 &     0,02\% &     0,0003 &   0,023844 \\
\hline
         4 &          5 & 1448151357 &    400.000 &    0,38614 &    0,00003 &     0,02\% &     0,0003 &   0,023844 \\
\hline \hline
\end{tabular}  
\end{table}
\end{center}	
\end{landscape}

\begin{figure}[p]
	\centering
  \begin{tabular}{@{}cc@{}}
    \includegraphics[width=.40\textwidth]{figs/resultados/fitnessEL/N-10_E-0_fitness-extendido.pdf} &
    \includegraphics[width=.40\textwidth]{figs/resultados/fitnessEL/N-10_E-0_rho_extendido.pdf}   \\
		\includegraphics[width=.40\textwidth]{figs/resultados/fitnessEL/N-10_E-1_fitness-extendido.pdf} &
    \includegraphics[width=.40\textwidth]{figs/resultados/fitnessEL/N-10_E-1_rho_extendido.pdf}   \\
		\includegraphics[width=.40\textwidth]{figs/resultados/fitnessEL/N-10_E-2_fitness-extendido.pdf} &
    \includegraphics[width=.40\textwidth]{figs/resultados/fitnessEL/N-10_E-2_rho_extendido.pdf}   \\
		\includegraphics[width=.40\textwidth]{figs/resultados/fitnessEL/N-10_E-3_fitness-extendido.pdf} &
    \includegraphics[width=.40\textwidth]{figs/resultados/fitnessEL/N-10_E-3_rho_extendido.pdf}   \\
		\includegraphics[width=.40\textwidth]{figs/resultados/fitnessEL/N-10_E-4_fitness-extendido.pdf} &
    \includegraphics[width=.40\textwidth]{figs/resultados/fitnessEL/N-10_E-4_rho_extendido.pdf} \\
		\includegraphics[width=.40\textwidth]{figs/resultados/fitnessEL/N-10_E-5_fitness-extendido.pdf} &
    \includegraphics[width=.40\textwidth]{figs/resultados/fitnessEL/N-10_E-5_rho_extendido.pdf}
  \end{tabular}
  \caption{Execuções para N = 10 com o \textit{fitness} $f_i = e^{-\lambda(\rho_i - E_L)^2}$.}
	\label{fig:execucoes_N10_EL}
	\end{figure}
		
		\begin{figure}[p]
	\centering
  \begin{tabular}{@{}cc@{}}
		\includegraphics[width=.40\textwidth]{figs/resultados/fitnessEL/N-20_E-1_fitness-extendido.pdf} &
    \includegraphics[width=.40\textwidth]{figs/resultados/fitnessEL/N-20_E-1_rho_extendido.pdf}   \\
		\includegraphics[width=.40\textwidth]{figs/resultados/fitnessEL/N-20_E-2_fitness-extendido.pdf} &
    \includegraphics[width=.40\textwidth]{figs/resultados/fitnessEL/N-20_E-2_rho_extendido.pdf}   \\
		\includegraphics[width=.40\textwidth]{figs/resultados/fitnessEL/N-20_E-3_fitness-extendido.pdf} &
    \includegraphics[width=.40\textwidth]{figs/resultados/fitnessEL/N-20_E-3_rho_extendido.pdf}   \\
		\includegraphics[width=.40\textwidth]{figs/resultados/fitnessEL/N-20_E-4_fitness-extendido.pdf} &
    \includegraphics[width=.40\textwidth]{figs/resultados/fitnessEL/N-20_E-4_rho_extendido.pdf} \\
		\includegraphics[width=.40\textwidth]{figs/resultados/fitnessEL/N-20_E-5_fitness-extendido.pdf} &
    \includegraphics[width=.40\textwidth]{figs/resultados/fitnessEL/N-20_E-5_rho_extendido.pdf}
  \end{tabular}
  \caption{Execuções para N = 20 com o \textit{fitness} $f_i = e^{-\lambda(\rho_i - E_L)^2}$.}
	\label{fig:execucoes_N20_EL}
	\end{figure}
	
		\begin{figure}[p]
	\centering
  \begin{tabular}{@{}cc@{}}
		\includegraphics[width=.40\textwidth]{figs/resultados/fitnessEL/N-30_E-1_fitness-extendido.pdf} &
    \includegraphics[width=.40\textwidth]{figs/resultados/fitnessEL/N-30_E-1_rho_extendido.pdf}   \\
		\includegraphics[width=.40\textwidth]{figs/resultados/fitnessEL/N-30_E-2_fitness-extendido.pdf} &
    \includegraphics[width=.40\textwidth]{figs/resultados/fitnessEL/N-30_E-2_rho_extendido.pdf}   \\
		\includegraphics[width=.40\textwidth]{figs/resultados/fitnessEL/N-30_E-3_fitness-extendido.pdf} &
    \includegraphics[width=.40\textwidth]{figs/resultados/fitnessEL/N-30_E-3_rho_extendido.pdf}   \\
		\includegraphics[width=.40\textwidth]{figs/resultados/fitnessEL/N-30_E-4_fitness-extendido.pdf} &
    \includegraphics[width=.40\textwidth]{figs/resultados/fitnessEL/N-30_E-4_rho_extendido.pdf} \\
		\includegraphics[width=.40\textwidth]{figs/resultados/fitnessEL/N-30_E-5_fitness-extendido.pdf} &
    \includegraphics[width=.40\textwidth]{figs/resultados/fitnessEL/N-30_E-5_rho_extendido.pdf}
  \end{tabular}
  \caption{Execuções para N = 30 com o \textit{fitness} $f_i = e^{-\lambda(\rho_i - E_L)^2}$.}
	\label{fig:execucoes_N30_EL}
	\end{figure}
	
		\begin{figure}[p]
	\centering
  \begin{tabular}{@{}cc@{}}
		\includegraphics[width=.40\textwidth]{figs/resultados/fitnessEL/N-40_E-1_fitness-extendido.pdf} &
    \includegraphics[width=.40\textwidth]{figs/resultados/fitnessEL/N-40_E-1_rho_extendido.pdf}   \\
		\includegraphics[width=.40\textwidth]{figs/resultados/fitnessEL/N-40_E-2_fitness-extendido.pdf} &
    \includegraphics[width=.40\textwidth]{figs/resultados/fitnessEL/N-40_E-2_rho_extendido.pdf}   \\
		\includegraphics[width=.40\textwidth]{figs/resultados/fitnessEL/N-40_E-3_fitness-extendido.pdf} &
    \includegraphics[width=.40\textwidth]{figs/resultados/fitnessEL/N-40_E-3_rho_extendido.pdf}   \\
		\includegraphics[width=.40\textwidth]{figs/resultados/fitnessEL/N-40_E-4_fitness-extendido.pdf} &
    \includegraphics[width=.40\textwidth]{figs/resultados/fitnessEL/N-40_E-4_rho_extendido.pdf}		\\
		\includegraphics[width=.40\textwidth]{figs/resultados/fitnessEL/N-40_E-5_fitness-extendido.pdf} &
    \includegraphics[width=.40\textwidth]{figs/resultados/fitnessEL/N-40_E-5_rho_extendido.pdf}
  \end{tabular}
  \caption{Execuções para N = 40 com o \textit{fitness} $f_i = e^{-\lambda(\rho_i - E_L)^2}$.}
	\label{fig:execucoes_N40_EL}
	\end{figure}
	
	%-------------------------------------------------
	
	\begin{figure}[p]
	\centering
  \begin{tabular}{@{}cc@{}}
    
		\includegraphics[width=.40\textwidth]{figs/resultados/variandoEL/T1E1_fitness.pdf} &
    \includegraphics[width=.40\textwidth]{figs/resultados/variandoEL/T1E1_rho.pdf}   \\
		
		\includegraphics[width=.40\textwidth]{figs/resultados/variandoEL/T1E2_fitness.pdf} &
    \includegraphics[width=.40\textwidth]{figs/resultados/variandoEL/T1E2_rho.pdf}   \\
		
		\includegraphics[width=.40\textwidth]{figs/resultados/variandoEL/T1E3_fitness.pdf} &
    \includegraphics[width=.40\textwidth]{figs/resultados/variandoEL/T1E3_rho.pdf}   \\
		
		\includegraphics[width=.40\textwidth]{figs/resultados/variandoEL/T1E4_fitness.pdf} &
    \includegraphics[width=.40\textwidth]{figs/resultados/variandoEL/T1E4_rho.pdf}   \\
		
		\includegraphics[width=.40\textwidth]{figs/resultados/variandoEL/T1E5_fitness.pdf} &
    \includegraphics[width=.40\textwidth]{figs/resultados/variandoEL/T1E5_rho.pdf}		
  \end{tabular}
  \caption{Cenário 1. Várias execuções com o $E_L$ um pouco acima de $E_0$ no \textit{fitness} $f_i = e^{-\lambda(\rho_i - E_L)^2}$. Semente 1445738835, N = 10.}
	\label{fig:variando_EL_pouco_acima}
	\end{figure}
	
\begin{figure}[p]
	\centering
  \begin{tabular}{@{}cc@{}}
			
		\includegraphics[width=.40\textwidth]{figs/resultados/variandoEL/T2E1_fitness.pdf} &
    \includegraphics[width=.40\textwidth]{figs/resultados/variandoEL/T2E1_rho.pdf}   \\

		\includegraphics[width=.40\textwidth]{figs/resultados/variandoEL/T2E2_fitness.pdf} &
    \includegraphics[width=.40\textwidth]{figs/resultados/variandoEL/T2E2_rho.pdf}   \\
		
		\includegraphics[width=.40\textwidth]{figs/resultados/variandoEL/T2E3_fitness.pdf} &
    \includegraphics[width=.40\textwidth]{figs/resultados/variandoEL/T2E3_rho.pdf}   \\
		
		\includegraphics[width=.40\textwidth]{figs/resultados/variandoEL/T2E4_fitness-extendido.pdf} &
    \includegraphics[width=.40\textwidth]{figs/resultados/variandoEL/T2E4_rho_extendido.pdf}   \\
		
		\includegraphics[width=.40\textwidth]{figs/resultados/variandoEL/T2E5_fitness.pdf} &
    \includegraphics[width=.40\textwidth]{figs/resultados/variandoEL/T2E5_rho.pdf}

  \end{tabular}
  \caption{Execuções com o $E_L$ um pouco abaixo de $E_0$ no \textit{fitness} $f_i = e^{-\lambda(\rho_i - E_L)^2}$. Semente 1445738835, N = 10.}
	\label{fig:variando_EL_pouco_abaixo}
	\end{figure}

\begin{figure}[p]
	\centering
  \begin{tabular}{@{}cc@{}}
   			
		\includegraphics[width=.40\textwidth]{figs/resultados/variandoEL/T3E1_fitness.pdf} &
    \includegraphics[width=.40\textwidth]{figs/resultados/variandoEL/T3E1_rho.pdf}   \\

		\includegraphics[width=.40\textwidth]{figs/resultados/variandoEL/T3E2_fitness.pdf} &
    \includegraphics[width=.40\textwidth]{figs/resultados/variandoEL/T3E2_rho.pdf}   \\
		
		\includegraphics[width=.40\textwidth]{figs/resultados/variandoEL/T3E3_fitness.pdf} &
    \includegraphics[width=.40\textwidth]{figs/resultados/variandoEL/T3E3_rho.pdf}   \\
		
		\includegraphics[width=.40\textwidth]{figs/resultados/variandoEL/T3E4_fitness.pdf} &
    \includegraphics[width=.40\textwidth]{figs/resultados/variandoEL/T3E4_rho.pdf}   \\
		
		\includegraphics[width=.40\textwidth]{figs/resultados/variandoEL/T3E5_fitness.pdf} &
    \includegraphics[width=.40\textwidth]{figs/resultados/variandoEL/T3E5_rho.pdf}
		
  \end{tabular}
  \caption{Execuções com o $E_L$ muito acima de $E_0$ no \textit{fitness} $f_i = e^{-\lambda(\rho_i - E_L)^2}$. Semente 1445738835, N = 10.}
	\label{fig:variando_EL_muito_acima}
	\end{figure}

\begin{figure}[p]
	\centering
  \begin{tabular}{@{}cc@{}}
 			
		\includegraphics[width=.40\textwidth]{figs/resultados/variandoEL/T4E1_fitness-extendido.pdf} &
    \includegraphics[width=.40\textwidth]{figs/resultados/variandoEL/T4E1_rho_extendido.pdf}   \\

		\includegraphics[width=.40\textwidth]{figs/resultados/variandoEL/T4E2_fitness-extendido.pdf} &
    \includegraphics[width=.40\textwidth]{figs/resultados/variandoEL/T4E2_rho_extendido.pdf}   \\
		
		\includegraphics[width=.40\textwidth]{figs/resultados/variandoEL/T4E3_fitness-extendido.pdf} &
    \includegraphics[width=.40\textwidth]{figs/resultados/variandoEL/T4E3_rho_extendido.pdf}   \\
		
		\includegraphics[width=.40\textwidth]{figs/resultados/variandoEL/T4E4_fitness-extendido.pdf} &
    \includegraphics[width=.40\textwidth]{figs/resultados/variandoEL/T4E4_rho_extendido.pdf}   \\
		
		\includegraphics[width=.40\textwidth]{figs/resultados/variandoEL/T4E5_fitness-extendido.pdf} &
    \includegraphics[width=.40\textwidth]{figs/resultados/variandoEL/T4E5_rho_extendido.pdf}
		
  \end{tabular}
  \caption{Execuções com o $E_L$ muito abaixo de $E_0$ no \textit{fitness} $f_i = e^{-\lambda(\rho_i - E_L)^2}$. Semente 1445738835, N = 10.}
	\label{fig:variando_EL_muito_abaixo}
	\end{figure}
	
	
	\section{Por que o $\lambda$ deve ser escolhido cuidadosamente?}
	
	Execuções para N=10 com diferentes $\lambda$'s. Com os gráficos, explicar o que o artigo de 2004 quis dizer com \textit{fitness overflow/underflow}.
	
	Gráficos com rho entre 0 e 250 (exemplo pra N=10), mas com cortes em diferentes rhos.

	Explicar que uma boa escolha do $\lambda$ deve cobrir todos os autovalores. Citar as execuções anteriores (boas e ruins em função de cada $\lambda$).
	
	Gráfico com $\lambda$ fazendo o fitness cortar em um $\rho$ muito baixo. Discutir puxando as execuções anteriores.
	
	Outro gráfico, mas com $\lambda$ fazendo o fitness cortar em um $\rho$ muito alto. Discutir puxando as execuções anteriores.
	
	Gráfico com uma boa escolha de $\lambda$. Discutir puxando as execuções anteriores.
	
	Após estimativa, refinar a obtenção do $\lambda$. Alterar o $lambda$ (valores em torno da estimativa), executar o programa para verificar se o fitness médio da primeira população é baixo. (se a população inicial tem fitness muito grande, há convergência prematura).
	
	Tabela com alguns $lambdas$ encontrados dessa maneira (estimativa e refinamento).
	
	Infelizmente, para cada matriz, um $\lambda$ diferente.
	
	Ponte pra equação empírica do $\lambda$.

%=========================================================
	\section{Equação empírica para o $\lambda$}\label{sec:eq_lambda}
%=========================================================
	
	Delineamento da equação como feito na reunião de 29/09.
	
	Isolar $\lambda$ a partir da $f=e^{-\lambda*(\rho - \rho_0)^2}$
	
	Fazer $f = 0.00001 \approx 0$.
	
	Substituir $(\rho - \rho_0)^2$ por $E_{central} - E_{mínimo}$. Justificar.
	
	Regressão linear para $E_{central} - E_{mínimo}$ com função apenas da ordem da matriz (N).
	
	Inserir a Equação obtida na regressão na equação de $\lambda$.
	
	Fator $0.65$: obtido empiricamente de modo que o $\lambda$ seja semelhante aos encontrados pelo processo de estimativa e refinamento.
	
	Exemplo de execução com $\lambda$ automático.
	
	Explicitar que essa equação é válida apenas para matrizes de Coope$-$Sabo. Apesar disso, foi importante para o estudo pois permitiu automação completa.
		
	\section{A mistura de $(\rho - \rho_0)^2$ com $\nabla\rho$ não leva a melhores resultados}
	
	Como em seção anterior verificamos que $f_i = e^{[-\lambda \nabla \rho]}$ é mais rápido do que $f_i = e^{[-\lambda (\nabla \rho)]}$, e que o $\nabla\rho$ está diretamente associado aos autovalores, pensei na seguinte hipótese: inserir $\nabla \rho$ ao fitness com $(\rho - \rho_0)^2$ traria resultados mais rápidos.
	
	Justificativas para a hipótese: 
	
	\begin{enumerate}
		\item Inserir $\nabla \rho$ no fitness puniria os $\rho$'s que, apesar de próximos de $\rho_0$, não fossem autovalor. Em outras palavras, o termo $\rho - \rho_0 \approx 0$, mas $\nabla \rho >> 0$ e, portanto, o fitness ficaria pequeno.
		
		\item Como o fitness, a princípio, estaria diferenciamento melhor os bons indivíduos, o algoritmo teria uma taxa de convergência maior.
		
	\end{enumerate}
	
		Executar $10$ para o primeiro fitness, e, utilizando as mesmas dez sementes, executar outros $10$ testes com ou outro fitness.
		
		Comparação dos resultados: gráficos do comportamento do fitness e tabela comparando a velocidade de convergência (em que geração o critério de parada foi atingido), tempo de execução e erro relativo ao menor autovalor ``exato'' (obtido no SciLab).

	\section{$f_i = e^{[-\lambda \nabla \rho]}$ é mais rápido do que $f_i = e^{[-\lambda (\nabla \rho)^2]}$}
	
	Como um dos critérios de parada utiliza $\nabla \rho$ (sem quadrado), testamos essa forma no fitness.
	
	Várias execuções.
	
	Gráfico comparando o comportamento (um termina mais rápido)
	
	Tabela com os detalhes explícitos do do ganho.
	
	Ponte pra falar sobre o outro fitness que encontra o mínimo.
	
	\section{Resultados preliminares na GPU}\label{sec:oneMaxNaGPU}
	
		O GA aqui desenvolvido buscou a solução do problema ONEMAX, cujo objetivo é encontrar uma sequência de $N$ bits com a maior quantidade possível de “1” a partir de uma sequência aleatória de “1” e “0”. O ONEMAX é especialmente indicado para o início dos estudos em GA. Além de permitir simples implementação, possui representação cromossomial binária que, junto com o crossover de ponto único, forma a base da teoria original de Holland \cite{Linden2008}.

	O programa paralelizado foi uma tradução literal do seu equivalente serial para a sintaxe do CUDA C. Ou seja, não houve nenhuma mudança estrutural no código, seja nas variáveis e estruturas de dados, seja na ordem de execução das funções e procedimentos. Apenas o preenchimento aleatório da população inicial é executado na CPU, de forma que o núcleo do programa é executado inteiramente na GPU \cite{onemaxNaGPU}. 

	A população do GA era constituída por indivíduos formados por cromossomos com \textit{\texttt{numGenes}} elementos do tipo \texttt{char}. Cada elemento do vetor (gene) podia ter um valor  “1” ou “0”. Dentro do problema ONEMAX, os melhores indivíduos foram os que apresentaram maior número de genes iguais a “1”.
		
	\begin{figure}[htbp]
		\centering
			\includegraphics[width=0.50\textwidth]{figs/resultados/onemax/onemax_objetivo.png}
		\caption{ONEMAX, um problema clássico nos Algoritmos Genéticos.}
		\label{fig:onemax_objetivo}
	\end{figure}
		
	Implementei um \emph{kernel} (função que tem sua execução feita pela GPU) para cada um dos quatro passos do GA: cálculo da função avaliadora (\emph{fitness}), seleção, \emph{crossover} e mutação. Eles são chamados um após o outro até que um número máximo de gerações seja atingido. Isso é feito dentro do \emph{loop} principal (realizado na CPU), mas sem troca de informação entre CPU e GPU.
	
	No início do programa duas gerações são alocadas na memória global da GPU, as quais são usadas alternadamente como \emph{input} e \emph{output} dos kernels. Apenas as chamadas dos kernels acontecem na CPU, enquanto o restante (execução + dados) está na GPU. Todos os \emph{kernels} tinham como \emph{input} e \emph{output} uma estrutura do tipo Geração. Ou seja, as funções operaram sobre toda a população do GA, levando-nos a adotar como estratégia o paralelismo no nível dos indivíduos.
	
	\begin{figure}[htbp]
		\centering
			\includegraphics[width=1.00\textwidth]{figs/resultados/onemax/onemax_execucao.png}
		\caption{Execucao do ONEMAX paralelo. Apenas as chamadas dos kernels acontecem na CPU, enquanto o restante (execução + dados) está na GPU.}
		\label{fig:onemax_execucao}
	\end{figure}
		
	No cálculo do \emph{fitness}, o \emph{input} foi uma população com \textit{\texttt{numIndividuos}} e o \emph{output} uma população com os mesmos \textit{\texttt{numIndividuos}} e suas respectivas notas. O cálculo do \emph{fitness} de cada indivíduo foi realizado por meio da soma dos valores de seus genes. Por exemplo, para um indivíduo formado por um cromossomo de 6 genes (\textit{\texttt{numGenes}} = 6) com a configuração “010101”, o valor do fitness é 3 e a solução ótima para esse caso seria “111111” (figura \ref{fig:onemax_objetivo}). No código serial a programação é simples e envolve apenas um laço for que percorre o cromossomo e soma os bytes. Porém, note que toda informação necessária para esse cálculo está contida no próprio cromossomo, ou seja, a obtenção do \emph{fitness} de um dado indivíduo não depende do restante da população. Assim, a paralelização do cálculo do \emph{fitness} deu-se por meio da associação de uma \emph{thread} para cada indivíduo.
	
	Para o operador de seleção, optei pela seleção via torneio com o tamanho do torneio fixo e igual a dois. Novamente, o \emph{input} e o \emph{output} são populações. Na entrada há \textit{\texttt{numIndividuos}} e suas notas. Os mais aptos (maiores \emph{fitness}) têm maiores chances de serem selecionados, e compõem os \textit{\texttt{numIndividuos}} da população na saída. Assim como no cálculo do \emph{fitness}, o paralelismo acontece no nível dos indivíduos. Para cada indivíduo na população de saída há uma \emph{thread}, que seleciona aleatoriamente dois cromossomos na população da entrada (memória global) e fica com o de maior \emph{fitness}.

O \emph{input} do \emph{crossover} é a população resultante da seleção. Utilizei o \emph{crossover} de dois pontos, independentemente da quantidade de genes do cromossomo, com probabilidade $p_C = 90\%$. Na implementação serial, apenas um indivíduo é gerado ao término do \emph{crossover}. Isso garantiu que, na versão paralela, a chamada da função de \emph{crossover} fosse configurada com exatamente o mesmo número de \emph{threads} dos operadores anteriores (avaliação e seleção): uma \emph{thread} para cada indivíduo na população de saída, que recebe um cromossomo resultante do \emph{crossover}.

Após o \emph{crossover} todos os indivíduos passam por uma mutação simples, onde cada gene do cromossomo tem baixa probabilidade (0,01\%) de ser invertido (0 $\rightarrow$ 1 ou 1 $\rightarrow$ 0). Logo, semelhante ao cálculo do \emph{fitness}, a mutação em um dado indivíduo é independente do restante da população. Mais uma vez, uma \emph{thread} foi associada a cada \textit{\texttt{iIndividuo}} cromossomo na saída, que recebe os genes modificados do \textit{\texttt{iIndividuo}} na entrada. 

	Os experimentos foram executados em um laptop equipado com uma CPU Intel Core 2 Duo T6600 - 2,2 GHz. A placa de vídeo utilizada foi uma GeForce G 130M, com quatro multiprocessadores a 1,5 GHz e memória global total de 466 MB. A versão da API CUDA foi a 4.0, programada com o Microsoft Visual C++ Express 2008.
	
	A placa G 130M possui capacidade de computação 1.1 (que indica a versão do hardware de computação presente na GPU). Comparada com a primeira versão (arquitetura original da G80), ela adiciona suporte à operações na memória global que permitem que múltiplas \emph{threads} executem, sem conflito, operações ler-modificar-escrever na memória. Como o suporte às operações de ponto flutuante com precisão dupla só foi disponibilizado na versão 1.3, tanto o programa serial quanto o paralelo utilizaram precisão simples.
	
	As medidas de desempenho foram feitas com o objetivo de observar a influência de dois parâmetros do GA: i) número de indivíduos na população e ii) tamanho do cromossomo. O ganho na velocidade foi calculado como a razão entre o tempo de execução do programa serial e o tempo de execução do programa paralelo.

Verifiquei que o ganho de desempenho da versão paralela do GA cresce com o aumento do número de indivíduos (figura \ref{fig:ganhoNumInd}). O programa serial é mais rápido (ganho < 1) para populações pequenas (< 50). Porém, a partir de uma população de 50 indivíduos, o programa paralelo apresenta desempenho superior. Com 600 indivíduos, a versão paralela é oito vezes mais rápida para um cromossomo de tamanho 10, e dez vezes para um cromossomo de tamanho 300. 

\begin{figure}[htbp]
	\centering
		\includegraphics[width=0.70\textwidth]{figs/resultados/onemax/ganhoNumInd.png}
	\caption{ONEMAX paralelo. Ganho de velocidade em função do número de indivíduos da população.}
	\label{fig:ganhoNumInd}
\end{figure}

	Ao analisarmos a influência do tamanho do cromossomo verificamos um comportamento aproximadamente constante do ganho (figura \ref{fig:ganhoTamCromo}). Isso era esperado, pois a paralelização ocorreu no nível dos indivíduos e não no nível dos cromossomos. Com 600 indivíduos o ganho fica em torno de 9x para qualquer tamanho de cromossomo. O comportamento se repete com uma população de 10 indivíduos, mas, nesse caso, o programa serial sempre é mais rápido, mesmo para cromossomos muito pequenos (ganho sempre < 1). 
		
	\begin{figure}[htbp]
		\centering
			\includegraphics[width=0.70\textwidth]{figs/resultados/onemax/ganhoTamCromo.png}
		\caption{ONEMAX paralelo. Ganho de velocidade em função do tamanho do cromossomo.}
		\label{fig:ganhoTamCromo}
	\end{figure}
\chapter{Conclusão}
\label{cap:conclusao}

	A epígrafe dessa dissertação cita a capacidade das pessoas seguirem ``\emph{longas cadeias de raciocínio contruídas com elos cuja verdade elas não observaram diretamente}'', e que a ciência é possível justamente por esse tipo de confiança. Baseado nos argumentos do artigo \cite{metodo2004}, confiei que, ao reproduzir seu método, eu obteria o autovalor mínimo de matrizes simétricas. Para minha surpresa, concluí, pelo contrário, que seguir \cite{metodo2004} não garante o mínimo global, mas mínimos locais.
	
	A função de avaliação $f_i = e^{-\beta |\nabla \rho_i|^2}$ depende apenas do gradiente do Quociente de Rayleigh [$f_i = g(\nabla \rho_i)$] que, quando nulo ($\nabla \rho = \textbf{0}$), indica que encontramos \emph{algum} dos autovalores, e não necessariamente o menor, como afirmaram os autores.
	
	Concluí que essa impossibilidade de encontrar o autovalor mínimo de matrizes simétricas com \cite{metodo2004} não reside em uma falha do método em si, mas apenas na má definição da função de avaliação. De fato, usando o mesmo Algoritmo Genético (GA) de \cite{metodo2004}, mas com o \emph{fitness} $f_i = e^{-\beta(\rho - E_L)^2}$ de \cite{metodo2011}, foi possível, com bom ajuste do novo parâmetro $E_L$, encontrar \emph{sempre} o mínimo. Em especial, afirmo que é impossível, para esse método, obter \emph{sempre} o menor autovalor com um \emph{fitness} que dependa apenas de $\nabla \rho$.

	As duas funções de avaliação podem ser utilizadas em conjunto. A de \cite{metodo2011} encontra o autovalor mínimo, porém, é necessário conhecimento prévio sobre a região onde ele se encontra, assim como a configuração de dois parâmetros ($\beta$ e $E_L$). Apesar de não chegar ao autovalor mínimo, o \emph{fitness} de \cite{metodo2004} é mais preciso e necessita de apenas um parâmetro ($\beta$). Além disso, ele identifica diferentes autovalores intermediários, e essa informação pode ser útil para auxiliar na definição de uma boa região para o \emph{fitness} de \cite{metodo2011}.
	
	A configuração do parâmetro $\beta$ merece cuidado, mas os criadores do método não deram detalhes sobre como determiná-lo. Descobri que ele deve ser escolhido de modo que as funções de avaliação (gaussianas) sejam estreitas, punindo muito os piores indivíduos, mas ao mesmo tempo garantindo que esses, na população inicial, tenham \emph{fitness} levemente superiores à zero. Observando simetrias da matriz de Coope--Sabo, propus uma equação para $\beta$ que depende apenas da ordem da matriz. Apesar de eu ter partido do \emph{fitness} de \cite{metodo2011}, após um pequeno ajuste a equação mostrou-se válida também para \cite{metodo2004}. Isso permite que a obtenção de $\beta$ para aquelas matrizes de teste seja automática.
	
	Portanto, o método apresentado nos artigos \cite{metodo2004} e \cite{metodo2011}, que transforma o cálculo de autovalores de matrizes simétricas em um problema de Otimização Combinatória por meio de GAs, realmente funciona. No entanto, há base para contradizer, parcialmente, \cite{metodo2004}.
	
	Apesar de não ter apresentado neste trabalho uma análise formal de desempenho do meu programa, percebi que ele não é competitivo com o Scilab do ponto de vista de tempo de processamento. Entretanto, há possibilidade de reduzir esse custo computacional para matrizes grandes paralelizando o código em unidades de processamento gráfico (GPUs, veja o apêndice \ref{apdx:oneMaxNaGPU}).

% --- Finaliza a parte no bookmark do PDF, para que se inicie o bookmark na raiz ---
\bookmarksetup{startatroot}%

% ---- ELEMENTOS PóS-TEXTUAIS ----
\postextual

% ---- Referências bibliográficas ----
\bibliography{tese}

% ---- Apêndices ----
\begin{apendicesenv}
% Imprime uma página indicando o início dos apêndices
\partapendices

\chapter{Autovalores do século $\mathsf{XVIII}$ ao $\mathsf{XXI}$}\label{apdx:historia}
	
	A primeira aparição do que hoje é chamado de autovalor aconteceu em 1743 \cite{Hawkins75}. Estudando o problema de várias massas ligadas umas às outras por molas, D'Alembert chegou a um sistema de equações diferenciais. Ao fazer algumas transformações de variáveis ele foi capaz de reduzir o estudo a apenas uma equação:

\begin{equation}\label{eq:EDO1}
	\frac{d^2u}{dt^2} + \lambda u = 0,
\end{equation}
sendo $u$ uma soma envolvendo o produto das velocidades e posições de cada massa, e $\lambda$ um escalar. D'Alembert aplicou o novo método a sistemas com duas e três massas ($n = 2$ ou $n = 3$) e, com argumentos relacionados à Física do problema, afirmou que $\lambda$ só poderia ser real. A partir de então, diversos matemáticos se dedicaram ao assunto.

	Na metade do século $\mathsf{XVIII}$, D'Alembert, aproveitando trabalho anterior de Euler, demonstrou que as soluções gerais da equação \ref{eq:EDO1} são da forma $g\mathsf{e}^{-\lambda t}$, $g$ sendo um escalar, e que $\lambda$ está associado com a estabilidade do sistema massa--mola. Lagrange estende a solução para $n$ massas e escreve a equação polinomial característica, mas naquele tempo nada se sabia sobre a natureza de suas raízes. Em 1775 ele aplica seu método para a rotação de corpos rígidos desenvolvida por Euler dez anos antes, e é a primeira vez que autovalores são utilizados fora do contexto massa--mola. Em seguida, em 1778, o mesmo Lagrange mostra que a mecânica celestial pode ser escrita como um sistema de equações diferenciais, e conclui que $\lambda$ está ligado à natureza das órbitas e à estabilidade do Sistema Solar.
	
	Entra em cena Laplace, descobrindo em 1784 que $\lambda$ depende \emph{apenas} dos coeficientes $A_{ij}$ envolvidos nos sistemas de equações diferenciais. Quatro anos depois mostra que um sistema discreto de massas próximo do equilíbrio pode ser escrito como 
	
	\begin{equation}
		\mathsf{B}\mathsf{X} = \lambda \mathsf{A}\mathsf{X},
	\end{equation}
	com as matrizes $\mathsf{B}$ e $\mathsf{A}$ ligadas, respectivamente, à Energia Potencial e Energia Cinética do sistema. Embasado na Convervação da Energia, argumenta que os autovalores $\lambda$ são reais, positivos e distintos. Finalmente, em 1789, Laplace percebe as simetrias envolvidas para construir o primeiro teorema, completo e com demonstração, da natureza dos autovalores.
	
	A partir do século $\mathsf{XIX}$ o problema dos autovalores e autovetores começa a tomar a forma que conhecemos hoje. Cauchy desenvolve em 1815 a Teoria dos Determinantes e em 1829 prova, com argumentos puramente matemáticos, que os autovalores de uma matriz simétrica são reais. Matrizes simétricas são quadradas, com elementos $a_{ij}$ reais, e $a_{ij} = a_{ji}$ para $i \neq j$. Nesse mesmo ano Sturn usa autovalores na Condução de Calor, levando as aplicações para além da Mecânica Clássica. Em 1839 Cauchy cunha o termo ``Equação Característica''. Em torno de 1855 os resultados obtidos por Cauchy tornam-se ``matemática básica'' entre os matemáticos da época.
	
	No artigo \cite{autovaloresSecXX} há uma revisão sobre o desenvolvimento do cálculo de autovalores no século $\mathsf{XX}$\footnote{É importante salientar que o problema de autovalores e autovetores foi fundamental em uma das grandes revoluções científicas e culturais da nossa era, a Mecânica Quântica.}. Impulsionado pelo advento do computador eletrônico na década de 1950, o alvo desse desenvolvimento foi a criação de métodos numéricos com convergência rápida e resultados precisos. Esponho aqui os dois tipos principais, os de potência e os que reduzem a matriz principal a uma forma mais eficiente.
	
	Os Métodos de Potência (\emph{Power Methods}) são mais simples. A ideia é multiplicar a matriz $\mathsf{A}$ repetidas vezes por um vetor inicial $\mathsf{x}$ bem escolhido, de modo que um de seus componentes, o que está na direção do autovetor associado ao maior autovalor em valor absoluto, é aumentado em relação aos outros componentes. Assim, obtém-se o maior autovalor. Uma variação mais efetiva é o Método da Potência Inverso (\emph{Inverse Power Method}), que trabalha com a matriz $(\mathsf{A} - \mu \mathsf{I})^{-1}$, onde $\mu$ é um valor de deslocamento em torno de $\mathsf{A}$ a cada iteração. Tais algoritmos não são mais competitivos, mas continuam sendo estudados pois formam a base de métodos modernos.
	
	Um deles é o Método da Iteração do Quociente de Rayleigh (\emph{Rayleigh Quotient Iteration}). Inspirado num algoritmo utilizado por Lord Rayleigh em 1870, usa um quociente de Rayleigh (equação \ref{eq:rho} do capítulo \ref{cap:algebra}) para o deslocamento $\mu$. O atual é muito rápido, e possui convergência cúbica [$O(n^3)$].
		
	Com relação aos métodos de redução, todos partem da ideia central que matrizes podem ser reduzidas a uma forma mais eficiente para as computações subsequentes, utilizando um número finito de passos em transformações ortogonais. Por exemplo, foi possível aproveitar a seguinte propriedade das matrizes simétricas: para qualquer matriz simétrica $\mathsf{A}$ sempre existe uma matriz $\mathsf{Q}$ de modo a fazer uma transformação do tipo $\mathsf{Q}^{\dag} \mathsf{A} \mathsf{Q} = \mathsf{D}$, em que $\mathsf{Q}^{\dag}$ é a transposta de $\mathsf{Q}$, $\mathsf{D}$ é diagonal e seus elementos são os autovalores de $\mathsf{A}$. O Método de Jacobi, desenvolvido em 1846, faz isso por meio de uma série de rotações. Originalmente não garantia convergência, problema que foi corrigido apenas em 1949.
	
	Em 1931 Kyrlov sugeriu um método baseado no fato de que toda matriz satisfaz sua equação (ou polinômio) característica(o). Ele usou os vetores $\mathsf{x}$, $\mathsf{A}\mathsf{x}$, $\mathsf{A}^2\mathsf{x}$ (...) gerados pelo método da potência para determinar os coeficientes dessa equação. A técnica não foi bem aceita porque era instável, pois pequenas modificações em $\mathsf{A}$ levam a grandes mudanças nos coeficientes do polinômio. Entretanto, ele teve sua importância pois inspirou os famosos métodos de Householder e Lanczos. O último, por exemplo, a partir de 1980 era o preferido para grandes matrizes simétricas e esparsas (com muitos zeros).
	
	 De acordo com \cite{autovaloresSecXX}, o Método QR era um dos mais populares e mais poderosos do ano 2000. Ele é capaz de calcular \emph{todos} os autovalores e autovetores de uma matriz simétrica e densa (não esparsa), sempre  com convergência cúbica [O($n^3$)].
	
	Porém, por volta de 1970 o rumo da pesquisa na área mudou. Naquela época o problema padrão para o cálculo numérico de autovalores (equação \ref{eq:detIntro}) foi visto como essencialmente resolvido para matrizes não muito grandes ($n \leq 25$). Então, além de tratar problemas generalizados e, consequentemente, mais complexos, o interesse voltou-se para matrizes maiores.
	
	Em 1981 Cuppen apresenta o primeiro algoritmo paralelo para matrizes tridiagonais de tamanho moderado, com $n > 25$ e menor do que alguns milhares. Da classe de algoritmos do tipo ``Divida e Conquiste'', a ideia foi dividir a matriz original em dois blocos com metade do tamanho original, além de gerar uma matriz que ele chamou de Matriz de Atualização. Cuppen mostrou como o problema de autovalores para cada um dos blocos poderia ser combinado para resolver o problema principal, e reconheceu que seu algoritmo era assintoticamente muito mais rápido que o QR. Novamente, problemas de instabilidade, principalmente relacionados aos autovetores de autovalores próximos, fizeram com que o método não fosse considerado competitivo para matrizes pequenas. Entretanto, ele continuou a ser desenvolvido pois apresentava propriedades paralelas interessantes. Após uma correção publicada em 1995 o método foi aceito pela comunidade.
	
	O século $\mathsf{XX}$ chega ao seu fim com \emph{software} consolidado, seja em forma de bibliotecas para uso de programadores, seja em ambientes numéricos comerciais e de código aberto. A biblioteca LINPACK cobriu soluções numéricas para sistemas lineares, enquanto a EISPACK se concentrou nos problemas de autovalores. A EISPACK foi substituída em 1995 pela LAPACK, que possui uma versão paralela, ScaLAPACK, cuja meta é fornecer \emph{software} para arquiteturas paralelas modernas. O ambiente MATLAB, comercial, está no estado da arte da computação para álgebra linear numérica, e tornou-se padrão na década de 1990. Boas alternativas não comerciais estão disponíveis, como o Octave e o SciLab.
		
	Autovalores continuam importantes no século $\mathsf{XXI}$. A busca pela palavra \emph{eigenvalue} em periódicos como \emph{Nature} e \emph{Science} leva a vários artigos em inúmeras áreas diferentes. Restringindo a pesquisa apenas ao ano de 2015, encontramos autovalores na descoberta de novos fármacos \cite{avMedicamento2015}, cultivo de cana de açúcar na China \cite{avCana2015}, física teórica \cite{avFisTeo2015} e ciência de materiais \cite{avCienciaMateriais2015}. No jornal PLOS ONE é possível navegar por artigos associados especificamente à palavra--chave \emph{eigenvalue}\footnote{\href{http://www.plosone.org/browse/eigenvalues}{http://www.plosone.org/browse/eigenvalues}}.
	
	Mas o destaque não está limitado apenas à ciência. O algoritmo \emph{PageRank}, base do mecanisno de busca do Google, tem em seu núcleo uma formulação do problema de autovalores e autovetores \cite{BrinPage98}. Em uma versão simplificada, define-se uma matriz quadrada $\mathsf{A}$ de modo que suas linhas e colunas representam páginas da \emph{Web}. Os elementos $\mathsf{A}_{u,v}$ são definidos de tal maneira que, se não houver um \emph{hyperlink} entre $u$ e $v$, $\mathsf{A}_{u,v} = 0$, caso contrário, $\mathsf{A}_{u,v}$ é inversamente proporcional ao número total de \emph{hyperlinks} que $u$ possui apontando para quaisquer outras páginas (uma característica, então, que depende apenas de $u$). A relevância das páginas (\emph{rank}) é definida como
	
	\begin{equation}
		\mathsf{\textbf{R}} = c\mathsf{A}\mathsf{\textbf{R}},
	\end{equation}
	onde $\mathsf{\textbf{R}}$ é o autovetor de $\mathsf{A}$ com autovalor associado $c$. O objetivo é encontrar o autovetor dominante, ou seja, aquele associado ao autovalor de maior valor absoluto. Ele terá as informações da ordem de relevância das páginas associadas à busca, da mais relevante para a menos. Ou seja, a ordem das páginas exibidas em uma busca no Google é a expressão direta de $\mathsf{\textbf{R}}$ na equação acima.
	
	Outra aplicação fundamental dos autovalores na atualidade está presente na Teoria Espectral dos Grafos, que ``\textit{busca analisar propriedades estruturais
de grafos através de matrizes e seus espectros, ou seja,
dos autovalores das matrizes associadas a eles}'' \cite{TEG2014}. Um grafo (ou rede) é um conjunto de itens, chamados de vértices ou nós, com conexões entre eles, chamadas de arestas. Na figura \ref{fig:grafo} há um exemplo. Há várias matrizes associadas a um grafo, e tem-se descoberto que seus autovalores trazem informações importantes sobre a estrutura da rede.
	
	\begin{figure}[htbp]
		\centering
			\includegraphics[width=0.33\textwidth]{figs/intro/grafo.PNG}
		\caption{Exemplo de um grafo. Fonte: Wikipedia.}
		\label{fig:grafo}
	\end{figure}
	
	Vários sistemas tomam a forma de redes, como a \emph{Web} e as Redes Sociais digitais \cite{Newman2003}. No caso do Facebook e Twitter, por exemplo, as redes são enormes, atingindo facilmente centenas de milhões de nós (usuários), levando a matrizes de dimensão equivalente a essa ordem de grandeza \cite{twitter2010}. Nesses casos, extrair informações estruturais por meio dos seus autovalores é uma tarefa desafiadora.
	
	Então, acredito que a pesquisa teória e computacional, assim como das aplicações dos autovalores, continuarão ativas por um bom tempo.
\chapter{Resultados preliminares na GPU}\label{apdx:oneMaxNaGPU}
	
		O GA aqui desenvolvido buscou a solução do problema ONEMAX, cujo objetivo é encontrar uma sequência de $N$ bits com a maior quantidade possível de “1” a partir de uma sequência aleatória de “1” e “0”. O ONEMAX é especialmente indicado para o início dos estudos em GA. Além de permitir simples implementação, possui representação cromossomial binária que, junto com o crossover de ponto único, forma a base da teoria original de Holland \cite{Linden2008}.

	O programa paralelizado foi uma tradução literal do seu equivalente serial para a sintaxe do CUDA C. Ou seja, não houve nenhuma mudança estrutural no código, seja nas variáveis e estruturas de dados, seja na ordem de execução das funções e procedimentos. Apenas o preenchimento aleatório da população inicial é executado na CPU, de forma que o núcleo do programa é executado inteiramente na GPU \cite{onemaxNaGPU}. 

	A população do GA era constituída por indivíduos formados por cromossomos com \textit{\texttt{numGenes}} elementos do tipo \texttt{char}. Cada elemento do vetor (gene) podia ter um valor  “1” ou “0”. Dentro do problema ONEMAX, os melhores indivíduos foram os que apresentaram maior número de genes iguais a “1”.
		
	\begin{figure}[htbp]
		\centering
			\includegraphics[width=0.50\textwidth]{figs/resultados/onemax/onemax_objetivo.png}
		\caption{ONEMAX, um problema clássico nos Algoritmos Genéticos.}
		\label{fig:onemax_objetivo}
	\end{figure}
		
	Implementei um \emph{kernel} (função que tem sua execução feita pela GPU) para cada um dos quatro passos do GA: cálculo da função avaliadora (\emph{fitness}), seleção, \emph{crossover} e mutação. Eles são chamados um após o outro até que um número máximo de gerações seja atingido. Isso é feito dentro do \emph{loop} principal (realizado na CPU), mas sem troca de informação entre CPU e GPU.
	
	No início do programa duas gerações são alocadas na memória global da GPU, as quais são usadas alternadamente como \emph{input} e \emph{output} dos kernels. Apenas as chamadas dos kernels acontecem na CPU, enquanto o restante (execução + dados) está na GPU. Todos os \emph{kernels} tinham como \emph{input} e \emph{output} uma estrutura do tipo Geração. Ou seja, as funções operaram sobre toda a população do GA, levando-nos a adotar como estratégia o paralelismo no nível dos indivíduos.
	
	\begin{figure}[htbp]
		\centering
			\includegraphics[width=1.00\textwidth]{figs/resultados/onemax/onemax_execucao.png}
		\caption{Execução do ONEMAX paralelo. Apenas as chamadas dos kernels acontecem na CPU, enquanto o restante (execução + dados) está na GPU.}
		\label{fig:onemax_execucao}
	\end{figure}
		
	No cálculo do \emph{fitness}, o \emph{input} foi uma população com \textit{\texttt{numIndividuos}} e o \emph{output} uma população com os mesmos \textit{\texttt{numIndividuos}} e suas respectivas notas. O cálculo do \emph{fitness} de cada indivíduo foi realizado por meio da soma dos valores de seus genes. Por exemplo, para um indivíduo formado por um cromossomo de 6 genes \mbox{(\textit{\texttt{numGenes}} = 6)} com a configuração “010101”, o valor do fitness é 3 e a solução ótima para este caso seria “111111” (figura \ref{fig:onemax_objetivo}). No código serial a programação é simples e envolve apenas um laço \texttt{\textbf{\textit{for}}} que percorre o cromossomo e soma os bytes. Porém, note que toda informação necessária para esse cálculo está contida no próprio cromossomo, ou seja, a obtenção do \emph{fitness} de um dado indivíduo não depende do restante da população. Assim, a paralelização do cálculo do \emph{fitness} deu-se por meio da associação de uma \emph{thread} para cada indivíduo.
	
	Para o operador de seleção, optei pela seleção via torneio com o tamanho do torneio fixo e igual a dois. Novamente, o \emph{input} e o \emph{output} são populações. Na entrada há \textit{\texttt{numIndividuos}} e suas notas. Os mais aptos (maiores \emph{fitness}) têm maiores chances de serem selecionados, e compõem os \textit{\texttt{numIndividuos}} da população na saída. Assim como no cálculo do \emph{fitness}, o paralelismo acontece no nível dos indivíduos. Para cada indivíduo na população de saída há uma \emph{thread}, que seleciona aleatoriamente dois cromossomos na população da entrada (memória global) e fica com o de maior \emph{fitness}.

O \emph{input} do \emph{crossover} é a população resultante da seleção. Utilizei o \emph{crossover} de dois pontos, independentemente da quantidade de genes do cromossomo, com probabilidade $p_C = 90\%$. Na implementação serial, apenas um indivíduo é gerado ao término do \emph{crossover}. Isso garantiu que, na versão paralela, a chamada da função de \emph{crossover} fosse configurada com exatamente o mesmo número de \emph{threads} dos operadores anteriores (avaliação e seleção): uma \emph{thread} para cada indivíduo na população de saída, que recebe um cromossomo resultante do \emph{crossover}.

Após o \emph{crossover} todos os indivíduos passam por uma mutação simples, onde cada gene do cromossomo tem baixa probabilidade (0,01\%) de ser invertido (0 $\rightarrow$ 1 ou 1 $\rightarrow$ 0). Logo, semelhante ao cálculo do \emph{fitness}, a mutação em um dado indivíduo é independente do restante da população. Mais uma vez, uma \emph{thread} foi associada a cada \textit{\texttt{iIndividuo}} cromossomo na saída, que recebe os genes modificados do \textit{\texttt{iIndividuo}} na entrada. 

	Os experimentos foram executados em um laptop equipado com uma CPU Intel Core 2 Duo T6600 - 2,2 GHz. A placa de vídeo utilizada foi uma GeForce G 130M, com quatro multiprocessadores a 1,5 GHz e memória global total de 466 MB. A versão da API CUDA foi a 4.0, programada com o Microsoft Visual C++ Express 2008.
	
	A placa G 130M possui capacidade de computação 1.1 (que indica a versão do hardware de computação presente na GPU). Comparada com a primeira versão (arquitetura original da G80), ela adiciona suporte à operações na memória global que permitem que múltiplas \emph{threads} executem, sem conflito, operações ler-modificar-escrever na memória. Como o suporte às operações de ponto flutuante com precisão dupla só foi disponibilizado na versão 1.3, tanto o programa serial quanto o paralelo utilizaram precisão simples.
	
	As medidas de desempenho foram feitas com o objetivo de observar a influência de dois parâmetros do GA: i) número de indivíduos na população e ii) tamanho do cromossomo. O ganho na velocidade foi calculado como a razão entre o tempo de execução do programa serial e o tempo de execução do programa paralelo.

Verifiquei que o ganho de desempenho da versão paralela do GA cresce com o aumento do número de indivíduos (figura \ref{fig:ganhoNumInd}). O programa serial é mais rápido (ganho < 1) para populações pequenas (< 50). Porém, a partir de uma população de 50 indivíduos, o programa paralelo apresenta desempenho superior. Com 600 indivíduos, a versão paralela é oito vezes mais rápida para um cromossomo de tamanho 10, e dez vezes para um cromossomo de tamanho 300. 

\begin{figure}[htbp]
	\centering
		\includegraphics[width=0.70\textwidth]{figs/resultados/onemax/ganhoNumInd.png}
	\caption{ONEMAX paralelo. Ganho de velocidade em função do número de indivíduos da população.}
	\label{fig:ganhoNumInd}
\end{figure}

	Ao analisar a influência do tamanho do cromossomo, verifiquei um comportamento aproximadamente constante do ganho (figura \ref{fig:ganhoTamCromo}). Isso era esperado, pois a paralelização ocorreu no nível dos indivíduos e não no nível dos cromossomos. Com 600 indivíduos o ganho fica em torno de nove vezes para qualquer tamanho de cromossomo. O comportamento repete-se com uma população de 10 indivíduos, mas, nesse caso, o programa serial sempre é mais rápido, mesmo para cromossomos muito pequenos (ganho sempre \mbox{< 1}). 
		
	\begin{figure}[htbp]
		\centering
			\includegraphics[width=0.70\textwidth]{figs/resultados/onemax/ganhoTamCromo.png}
		\caption{ONEMAX paralelo. Ganho de velocidade em função do tamanho do cromossomo.}
		\label{fig:ganhoTamCromo}
	\end{figure}

\end{apendicesenv}

% ---- Anexos ----
%\begin{anexosenv}
% Imprime uma página indicando o início dos anexos
%\partanexos

%\include{anexoA}
%\include{anexoB}

%\end{anexosenv}

% ---- INDICE REMISSIVO ----
%\printindex

\end{document} 