\chapter{Materiais e Métodos}
\label{cap:metodologia}

Antes de atacar o método em si foi necessário estudo de Algoritmos Genéticos e Linguagem C, escolhida pensando em CUDA.

\section{CUDA}

Metodologia: revisão bibliográfica e aplicação no ONEMAX.

Material do ERAD.

\section{Método}

	Artigo de 2004.	Aproveitar conteúdo do segundo relatório de estudos dirigidos.

	Apresentar o fitness \textit{fitness} com $\rho - \rho_0$ utilizado no Artigo de 2006.

	Apresentar e definir a matriz de Coope-Sabo (citar artigo original de 77), utilizada no pelos indianos no artigo de 2006.
	
\section{Framework/Programa Serial}

	Optou-se por desenvolver do zero todo o programa. Motivo: controle sobre todas as características do método.

	Linguagem C: linguagem nativa para CUDA.

	Totalmente parametrizado. 

	Com foi será possível:

	\begin{enumerate}
		\item Estudo do método 
		\item Estudo de algoritmos genéticos
		\item Novas aplicações como máximo de função
		\begin{enumerate}
			\item Mudança no fitness 
			\item Mudança nos critérios de parada
		\end{enumerate}	
	\end{enumerate}


	Descrição dos parâmetros.

	Descrição de cada função, incluindo \textit{print screen} das partes de código mais fundamentais:

	\begin{enumerate}
		\item Estruturas de dados
		\item Geração de números pseudoaleatórios
		\item Geração das Matrizes de Coope
		\item Geração da População Inicial
		\item Fitness: as várias equações
		\item Fitness: cálculo de $\rho_i$
		\item Fitness: cálculo de $\nabla\rho_i$
		\item Seleção
		\item Crossover
		\item Mutação
		\item Álgebra Linear: multiplicação de matrizes
		\item Álgebra Linear: multiplicação de matriz por escalar
		\item Álgebra Linear: subtração de matrizes
	\end{enumerate}

	Qualidade dos números pseudo-aleatórios (base do GA)
	
	

	\begin{enumerate}
			\item Mostrar que a distribuição dos números segue o esperado para números aleatórios
			\item Quanto maior a quantidade de pontos, melhor a distribuição
			\item Gráfico: histograma de frequência.
	\end{enumerate}

	Exemplo de execução no Windows.

	Reprodutibilidade.
	
	Exemplo de reprodutibilidade.

	Código disponível em

	\texttt{https://github.com/prietoab/msc\_code}