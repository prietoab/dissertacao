\chapter{Formato matricial da Equação de Schrödinger independente do tempo}

Página 36 do \cite{Cohen1}.

-------------------------------

Equação de Schrödinger com potencial dependente do tempo:
	
	\begin{equation}
		i\hbar\frac{\partial}{\partial t} \psi(\textbf{r},t) =  - \frac{\hbar^2}{2m}\Delta\psi(\textbf{r},t) + V(\textbf{r},t)\psi(\textbf{r},t),
	\end{equation}
	onde $\Delta$ é o operador Laplaciano $\partial / \partial x^2 + \partial / \partial y^2 + \partial / \partial z^2$.
	
	
	Equação de Schrödinger com potencial independente do tempo:
	
	\begin{equation}\label{eq:SchrodingerIndTempo}
		i\hbar\frac{\partial}{\partial t} \psi(\textbf{r},t) =  - \frac{\hbar^2}{2m}\Delta\psi(\textbf{r},t) + V(\textbf{r})\psi(\textbf{r},t)
	\end{equation}
	
	Existe soluções na forma
	
	\begin{equation}
		\psi(\textbf{r},t) = \phi(\textbf{r})\chi(t)
	\end{equation}
	
	A função 
	
	\begin{equation}\label{eq:solucaoSchro}
		\psi(\textbf{r},t) = \phi(\textbf{r})\mbox{e}^{-i \omega t}
	\end{equation}
	é uma solução da equação \ref{eq:SchrodingerIndTempo}, chamada de Solução Estacionária da Equação de Schrödinger. Ela leva a uma densidade de probabilidade $|\psi(\textbf{r},t)|^2 = |\phi(\textbf{r})|^2$ independente do tempo. Dizemos que as variáveis de espaço e tempo foram separadas.
	
	A função $\phi(\textbf{r})$ deve satisfazer a equação
	
	\begin{equation}\label{eq:SchroNoR}
		- \frac{\hbar^2}{2m} \Delta \phi(\textbf{r}) + V(\textbf{r}) \phi(\textbf{r}) = \hbar \omega \phi(\textbf{r})
	\end{equation}
	
	A equação \ref{eq:SchroNoR} pode ser reescrita como
	
	\begin{equation}\label{eq:SchroNoRReescrita}
			\left[-\frac{\hbar^2}{2m}\Delta + V(\textbf{r})\right] \phi(\textbf{r}) = \hbar \omega \phi(\textbf{r})
	\end{equation}
	
	Definindo 
	
	\begin{equation}
	H  = -\frac{\hbar^2}{2m}\Delta + V(\textbf{r}),
	\end{equation}
	a equação \ref{eq:SchroNoRReescrita} fica
	
	\begin{equation}\label{eq:H_com_hw}
		H\phi(\textbf{r}) = \hbar \omega \phi(\textbf{r})
	\end{equation}
	
	De acordo com as relações de Planck-Einstein, um estado estacionário é um estado com energia bem definida e com valor $E = \hbar \omega$ ($\omega = 2 \pi \nu$). Finalmente, a equação \ref{eq:H_com_hw} fica
	
	\begin{equation}\label{eq:Schro_Indepentende_do_Tempo}
		H\phi(\textbf{r}) = E \phi(\textbf{r}).
	\end{equation}
	
	A equação \ref{eq:Schro_Indepentende_do_Tempo} é, então, uma equação de autovalores para o operador linear $H$: a aplicação de $H$ à autofunção $\phi(\textbf{r})$ leva à mesma função, multiplicada pelo autovalor correspondente $E$. As energias permitidas são, portanto, os autovalores do operador $H$. Está associada à origem da quantização da energia.
	
	