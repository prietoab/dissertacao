%ARQUIVO DE PREAMBULO DA TESE - PACOTES E CONFIGURAÇÕES

\documentclass[
	% -- opções da classe memoir --
	12pt,				% tamanho da fonte
	openright,			% capítulos começam em pág ímpar (insere página vazia caso preciso)
	oneside,			% para impressão em verso e anverso. Oposto a oneside. twoside
	a4paper, %letterpaper,		% tamanho do papel.
	% -- opções da classe abntex2 --
	%chapter=TITLE,		% títulos de capítulos convertidos em letras maiúsculas
	%section=TITLE,		% títulos de seções convertidos em letras maiúsculas
	%subsection=TITLE,	% títulos de subseções convertidos em letras maiúsculas
	%subsubsection=TITLE,% títulos de subsubseções convertidos em letras maiúsculas
	% -- opções do pacote babel --
	english,			% idioma adicional para hifenização
	%french,			% idioma adicional para hifenização
	%spanish,			% idioma adicional para hifenização
	brazil,				% o último idioma é o principal do documento
	sumario=tradicional,
	]{abntex2}

	% ---
	% Pacotes fundamentais
	% ---
	\usepackage{cmap}				% Mapear caracteres especiais no PDF
	\usepackage{lmodern}			% Usa a fonte Latin Modern		
	\usepackage[T1]{fontenc}		% Selecao de codigos de fonte.
	%\usepackage[latin1]{inputenc}	% Codificacao do documento (conversão automática dos acentos)
	\usepackage[utf8]{inputenc}		% Codificacao do documento (conversão automática dos acentos)
	\usepackage{lastpage}			% Usado pela Ficha catalográfica
	\usepackage{indentfirst}		% Indenta o primeiro parágrafo de cada seção.
	\usepackage{color}				% Controle das cores
	\usepackage[pdftex]{graphicx}	% Inclusão de gráficos
	\usepackage{epstopdf}           % Pacote que converte as figuras em eps para pdf
	% ---
	\usepackage{hyphenat} % para deixar de fazer a hifenização em trechos específicos
	% ---
	% Pacotes adicionais, usados apenas no âmbito do Modelo Canônico do abnteX2
	% ---
	\usepackage{nomencl}
	\usepackage{amsmath}
	\usepackage{bbm}
	\usepackage[chapter]{algorithm}
	\usepackage{algorithmic}
	\usepackage{multirow}
	\usepackage{rotating}
	\usepackage{pdfpages}
	% ---

	% ---
	% Pacotes de citações
	% ---
	%\usepackage[brazilian,hyperpageref]{backref}	 % Paginas com as citações na bibl
	\usepackage[alf,abnt-etal-cite=2,abnt-etal-list=0,abnt-etal-text=emph]{abntex2cite}	% Citações padrão ABNT

	% ---
	% Pacote de customização - Unicamp
	% ---
	\usepackage{unicamp}
	
	% ---
	% CONFIGURAçõES DE PACOTES
	% ---

	% ---
	% Configurações do pacote backref
	% Usado sem a opção hyperpageref de backref

	%customização do negrito em ambientes matemáticos
	\newcommand{\mb}[1]{\mathbf{#1}}

	%customização de teoremas
	\newtheorem{mydef}{Definição}[chapter]
	\newtheorem{lemm}{Lema}[chapter]
	\newtheorem{theorem}{Teorema}[chapter]
	\floatname{algorithm}{Pseudoc\'{o}digo}
	\renewcommand{\listalgorithmname}{Lista de Pseudoc\'{o}digos}

	% Para mudar fonte dos títulos de capítulos, de secões, etc  para padrão Latex (fonte cmr) descomente a linha a seguir. 
	%\renewcommand{\ABNTEXchapterfont}{\fontfamily{cmr}\fontseries{b}\selectfont}

	% ---
	% Configurações de aparência do PDF final

	% alterando o aspecto da cor azul
	\definecolor{blue}{RGB}{41,5,195}



	% informações do PDF
	\makeatletter
	\hypersetup{
				%pagebackref=true,
			pdftitle={}, %\@title},
			pdfauthor={\@author},
				pdfsubject={\imprimirpreambulo},
				pdfcreator={LaTeX com abnTeX2},
			pdfkeywords={abnt}{latex}{abntex}{abntex2}{trabalho acad\^{e}mico},
			%hidelinks,					% desabilita as bordas dos links
			colorlinks=true,       	    % false: boxed links; true: colored links
			% aqui vc pode trocar a cor dos links do sumário, lista de figs, etc.
				linkcolor=red,          	% color of internal links
				citecolor=blue,        		% color of links to bibliography
				filecolor=magenta,      	% color of file links
			urlcolor=blue,
	%		linkbordercolor={1 1 1},	% set to white
			bookmarksdepth=4
	}
	\makeatother
	% ---

	% ---
	% Espaçamentos entre linhas e parágrafos
	% ---

	% O tamanho do parágrafo é dado por:
	\setlength{\parindent}{2.0cm}

	% Controle do espaçamento entre um parágrafo e outro:
	\setlength{\parskip}{0.2cm}  % tente também \onelineskip

	% ---
	% Informacoes de dados para CAPA e FOLHA DE ROSTO
	% ---
	\titulo{Título da Dissertação/Tese}
	\autor{Adriano Batista Prieto}
	\local{Limeira}
	\data{2015}
	\orientador{Prof. Dr. Vitor Rafael Coluci}
	%\coorientador[Co-orientador]{Prof. Dr. Co-orientador}
	\instituicao{
			UNIVERSIDADE ESTADUAL DE CAMPINAS
			\par
			Faculdade de Tecnologia
			}

	%\tipotrabalho{Doutorado}	
	%\preambulo{ Dissertação/Tese apresentada à Faculdade/Instituto da Universidade Estadual de Campinas como parte dos requisitos exigidos para a obtenção do título de Mestre(a)/Doutor(a) em <NOME DO TÍTULO>, na Àrea de <NOME DA ÁREA> (para teses/dissetações redigidas em português)}

	\tipotrabalho{Dissertação (Mestrado)}
	\preambulo{ \textit{\nohyphens{Dissertação apresentada à Faculdade de Tecnologia da Universidade Estadual de Campinas como parte dos requisitos exigidos para a obtenção do título de Mestre em Tecnologia, na área de Tecnologia e Inovação.}}}
	% ---